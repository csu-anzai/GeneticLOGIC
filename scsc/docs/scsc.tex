%%%\documentstyle[us,11pt,c++,fancyfigures,harvard]{article}  %% US letter paper
\documentstyle[a4,11pt,c++,fancyfigures,harvard]{article}  %% European A4 paper


%% -=-=-=-=-=-=-=-=-=-=-=-=-=-=-=-=-=-=-=-=-=-=-=-=-=-=-=-=-=-=-=-=-=-=-=-=-=-
%%  PREAMBLE
%% -=-=-=-=-=-=-=-=-=-=-=-=-=-=-=-=-=-=-=-=-=-=-=-=-=-=-=-=-=-=-=-=-=-=-=-=-=-
\renewcommand{\textfraction}{0.05}
\renewcommand{\topfraction}{0.95}
\parskip=0.5\baselineskip
\parindent=0pt
\fancyovalframe


%% -=-=-=-=-=-=-=-=-=-=-=-=-=-=-=-=-=-=-=-=-=-=-=-=-=-=-=-=-=-=-=-=-=-=-=-=-=-
%%  TITLEPAGE
%% -=-=-=-=-=-=-=-=-=-=-=-=-=-=-=-=-=-=-=-=-=-=-=-=-=-=-=-=-=-=-=-=-=-=-=-=-=-
\title{	SCS-C:							\\
	A C-language Implementation of a			\\ 
	Simple Classifier System}

\author{{\bf J\"org Heitk\"otter}				\\
	\verb+<joke@ls11.informatik.uni-dortmund.de>+		\\
	\							\\
	University of Dortmund					\\
	Department of Computer Science				\\
	Systems Analysis Research Group, LSXI			\\
	44221 Dortmund						\\
	Germany							\\
}

\begin{document}
\maketitle
\thispagestyle{empty}


%% -=-=-=-=-=-=-=-=-=-=-=-=-=-=-=-=-=-=-=-=-=-=-=-=-=-=-=-=-=-=-=-=-=-=-=-=-=-
%%  ABSTRACT
%% -=-=-=-=-=-=-=-=-=-=-=-=-=-=-=-=-=-=-=-=-=-=-=-=-=-=-=-=-=-=-=-=-=-=-=-=-=-
\begin{abstract}
This paper describes a C implementation of a Simple Classifier System,
thus called SCS-C.
\end{abstract}

\newpage
%% -=-=-=-=-=-=-=-=-=-=-=-=-=-=-=-=-=-=-=-=-=-=-=-=-=-=-=-=-=-=-=-=-=-=-=-=-=-
\section{Introduction}

SCS-C\footnote{{\bf Disclaimer:} SCS-C is distributed under the terms
described in the {\tt LICENSE} file, accompanying the sources of this
program. This license is also known as the GNU General Public License.
This means that SCS-C has no warranty implied or given, and that the
authors assume no liability for damage resulting from its use or misuse.
Try also `{\tt scsc --warranty}'.} 
is a C-language translation and extension of the original
Pascal SCS code presented by Goldberg \citeyear{Goldberg:89e}. 
It has some additional features, but its operation is 
essentially the same as that of the original, Pascal version. 
This report is included as an overview and reference to the SCS-C distribution.
It is presented with the assumptions that the reader has a general
understanding of Goldberg's original Pascal SCS code, and a good working
knowledge of the C programming language.

The report begins with an overview of the files included in the SCS-C
distribution, but not the routines they contain. 
Then follows a discussion of significant features of SCS-C
that differ from those of the Pascal version.
For the occasional ftp-user, we provide an example session to enable
the reader to receive the latest version of SCS-C via anonymous ftp.


%% -=-=-=-=-=-=-=-=-=-=-=-=-=-=-=-=-=-=-=-=-=-=-=-=-=-=-=-=-=-=-=-=-=-=-=-=-=-
\section{The SCS-C distribution}
\label{dist}

The following gives a quick overview of the files distributed
with SCS-C:

%%For a concise explanation of the contents of each file, refer
%%to the companion {\tt OVERVIEW} file (cf.~Figure~\ref{fig-dist}). 

Figure~\ref{fig-dist} displays the directory tree of SCS-C at a glance.
\psFancyFigure*{fig-dist}[height=0.8\textheight]{The SCS-C distribution
at a glance.}
Figure~\ref{fig-cont} zooms into SCS-C's {\tt ./contrib} folder.
\psFancyFigure*{fig-cont}[]{The {\tt contrib} folder:
various useful contributions to play with.}
\psFancyFigure*{fig-figs}[]{The {\tt figs} folder contains the figures,
generated for the {\tt docs/repro.tex} document; a reproduction of
figures 6.18 to 6.22 in Goldber'g book.}
While Figure~\ref{fig-figs} zooms into SCS-C's {\tt ./figs} folder.


%% -=-=-=-=-=-=-=-=-=-=-=-=-=-=-=-=-=-=-=-=-=-=-=-=-=-=-=-=-=-=-=-=-=-=-=-=-=-
\section{Configuring SCS-C for Your System}
\label{config}

SCS-C is distributed in gzip'ed tar form, ie. the complete directory
tree has been glued together into one file using the Unix {\tt tar}
command and afterwards this single file has been compressed with the
GNU standard {\tt gzip} utility.\footnote{See the Unix manul pages
of tar(1), and gzip(1) respectively.}
After {\em gunzip\/}ing and un{\em tar\/}ing SCS-C, you have to
{\em configure} it for your system.

%%\clearpage
Configuration means, in this case, supplying operating system dependencies,
compiler dependencies, and some other, ie. maintenance information,
that have been encapsulated from the SCS-C code
into platform specfic configuration files, collected in the {\tt ./config}
folder. All there is to do is (1) find the appropriate configuration file
and (2) make this file known to the rest of the source code.\footnote{%
If there is no appropriate configuration file, we think it's very easy
to generate one for your system, using eg. {\tt bsd.cf} as a template file.}
The latter can either be done manually, ie. by copying the appropriate
file into the {\tt ./src} folder, renaming it to {\tt config.h}.
For example on a DOS machine all there is to do is to issue:
\begin{verbatim}
  C:\SCS-C> copy config\dos.cf src\config.h
  C:\SCS-C> make
\end{verbatim}
For unixoid systems\footnote{that have either a {\tt csh} or {\tt tcsh}
installed} we provide a simple {\tt configure} shell script, with the
following command syntax:
\begin{verbatim}
  example% configure
  configure: usage: configure --<os-type>    (eg. configure --sunos)

  example% configure --unix
  configure: try either: configure --bsd, or: configure --sys5

  example% configure --bsd
  configure: SCS-C configured for generic BSD unix.
  example% make
\end{verbatim}
In the end it's up to you which way you prefer. The {\tt configure} script
merely {\em softlink\/}s a selected configuration file to
{\tt ./src/config.h}.

Please refer to the files {\tt INSTALL}, {\tt CONFIGME}, and {\tt Makefile}
for more concise information.


%% -=-=-=-=-=-=-=-=-=-=-=-=-=-=-=-=-=-=-=-=-=-=-=-=-=-=-=-=-=-=-=-=-=-=-=-=-=-
\section{New Features of SCS-C}
\label{features}

SCS-C has several features that differ from those of the Pascal version.
One is the ability to use various runtime options on the command line, i.e.
{\tt scsc --help} results in the following message:
{\small
\begin{verbatim}
  usage: scsc [options]
         [-N, --no-noise-in-auction]
         [-S, --no-specificity-in-auction]
         [-b, --batch]
         [-h, --help]
         [-v, --version]
         [-w, --warranty]

         [-c, --classifier-data <classifier data file name>]
         [-d, --detector-data <detector data file name>]
         [-e, --environmental-data <environmental data file name>]
         [-g, --genetics-data <GA data file name>]
         [-r, --reinforcement-data <reinforcement data file name>]
         [-t, --timekeeper-data <timekeeper data file name>]

         [-l, --log-file <log file name>]
         [-p, --plot-file <plot file name>]

         [-s, --seed-for-random <seed value>]
         [-T, --time-steps <number>]
\end{verbatim}
}

Another new feature of SCS-C is its method of reading command line parameters
from so-called {\em resource} files.\footnote{Unfortunately this feature
is not included in the initial releases of SCS-C, ie. version numbers below
1.00.}


%% -=-=-=-=-=-=-=-=-=-=-=-=-=-=-=-=-=-=-=-=-=-=-=-=-=-=-=-=-=-=-=-=-=-=-=-=-=-
\section{Summary}
\label{summary}

SCS-C is intended to be a simple program for first-time LCS experimentation.
It is not intended to be definitive in terms of its efficiency, but
some work has been put into the grace of its implementation. To be honest,
I think it's really nifty stuff.
And it should be very straight forward to tune SCS-C towards an
object-oriented implemenation, for example by using \C++.
The key issues of the current design can be summarized as:
\begin{itemize}
\item
portability, by using customizable and easy-to-write configuration
{\tt ./config/ *.cf} files
\item
configurability, also related to configuration files, and the
companion {\tt configure} C-shell script
\item
extendability, by providing source-documented hooks and lots of
LOF's simply included for encapsulation and prototyping\footnote{%
The {\sc VAX} version of SCS placed in the {\tt ./contrib/vax-scs} folder
consist of approx. 1,800 lines, SCS-C has some 5,000 lines of code (LOF),
leaving alone the documention, and random number generator examples in
{\tt ./contrib/knuth}.}
\item
accuracy, by providing a sample document {\tt ./docs/repro.tex} that
reproduces some figures from Chapter~6 of David Goldberg's book.
\end{itemize}

The author is interested in the comments, criticisms, and bug reports
from SCS-C users, so that the code can be refined for easier use in
subsequent versions.

%%\clearpage
Please email your comments to \verb+<joke@ls11.informatik.uni-dortmund.de>+
or write to:
\begin{center}
J\"org Heitk\"otter				\\
c/o Systems Analysis Research Group, LSXI	\\
University of Dortmund				\\
Department of Computer Science			\\
44221 Dortmund, Germany				\\
\end{center}


%% -=-=-=-=-=-=-=-=-=-=-=-=-=-=-=-=-=-=-=-=-=-=-=-=-=-=-=-=-=-=-=-=-=-=-=-=-=-
\section{Availability}

The SCS-C sources and their documentation are available through the
{\em anonymous ftp} service of the Systems Analysis Research Group from
server {\tt lumpi.informatik.uni- dortmund.de} (129.217.36.140).
Get file `scsc-0.99j.tar.gz' from {\tt /pub/LCS/src}.

What follows, is a short demo session for the occasional ftp user:
{\footnotesize
\begin{verbatim}
    example% ftp lumpi.informatik.uni-dortmund.de
    Connected to lumpi.
    220 lumpi FTP server (Version 6.14 Fri Feb 21 18:11:53 MET 1992) ready.
    Name (lumpi:heitkoet): anonymous
    331 Guest login ok, send e-mail address as password.
    Password: <your email address goes here>
    230-
    230-  Welcome to the official ES/GA FTP server at the University of Dortmund.
    230-
    230-  Local time here is Mon Feb  1 12:11:26 1993
    230-
    230-  --> This is an unsecure service now. It may be terminated at any
    230-  --> time without prior announcement.
    230-
    230-  Make sure to transfer the files in binary mode!
    230-
    230-  Ulrich Hermes, hermes@ls11.informatik.uni-dortmund.de
    230-  Frank Wiesenfeller, fw@irb.informatik.uni-dortmund.de (FTP)
    230-
    230-
    230 Guest login ok, access restrictions apply.
    ftp> dir
    200 PORT command successful.
    150 Opening ASCII mode data connection for /bin/ls.
    total 9
    -rw-------  1 11002 11000     0 Jul  7  1992 .hushlogin
    drwxr-xr-x  5 11002 11      512 Apr 27  1992 EA
    drwxr-xr-x  5 11002 11      512 Apr 27  1992 ES
    drwxr-xr-x  5 11002 11      512 Jul  7  1992 GA
    drwxr-xr-x  5 11002 11      512 Feb  8  1993 LCS
    -rw-r--r--  1 11002 11      863 Jan 13 06:40 README
    -rw-r--r--  1 11002 11       63 Jan 17 16:37 _THIS_SITE_IS_IN_GERMANY,_EUROPE
    drwxr-xr-x  2 11002 11     1024 Dec 27 15:34 idl
    drwxr-xr-x  3 11002 11      512 Nov 10 06:47 idl-pvwave
    -rw-r--r--  1 11002 11     1864 Jan 23 15:01 ls-Ral.Z
    226 Transfer complete.
    588 bytes received in 0.43 seconds (1.3 Kbytes/s)
    ftp> cd pub/LCS/src
    ftp> binary
    200 Type set to I.
    ftp> get scsc-0.99j.tar.gz
    [..]
    ftp> bye
    221 Goodbye.
\end{verbatim}
}

{\bf Note:} version numbers are subject to change, due
to the nature of the beast, so please don't get confused if it's not
0.99j on your visit.


%% -=-=-=-=-=-=-=-=-=-=-=-=-=-=-=-=-=-=-=-=-=-=-=-=-=-=-=-=-=-=-=-=-=-=-=-=-=-
\section*{Acknowledgments}

The author gratefully acknowledge support provided by Uwe Schnepf
of the AI Research Division of the GMD, and Thomas B\"ack
of the Systems Analysis Research Group, over the course of this project.

He also greatly appreciates the work of David E. Goldberg
on the Pascal version of SCS, and says `Thanks!' to  Uwe Schnepf for his
unforseen enthusiasm and `for lending me a Sun to hack this stuff together,'
during a nightshift week January
11--16, 1993, at the AI research division of the GMD.

Additional thanks must go to Monika Heitk\"otter, for hosting me during this
week.


%% -=-=-=-=-=-=-=-=-=-=-=-=-=-=-=-=-=-=-=-=-=-=-=-=-=-=-=-=-=-=-=-=-=-=-=-=-=-
\bibliographystyle{kluwer}
\bibliography{thebib}

\end{document}
