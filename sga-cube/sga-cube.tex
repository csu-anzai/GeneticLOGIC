\documentstyle[apaua]{article}
\pagestyle{empty}
\oddsidemargin=0in
\evensidemargin=0in
\topmargin=0.5in
\textheight=9in
\textwidth=6.5in 
\footheight=0in
\columnsep=0.25in
\headsep=0in
\headheight=0in


\def\btt#1{\bf{\tt #1}}

\begin{document}
\bibliographystyle{apaua}

\begin{titlepage}
\begin{center}
\vspace*{2.35in}
SGA-Cube: A Simple Genetic Algorithm for  \\
nCUBE 2 Hypercube Parallel Computers
\ \\
by\\
\ \\
Jeff A. Earickson\\
Alabama Supercomputer Network\\
\ \\
Robert E. Smith\\
The University of Alabama\\
and\\
David E. Goldberg\\
The University of Illinois\\
\vspace{0.7in}
TCGA Report No. 91005\\
\today \\
\vspace{2.2in}
The Clearinghouse for Genetic Algorithms\\
The University of Alabama\\
Department of Engineering Mechanics\\
Tuscaloosa, AL 35487
\end{center}
\end{titlepage}

\title{SGA-Cube: A Simple Genetic Algorithm for  \\
nCUBE 2 Hypercube Parallel Computers}
\author{ {\bf Jeff A. Earickson}\\
Alabama Supercomputer Network\\
The Boeing Company\\
Huntsville, Alabama 35806
\and {\bf Robert E. Smith}\\
The University of Alabama\\
Department of Engineering Mechanics\\
Tuscaloosa, Alabama 35405 
\and {\bf David E. Goldberg}\\
The University of Illinois\\
Department of General Engineering\\
Urbana, Illinois 61801
}

\maketitle

\section{Introduction}
SGA-Cube\footnote{{\bf Disclaimer:} SGA-Cube is distributed under the terms described
in the file {\btt{NOWARRANTY}}. These terms are taken from the GNU General Public
License.
This means that SGA-Cube has no warranty implied or given, and that the
authors assume no liability for damage resulting from its use or misuse.} 
is a C-language translation of the original
Pascal SGA code presented by Goldberg \citeyear{Goldberg:89e}
with modifications to allow execution on the nCUBE 2 Hypercube Parallel
Computer. 
When run on the nCUBE 2,
SGA-Cube takes advantage of the hypercube architecture, and is scalable to
any hypercube dimension. 
The hypercube implimentation is modular, so that the algorithm for
exploiting parallel processors can be easily modified.
In addition to its parallel capabilities, SGA-Cube can be
compiled on various serial computers via compile-time options.
In fact, when compiled on a serial computer, SGA-Cube is essentially
identical to SGA-C \cite{Smith:91}.
SGA-Cube has been nominally
tested on a Sun 470 workstation, a Vax Ultrix system, a Cray X-MP/24 
running UNICOS 5.1, and the nCUBE 2.

This report is included as a concise introduction to the SGA-Cube distribution.
It is presented with the assumptions that the reader has a general understanding
of Goldberg's original Pascal SGA code, and a good working knowledge of the
C programming language.
The report begins with an outline of the files included in the SGA-Cube
distribution, and the routines they contain. 
The outline is followed by a discussion of the 
significant features of SGA-Cube.
The report concludes with a
discussion of routines that must be altered to
implement one's own application in SGA-Cube.

\section{Files Distributed with SGA-Cube}
\label{files}
The following is an outline of the files distributed 
with SGA-Cube, the routines contained in those files, and the
{\btt{include}} structure of the SGA-Cube distribution.
\begin{description}
\item[{\btt{sga.h}}] contains declarations of global variables and structures 
for SGA-Cube.  This file is included by {\btt{main()}}.
Both {\btt{sga.h}} and {\btt{external.h}} have two
{\btt{defines}} set at the top of the files; {\btt{LINELENGTH}}, which determines the
column width of printed output, and {\btt{BITS\_PER\_BYTE}}, which specifies the number
of bits per byte on the machine hardware.  {\btt{LINELENGTH}} can be set to any
desired positive value, but {\btt{BITS\_PER\_BYTE}} must be set to the correct value for your
hardware.
\item[{\btt{external.h}}] contains external declarations for inclusion
in all source code files except {\btt{main()}}.  The {\bf extern} declarations in {\btt{external.h}}
should match the declarations 
in {\btt{sga.h}}.
\item[{\tt main.c}] contains the main SGA program loop, 
{\btt{main()}}. 
\item[{\btt{exchange.c}}] contains the code needed to exchange
individuals between hypercube nodes.
As distributed, SGA-Cube exchanges  the fittest {\btt{nexchange}} 
individuals from each node with a nearest-neighbor node.  
For a more 
detailed explanation of the algorithm see Section~\ref{parallel}.
To impliment another exchange policy, change this file.
\begin{description}
\item[{\btt{exchange()}}]  sends the {\btt{nexchange}} fittest individuals 
to the
nearest neighbor in the specified hypercube dimension, and receives the 
neighbor's
corresponding fittest individuals.  The node's least fit individuals 
are replaced
by the received individuals.
\item[{\btt{nearest()}}] determines a node's {\btt{n}} nearest neighbors in an
n-dimensional hypercube.
\end{description}
\item[{\btt{generate.c}}] contains {\btt{generation()}}, a routine which generates and 
evaluates a new GA population.  
\item[{\btt{initial.c}}] contains routines that are called at the beginning 
of a GA run.
\begin{description}
\item[{\btt{initialize()}}] is the central initialization routine called by 
{\btt{main()}}.
\item[{\btt{initdata()}}] is a routine to prompt the user for SGA parameters.
\item[{\btt{initpop()}}] is a routine that generates a random population. Currently, 
SGA-Cube includes no facility for using seeded populations.
\item[{\btt{initreport()}}] is a routine that prints a report after initialization 
and before the first GA cycle.
\end{description}
\item[{\btt{memory.c}}] contains routines for dynamic memory management.  
\begin{description}
\item[{\btt{initmalloc()}}] is a routine that dynamically allocates space for the GA 
population and other necessary data structures.
\item[{\btt{freeall()}}] frees all memory allocated by {\btt{initmalloc()}}.
\item[{\btt{nomemory()}}] prints out a warning statement
when a call to {\btt{malloc()}} fails.
\end{description}
\item[{\btt{operators.c}}]  contains the routines for genetic operators.
\begin{description}
\item[{\btt{crossover()}}] performs single-point crossover on two mates, producing
two children.
\item[{\btt{mutation()}}] performs a point mutation.
\end{description}
\item[{\btt{random.c}}] contains random number utility programs, including:
\begin{description}
\item[{\btt{randomperc()}}] returns a single, uniformly-distributed, real, 
pseudo-random number between 0 and 1. 
This routine uses the subtractive method specified by Knuth \citeyear{Knuth:81}.
\item[{\btt{rnd(low,high)}}] returns an uniformly-distributed integer between 
{\btt{low}} and {\btt{high}}.
\item[{\btt{rndreal(low,high)}}] returns an uniformly-distributed floating point number 
between  {\btt{low}} and {\btt{high}}.
\item[{\btt{flip(p)}}] flips a biased coin, returning 1 with probability {\btt{p}}, 
and 0 with probability {\btt{1-p}}.
\item[{\btt{advance\_random()}}] generates a new batch of 55 random numbers.
\item[{\btt{randomize()}}] asks the user for a random number seed.
\item[{\btt{warmup\_random()}}] primes the random number generator.
\item[{\btt{noise(mu, sigma)}}] generates a normal random variable 
with mean {\btt{mu}} and standard
deviation {\btt{sigma}}.  This routine is not currently used in SGA-Cube, and is only included as a general utility.
\item[{\btt{randomnormaldeviate()}}] is a utility routine used by noise.
It computes a standard normal random variable.
\item[{\btt{initrandomnormaldeviate()}}] is an initialization routine for
{\btt{randomnormaldeviate()}}.
\end{description}
\item[{\btt{report.c}}] contains routines used to print a report from each cycle of SGA-C's operation.
\begin{description}
\item[{\btt{report()}}] controls overall reporting.
\item[{\btt{writepop()}}] writes out the population at every generation.
\item[{\btt{writechrom()}}] writes out the chromosome as a string of ones and zeroes.
In the current implementation, the most significant bit is the {\it rightmost} bit.
\item[{\btt{printnodes()}}] is for use with the nCUBE version only. This routine
cycles sequentially through all hypercube nodes, performing whatever printing
function was passed to it.  Used to insure that printed output from multiple
nodes comes out in order.
\item[{\btt{thebest()}}] is for use with the nCUBE version only. This routine
queries all nodes
for their best current individual, compares each of them to current best individual
found so far, and then broadcasts the global best to every node.
\item[{\btt{writestats()}}] is for use with the nCUBE version only. This routine writes out the
per-node population statistics.
\end{description}
\item Three selection routines are included with the SGA-Cube distribution:
\begin{description}
\item[{\btt{rselect.c}}] contains routines for roulette-wheel selection.  
\item[{\btt{srselect.c}}] contains the routines for stochastic-remainder selection 
\cite{Booker:82}.
\item[{\btt{tselect.c}}] contains the routines for tournament selection 
\cite{Brindle:81a}.  Tournaments of any size up to the population size can be held
with this implementation\footnote{The tournament selection routine
included with the distribution was written by Hillol Kargupta, of the University
of Alabama.}.
\end{description}
For modularity, each selection method 
is made available as a compile time option.
Edit {\btt{Makefile}} to choose a selection method. Each of the three
selection files 
contains the routines
{\btt{select\_memory}} and {\btt{select\_free}} (called by {\btt{initmalloc}} and {\btt{freeall}}, respectively), which perform
any necessary auxiliary memory handling,
and the routines {\btt{preselect()}} and {\btt{select()}}, 
which implement the particular selection method.
\item[{\btt{stats.c}}] contains the routine {\btt{statistics()}}, which calculates 
population statistics for each generation.
\item[{\btt{utility.c}}] contains various utility routines. Of particular interest 
is the routine {\btt{ithruj2int()}}, which returns bits $i$ through $j$ of a 
chromosome interpreted as an {\btt{int}}. 
This file also includes {\btt{fitcompare()}}, 
a comparison routine used for
sorting populations in the nCUBE version.
\item[{\btt{app.c}}] contains application dependent routines. 
Unless you need to change the basic operation of the GA itself, 
you should only have to alter this file.
Further instructions for altering the SGA application are 
included in section~\ref{app}.
\begin{description}
\item[{\btt{application()}}] should contain any application-specific computations
needed before each GA cycle.  It is called by {\btt{main()}}.
\item[{\btt{app\_data()}}] should ask for and read in any application-specific 
information.  This routine is 
called by {\btt{init\_data()}}.
\item[{\btt{app\_malloc()}}] should perform any application-specific calls to 
{\btt{malloc()}} to dynamically allocate memory.  This routine is 
called by {\btt{initmalloc()}}.
\item[{\btt{app\_free()}}]  should perform any application-specific calls to 
{\btt{free()}}, for release of dynamically allocated memory.  
This routine is called by {\btt{freeall()}}.
\item[{\btt{app\_init()}}] should perform any application-specific initialization
needed.  It is called by {\btt{initialize()}}.
\item[{\btt{app\_initreport()}}] should print out an application-specific initial
report before the start of generation cycles.  This routine is 
called by {\btt{initialize()}}.
\item[{\btt{app\_report()}}]  should print out any application-specific  
output 
after each GA cycle.  It is called by {\btt{report()}}.
\item[{\btt{app\_stats()}}]  should perform any application-specific statistical
calculations.  It is called by {\btt{statistics()}}.
\item[{\btt{objfunc(critter)}}]  is the objective function for the specific application.
The variable {\btt{critter}} is a pointer to an {\btt{individual}} 
(a GA population member), to which this routine must assign a fitness.
This routine is called by {\btt{generation()}}.
\end{description}
\item[{\btt{Makefile}}] is a UNIX makefile for SGA-Cube.  With minor modifications
(see the comments in the file) this makefile can be used on most UNIX machines
and the nCUBE 2 parallel supercomputer.  This makefile relies on C preprocessor
{\btt{defines}} to correctly compile either the generic UNIX version or the nCUBE
version of SGA-Cube.
\end{description}

\section{New Features of SGA-Cube}
Like SGA-C \cite{Smith:91}, SGA-Cube has several features that differ from those of the Pascal version.
One is the ability to name the input and output files on the command line, i.e.
{\btt{sga my.input my.output}}.  If either of these files is not named on the command line, 
SGA-Cube assumes 
{\btt{stdin}} and {\btt{stdout}}, respectively.  
Another new feature of SGA-Cube is its method of representing
chromosomes in memory.
SGA-Cube stores its chromosomes in bit strings at the machine level. 
Another new feature in SGA-Cube is, of course, its nCUBE capabilities.
Input-output, chromosome storage, and the
parallel features of SGA-Cube are discussed in the following sections.

\subsection{Input-Output}
SGA-Cube allows for multiple GA runs.
When the program is executed, the user is first prompted
for the number of GA runs to be
performed.  After this, the quantity of input needed depends on 
the selection routine chosen at compile-time, any
application-specific information required, and whether the
code is being executed on the nCUBE or on a serial computer.  
When compiled
with roulette wheel selection, 
the input requested from the user is as follows:
\begin{itemize}
\item The number of GA runs to be performed ({\btt{int}}).
\item The population size for UNIX versions, {\em or} the population size 
{\em per node} for the nCUBE version ({\btt{int}}).
\item The chromosome length ({\btt{int}}).
\item Print the chromosome strings each generation ({\btt{y/n}})?
\item The maximum number of generations for the run ({\btt{int}}).
\end{itemize}
If the nCUBE version is run, the following information is requested:
\begin{itemize}
\item The percent of the population per node to exchange with nearest neighbor
nodes({\btt{float}}).  A integer number of fittest individuals will be sent to a
node, where the integer is {\btt{(population size per node * percent)/100.0}}.
Percentages greater than 50\% are not recommended, and percentages greater than
100\% cause a fatal error.
\item The number of generations between node population exchanges ({\btt{int}}).
\end{itemize}
For all versions of SGA-Cube, the following input is also requested from the user:
\begin{itemize}
\item The probability of crossover ({\btt{float}}).
\item The probability of mutation ({\btt{float}}).
\item Application-specific input, if any.
\item The seed for the random number generator ({\btt{float}}).
\end{itemize}

\subsection{Chromosome Representation and Memory Utilization}
\label{memstuff}
SGA-Cube uses a machine level representation of bit strings
to increase efficiency.
This allows crossover and mutation to be implemented 
as binary masking operations (see {\btt{operators.c}}).
Every chromosome (as well as the population arrays and some
auxiliary memory space) are allocated dynamically at run
time. The dynamic memory allocation scheme allocates
a sufficient number of unsigned integers for each population member
to store bits for the user-specified chromosome length. Because
of this feature, it is extremely important that {\btt{BITS\_PER\_BYTE}}
be properly set (in
{\btt{sga.h}} and {\btt{external.h}})
for your machine's hardware and C compiler.

\subsection{Parallel Hypercube Algorithm}
\label{parallel}
The parallel implementation of the SGA algorithm contained in SGA-Cube
is a straight-forward extension of the algorithm for a single processor.  
On an n-dimensional hypercube, there are $2^n$ processors, or nodes.  
Every node in an n-dimensional hypercube has $n$
nearest neighbors in $n$ different directions.  This implementation
starts with a different random population of individuals on each node,
spawned by a different random number seed for each node.  For each generation,
every node in the hypercube employs the SGA algorithm independently, using
crossover and mutation to improve the population on the node.  The parallel
algorithm, however, allows for the propagation of fittest individuals among
nodes.  The parallel SGA-Cube code requires two additional input parameters:
{\btt{nexchange}}, the number of individuals to exchange between nodes, and
{\btt{exchangegen}}, the number of generations between coordinated exchanges.
Every {\btt{exchangegen}} generations the following events happen in the
parallel SGA-Cube algorithm:
\begin{enumerate}
\item Each node quick-sorts its own population by fitness, ranking the 
individuals from fittest to weakest.
\item A random number in the range from $1$ to $n$ is picked as the 
exchange direction, and broadcasted to all nodes.
\item Every node sends information on its {\btt{nexchange}} fittest 
individuals to the nearest node in the chosen direction, and receives
information from that neighbor about the neighbor's fittest individuals.
\item Every node then replaces its {\btt{nexchange}} weakest individuals
with the individuals received from the neighbor.
\end{enumerate}
After these events, each node re-begins
its own GA algorithm independently.

The SGA-Cube code is fully parallel and scales
to any hypercube dimension.  Specification of the hypercube dimension at 
execution time should be done with the {\btt{xnc}} command; for instance
{\btt{xnc -d7 sga \ldots}} would execute the parallel SGA-Cube program on $2^7$
or $128$ nodes.

\section{Implementing Application Specific Routines}
\label{app}
To implement a specific application, you should only have to change the file
{\btt{app.c}}.
Section~\ref{files} describes the routines in {\btt{app.c}} in detail.
If you use additional variables for your specific problem, the easiest method
of making them available to other program units is to declare them in 
{\btt{sga.h}} and {\btt{external.h}}.  However, take care that you do not
redeclare existing variables.

Two example applications files are included in the SGA-Cube distribution.  The
file {\btt{app1.c}} performs the simple example problem 
included with the Pascal version; 
finding the maximum of $x^{10}$, where $x$
is an integer interpretation of a chromosome.  

A slightly more complex 
application 
is include in {\btt{app2.c}}.
This application illustrates two features that have been added
to SGA-Cube. The first of these is the {\bf ithruj2int} function,
which converts bits $i$ through $j$ in a chromosome to an integer.
The second new feature is the {\bf utility} pointer that is associated with each population member.

The example application interprets each chromosome as a set 
of concatenated integers in binary form. The lengths of these integer fields is 
determined by the user-specified value of {\bf field\_size}, which is 
read in by the function
{\btt{app\_data()}}.  The field size must be less than the smallest of
the chromosome
length and the length of an unsigned integer.  
An integer array for storing the interpreted form of each chromosome
is dynamically allocated and assigned to the chromosome's {\bf utility} pointer
in {\btt{app\_malloc()}}.
The {\btt{ithruj2int}} routine (see {\btt{utility.c}}) is used to translate 
each chromosome into its associated vector.
The fitness for each chromosome is simply the sum of the squares of these 
integers. This example application will function for any chromosome length. 

\section{Final Comments}
SGA-Cube is intended to be a simple
code for first-time experimentation with parallel GAs. It is 
not intended to be 
definitive in terms of its efficiency 
or the grace of its implementation. The
authors are interested in the comments, criticisms, and bug reports
from SGA-Cube users, so that the code can be refined for
easier use in subsequent versions.
Please email your comments to {\bf rob@comec4.mh.ua.edu},
or write to TCGA:
\begin{center}
The Clearinghouse for Genetic Algorithms\\
The University of Alabama\\
Department of Engineering Mechanics\\
P.O. Box 870278\\
Tuscaloosa, Alabama 35487
\end{center}

\subsection*{Acknowledgments}
The authors gratefully acknowledge support provided by NASA under
Grant NGT--50224 and support provided by the 
National Science Foundation under Grant CTS--8451610.
We also thank Hillol Kargupta for donating his tournament selection
implementation.


\bibliography{sga-cube}
\end{document}

