\documentstyle[12pt]{article}
\textwidth=6.0in
\textheight=9.5in
\hoffset=-0.5in
\voffset=-1.0in
\parskip=0.1in
\raggedright
\begin{document}

\title{Parallel Genesis User's Guide}
\author{Michael C. van Lent}
\date{21 August 1992}

\maketitle

\section{Introduction}

This User's Guide is for use with the Parallel Implementation of Genesis which
runs on the CM-200.  Described here will be the differences between Parallel
Genesis and standard Serial Genesis, how to use Parallel Genesis, and current
bugs and ideas for future improvements to Parallel Genesis.  It is assumed
that the user is familiar with Serial Genesis, C* and knows how to operate 
the Connection
Machine.  Instructions on using the NRL Connection Machine can be found in
the report entitled {\it Using the NRL CM-200}.  This guide also does not
comment on the implementation details of Parallel Genesis.  These are discussed
in the report {\it Parallelizing Genetic Algorithms}.

\section{Differences between Genesis and Parallel Genesis}

Parallel Genesis is an implementation of Genesis on the Connection Machine that
attempts to retain as much of the semantics and behavior of Serial Genesis as
possible while still providing maximum performance.  The sole difference 
between the semantics of Parallel and Serial Genesis is the manner in which 
random numbers are generated.  Theoretically if the parallel random number
generator were controlled to match the serial random number generator both
versions of Genesis would give identical reports(with the exception of 
Offline performance as mentioned below).  
This means that population statistics from a run of Parallel Genesis
and a run of Serial Genesis can be compared directly.

Other minor, non-semantic differences are
\begin{itemize}
\item Parallel Genesis always evaluates every member of the population each
generation even if a member's genotype has not changed.  This is equivalent
to having the {\tt a} flag always on in Serial Genesis.  Note that Parallel
Genesis pays no time penalty for this.

\item Parallel Genesis calculates Offline Performance slightly differently.
Since the entire population is evaluated at once Offline Performance is
always calculated with the best member of the population.  Serial Genesis
evaluates the population one at a time and therefore can only use the best
member found so far to calculate Offline Performance.

\item The order in which the best members are chosen for inclusion into the
{\tt min} file is slightly different however the fitnesses of the values in
the {\tt min} file will be the same.  For example if there are 11 members with
optimal genes then Parallel Genesis and Serial Genesis may not choose the
same subset of 10 to store in {\tt min}.

\item The internal representations have been changed to remove the need for
dynamically allocated structures.  This has been done to speed the parallel
implementation and to avoid difficulties in the C* memory management.
The effects of this change are explained in Section 3.
\end{itemize}

In addition to these differences Parallel Genesis has been expanded to include
a number of new options.  One of these new options, local selection, is
specifically oriented towards the parallel implementation while the others
are applicable to both Parallel and Serial Genesis(see the report 
{\it Modifications to Serial Genesis}).

\section{Using Setup}

As mentioned before Parallel Genesis has been modified to remove all dynamic
memory allocation.  Instead the {\tt setup} program creates a file called
{\tt parameters.h} which defines the constants necessary to allocate the
structures statically.

The only effect this has on the user is that the program must be recompiled 
whenever one of the following parameters is changed.
\begin{itemize}
\item Population Size
\item Number of Genes
\item Structure Length
\item Scaling Window
\item Structures Saved
\end{itemize}
The {\tt setup} program will automatically notify the user to recompile
whenever one of these parameters is changed.

Some techniques to avoid frequent compilation are
\begin{enumerate}
\item Run {\tt setup} before compiling a version for the first time.
\item Save executables in a subdirectory with their {\tt parameters.h} file so
that that configuration will not have to be built again.
\end{enumerate}

\section{Parallel Fitness Functions}

To be used with Parallel Genesis a fitness function must be written in C* and
use the following header.

\begin{verbatim}
double:physical Peval(str, length, vect, genes)
char:physical str[];	
int length;	
double:physical vect[];
int genes;
\end{verbatim}

The character string {\tt str} will be of length {\tt length} and the vector
array {\tt vect} will have {\tt genes} positions from 0 to {\tt genes}-1.
As usual the fitness function must also use the {\tt .cs} extension with its
file name.

\section{Compiling Parallel Genesis}

Once the fitness function is ready and {\tt setup} has been run, Parallel 
Genesis can be compiled using the {\tt make} command.

{\tt make f=}{\it fitness function} {\tt ga.}{\it fitness function}

For example if the fitness function were named {\tt F1.cs} then the {\tt make}
command to compile Parallel Genesis would be,

{\tt make f=F1 ga.F1}

{\bf Note:} The {\tt make} utility seems to have some trouble keeping
track of the C* files and it may be necessary to explicitly move or remove
any executables or {\tt .o} file that need to be recompiled.

\section{New Options}

\begin{description}

\item[a] This option tells Serial Genesis to evaluation the entire population
every generation.  Parallel Genesis always does this and ignores this flag.

\item[n] This is the Neighborhood Selection flag which directs Parallel
Genesis to have each member select its parent probabilistically from its
four neighbors.  The poplation is placed on a torus and each member of the
population interacts only with its North, East, West, and South neighbors.

\item[p] This is the Probabilistic Selection flag which directs Parallel
Genesis to do selection probabilistically as opposed to using the Baker
Stochastic Selection Algorithm.

\item[T] This is the Time flag which directs Parallel Genesis to print timing
information on each of it various parts as it runs.  Times are printed in
seconds.

\item[u] This is the Uniform Crossover flag which directs Parallel Genesis to
use uniform crossover as opposed to two point crossover.

\end{description}

\section{Bugs and Future Work}

{\large Bugs and Minor Fixes:}
\begin{itemize}
\item Reset defaults in {\tt setup} program.
\item Include dynamic allocation in Parallel Genesis.
\end{itemize}
{\bf Note:} Parallel Genesis has not been extensivly tested and may contain
undetected bugs.

\medskip

{\large Future Work:}
\begin{itemize}
\item Add a parasites option per Hillis.
\item Add a local selection radius parameter.
\item Allow local selection on other topologies.
\item Add a Punctuated Equilibria option.
\end{itemize}

\end{document}