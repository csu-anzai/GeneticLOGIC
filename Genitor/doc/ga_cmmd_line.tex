\documentstyle [12pt]{article}

\setlength{\textwidth}{7.25in}
\setlength{\textheight}{9.25in}
\setlength{\oddsidemargin}{-.2in}
\setlength{\evensidemargin}{-.2in}
\setlength{\topmargin}{-.5in}

\begin{document}

\begin{center}
\section*{Genitor Command Line Options}
\end{center}

To get a list of the command line options for the Genitor program,
simply type `Genitor' with no arguments.  The following usage message
appears.

\begin{verbatim}
  USAGE:

  Genitor [-option value]
  -c file < Config Filename; string >
  -b 2.0  < Selection Bias; floating point >
  -d 4500 < Dump Interval; integer>
  -f file < Final Population Filename; string >
  -g 124  < Current Generation Number; integer >
  -i file < Initial Population Filename; string >
  -l 25   < String Length; integer >
  -m .15  < Mutation Rate; floating point (0-1.0) >
  -n file < Node file (TSP only); string>
  -o file < Output Dump File basename; string >
  -p 100  < Population Size;  integer >
  -r 8765 < Random Seed; (long) integer >
  -s 750  < StatusInterval; integer >
  -t 9000 < Number Trials; integer >
\end{verbatim}

Each tag-value pair is associated with a specific parameter of the Genitor
algorithm.  The following parameters must be explicitly given values: StringLength,
PopulationSize, NumberTrials.  The remainder of the parameters have default
settings which may be optionally altered by the user.

Tags and their associated values must always be separated from each other by 
at least one space.  These command line options may be given in any order.
However, the Genitor parameters will be set in the order used on the command line.  

The semantics of each parameter are described in detail in the following sections.


\subsection*{Configuration Filename (-c)}

An alternative to setting each desired parameter from the command line with
a tag-value pair is to use the -c option to specify a configuration file.  
You may specify all parameters in a configuration file, thus the simplest
possible command line is `Genitor -c ga.config'.  

Additionally, you may override configuration file values by including other
command line options AFTER the configuration file option.  For example, the
command line `Genitor -c ga.config -p 300' will first set all values as indicated
in ga.config (including PoolSize, if present in ga.config), but then override
any previous value of PoolSize with the 300 specified with the -p option.  
Conversely, the command line `Genitor -p 300 -c ga.config' would set PoolSize 
to the value in ga.config (if PoolSize is present in ga.config) rather than
the 300 specified with the -p option.  Finally, it should be noted that it is
not necessary to specify all parameters in a configuration file.  The 
configuration file can give values to an often-used set of parameters in a
consistent manner while other command line options set parameters which vary
from experiment to experiment.

The format of a configuration file is as follows:

A configuration file tag is a word beginning with the same character as the
command line tag. Actually, the configuration file tag may be the character which
follows the dash of the command line tag; as you like. The configuration file tag
and its value must appear on the same line and be separated by at least one space.
Only one tag value pair is allowed per configuration file line.  For example:

\begin{verbatim}
          PoolSize       120
          Length_String  25
          NumberTrials   25000
          StatusInterval 1500
          RandomSeed     12345678
          Bias           1.900000
          MutateRate     0.000000
          InitPool       dump.pool
          FinalPool      final.pool
\end{verbatim}



NOTE: the `c' tag is disallowed in the config file since it could cause infinite 
recursion (and does not make much sense anyway).



\subsection*{Selection Bias (-b)}
DEFAULT = 2.0

The Selection Bias is a floating point number used in the selection of 
two genes (parents) for genetic reproduction (crossover).  This number 
specifies the amount of preference to be given to the superior individuals
in a genetic population.  For example, a bias of 2.0 indicates that the
best individual has twice the chance of being chosen as the mean individual.



\subsection*{Dump Interval (-d)}
DEFAULT = 0

When this parameter is given a non-zero value N, the current state of the
genetic alogrithm will be saved to disk every Nth generation.  The process
of saving state is as follows:

Two disk files are created of the form `dump.ext' where `.ext' distinguishes
between the files. dump.config is a regulation configuration file (as described
above in the c option section) containing all parameter settings used in the
current experiment.  It sets the Initial Population Filename (i) option
(see below) to the name of the second file created by a dump, the dump.pool
file.  dump.pool contains the contents of the genetic pool at the time of
interrupt.

To restart an experiment from a dump, simply supply the configuration file
created, `dump.config', on the command line: Genitor -c dump.config. 


\subsection*{Final Population Filename (-f)}
DEFAULT = none

This option allows you to specify a file in which to place the final population
produced by a Genitor execution.  This file has several potential uses.  One
use is simply to look at the final population data, rather than just the result:
the best gene created.  Another possible use is to initialize another experiment's
population (see option i below), as would be warranted when the final results are 
not good enough.



\subsection*{Current Generation Number (-g)}
DEFAULT = 0

This option is most commonly set in the configuration file generated by a dump
procedure to indicate the generation of the experiment being processed when an
interrupt occurred.  For example, if the Trials parameter (see below) of an
experiment is set to 50,000 and you interrupt after 20,000, the dump configuration
file will record 20,000 as the value of the Current Generation Number, both for
your information and for the purposes of a later restart.



\subsection*{Initial Population Filename (-i)}
DEFAULT = none

The genetic pool may be intialized randomly, in which case this option is not
used, or from a data file, which this option allows you to specify.  The data file
consists of a line for each member of the population, where a line contains an ascii
representation of the (binary) gene and its associated (floating point) worth.

\begin{verbatim}
1010101110000 40.0
	  :         :
\end{verbatim}

The pool file created by a {\it dump} (see option d above) is in this format, as is
the pool file created using (option f above).  Additionally, you are free to 
create this file by any means and use it to initialize your population; this
last idea seems particularly useful in approaches which comibine the genetic
alogrithm with other optimization approaches.


\subsection*{String Length (-l)}
REQUIRED PARAMETER

The string length option simply specifies the number of unique positions in each
gene.  Currently, all genes in an experiment have the same String Length.



\subsection*{Mutation Rate (-m)}
DEFAULT = 0.0

Mutation itself consists of randomly altering a gene's data.  Contrary to 
popular belief, a high rate of mutation has the potential to be a disruptive,
degrading aspect of search.  Its role in the genetic algorithm has traditionally
been as a background operator, applied infrequently.  

However, recent research indicates that when a genetic pool begins to converge 
prematurely, mutation can revive the genetic diversity necessary to sustain search. 
As such, the Genitor algorithm supports an {\it adaptive} form of mutation in the spirit 
of maintaining diversity.  The implementation here applies mutation in response to the 
hamming distance between two parents selected for crossover (a recombination operation 
in which two high-performance genes interchange information).  The more different
the parents are, the more likely they are to be mutated, and conversely.

If the mutation rate is set to zero, no mutation occurs whatsoever.  The highest
rate of mutation recommended is about .20.  When a non-zero mutation rate is 
specified, each gene's chance of being mutated is between 0 and the mutation rate,
which increases with its similarity to other population members.

\subsection*{Node File (-n)}
DEFAULT = none

This parameter is only useful when running a Traveling Salesman Problem (TSP).
The file contains the x,y coordinates of the ``cities'' (or nodes) to
be visited in a TSP tour.  The format is two integers per line, separated by
white space.  The Genitor code package includes routines 
\footnote{see the {\bf Genitor Code Structure} document {\bf Examples} section
for a description of these Traveling Salesman Problem utilities.}
to read this file and construct a table of distances between cities for use 
by the evaluation function.  Each x,y coordinate set represents one city and
is assigned a symbolic integer according to its position in this file.  Thus,
the first x,y pair is city 1, the second city 2, and so on.

\subsection*{Output Dump File Basename (-o)}
DEFAULT = ``dump''

Dumps are created in one of two ways:  1) the -d option is used to specify
an interval at which the algorithm is saved automatically or 2) you may
hits `CTRL Z' when Genitor is executing in the foreground.  A dump does
two things: 1) print the current status of the population 
(Best, Worst, Mean, Average) to the screen and 2) save the current state
to disk file(s).  The disk file basename may be specified with the -o 
option.

To restart an experiment from a dump, simply supply the configuration file
created, `dump.config', on the command line: Genitor -c dump.config. 


\subsection*{Population Size (-p)}
REQUIRED PARAMETER

The population size option simply specifies the number of individuals (genes) in
the genetic pool.  Research has shown that population size is an important parameter
in effecient, effective genetic search and thus you will probably change this
parameter's value frequently.



\subsection*{Random Seed (-r)}
DEFAULT = system max int

This option allows you to initialize the random number generator used throughout
the Genitor algorithm.  


\subsection*{Status Interval (-s)}
DEFAULT = 0

When this parameter is given a non-zero value N, a brief report of the
genetic alogrithm's status will be printed to stdout every Nth generation.
Status reports are of the following form:

\begin{verbatim}
gen# | Bst: 21.43000  Wst: 17.53400  Mean: 19.87600  Avg: 19.99301
\end{verbatim}

\subsection*{Number Trials (-t)}
REQUIRED PARAMETER

This option is used to indicate the total number of genetic operations cycles 
executed in a particular experiment.  This parameter, like population size
(see option p above) must be tuned to achieve successful and efficient optimization.

\end{document}
\end{article}
