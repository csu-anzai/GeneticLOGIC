\documentstyle [12pt]{article}

\setlength{\textwidth}{7.25in}
\setlength{\textheight}{9.25in}
\setlength{\oddsidemargin}{-.2in}
\setlength{\evensidemargin}{-.2in}
\setlength{\topmargin}{-.5in}

\begin{document}

\begin{center}
{\huge Genitor Code Structure}
\end{center}

\section{Introduction}

Genetic algorithms have been applied to a wide variety of optimization
problems.  However, it is possible to utilize many of the basic steps
of the algorithm with little or no change, regardless of the application.
With this in mind, the Genitor code has been software engineered to 
maximize code reuse across applications.  Thus the developer of applications
which are optimized by a genetic algorithm can concentrate more effort on
application specific concerns.

\subsection{Fundamental Functional Modules}

The Genitor code is broken apart into functional modules. The following
lists Genitor's Fundamental Functional Modules. Section 2 describes
the function calls available to utilize each fundamental functional
module.

\begin{itemize}

 \item Execution Parameter Manipulation
  \begin{itemize}
   \item Process command line arguments.
   \item Set execution parameters.
  \end{itemize}
 \item Genetic Pool Management
  \begin{itemize}
   \item Allocate data structures for the genetic pool.
   \item Initialize the genetic pool.
   \item Sort the genetic pool by fitness.
   \item Insert offspring into the sorted genetic pool.
  \end{itemize}
 \item Selection
  \begin{itemize}
   \item Select two parents for genetic recombination.
   \item Bias functions for use in selection.
  \end{itemize}
 \item Search Status
  \begin{itemize}
   \item Compute average population fitness.
   \item Output status information.
   \item Save current status.
  \end{itemize}
 \item Signal Handling
  \begin{itemize}
   \item Recieve and process user-generated interrupts/signals.
  \end{itemize}
 \item Utilities
  \begin{itemize}
   \item Copy one gene to another.
  \end{itemize}
 
\end{itemize}

\subsection{Data Type Independence}

The genetic algorithm may be applied to a number of applications;
this inherently means that the algorithm must be able to manipulate
a number of different data types.  Therefore, the software engineering
of the Genitor code includes methodologies for using the fundamental
functional modules such that they are virtually data type independent.
Section 3 describes these methodologies in detail.


\subsection{Genetic Operators}

Additionally, code modules for several genetic {\it operators} are
included with the Genitor package.  The basic crossover operator
provided is {\bf Reduced Surrogate Crossover} [ref].  
At this point, this operator is standard within the genetic algorthim
research community.  

The choice of operator can be a very problem specific issue, and thus
the development of new genetic operators is a significant research issue.
A special {\bf Adaptive Mutation} operator [ref] is included with the Genitor
package.  Also included with this package is the {\bf Edge 
Recombination Operator} [ref] developed at Colorado State University 
for use with the classic Traveling Salesman Problem.

The genetic operator modules provided are discussed in detail in Section 4.


\subsection{Examples Provided}
Finally, several example main() driver functions are included with
the Genitor package.  These main() functions iteratively invoke the
{\bf fundamental functionalities} of a genetic algorithm and provide
examples of {\bf usage protocols}.  Section 5 presents the sample
problems included with the Genitor package.

\newpage


\section{Fundamental Functional Modules}

\subsection{Execution Parameter Manipulation}

ga\_params.c, ga\_params.h
\\
The file `ga\_params.c' respresents the functional module for manipulating
genetic algorithm execution parameters.  These subroutines can directly set
the values of Genitor's global variables, as are defined and declared in
ga\_global.h, ga\_global\_extern.h

There is a global variable for each essential execution parameter 
of the genetic algorithm.  The `ga\_global.h' file contains the
definition of Genitor's global variables.  It should be \#included in
Genitor's main() driver procedure file, which you will write yourself
\footnote{ Your driver procedure must set the values of those global variables
which are not set in this module.  See section 5 {\bf Examples Provided} for
sample implementations}.  

The header file `ga\_global\_extern.h' contains extern declarations
for these variables is available for inclusion in other files which
require access.

By default, access to these global variables is restricted to the
{\bf Execution Parameter Manipulation} functional module (which
initializes most of the global variables) and to the {\bf Signal Handling}
functional module (since calls to signal handling routines are invoked
automatically and cannot be passed parameters directly).  
\footnote {In fact, the Execution Parameter Manipulation and Signal
Handling modules depend on the naming of these globals and thus changing
the names is NOT recommended.} 
All other functional modules are passed values via subroutine calls.
The intent is to encourage modular programming and reduce dependence on
global variables.

Genitor execution parameters may be manipulated with the following
functions:

\begin{itemize}
\item Parse a Genitor command line.\footnote{See the {\bf Genitor Command Line
Options} section for detailed requirements about command line format.}
\begin{verbatim}
  parse_command_line (argc, argv)
  int                 argc;   /*number command line args*/
  char               *argv[]; /*command line tags and values*/

\end{verbatim}
\item Parse a Genitor configuration file.\footnote{See the {\bf Genitor Command
Line Options} section for detailed requirements about configuration file format.}
\begin{verbatim}
 int
 parse_config_file (filename)
 char               filename[];
\end{verbatim}
\item Set a Genitor execution parameter.
\begin{verbatim}
 int
 set_parameter (tag, value)
 char           tag;     /*each param has associated one char tag*/
 char           value[];
\end{verbatim}
\item Print the Genitor execution parameters.
\begin{verbatim}
 void
 print_params (fileptr)
 FILE         *fileptr;  /*direct output to this file pointer*/
\end{verbatim}
\item Display a command line usage message.
\begin{verbatim}
 void
 usage();
\end{verbatim}
\end{itemize}
\subsection{Genetic Pool Management}

ga\_pool.c, ga\_pool.h
\\
A genetic pool structure (POOL, POOLPTR) has three basic components: 

\begin{enumerate}
 \item a {\bf size} variable (int) indicating the number of genes in the pool.
 \item a {\bf string\_length} variable (int) indicating the number of
	   chromosomes (i.e. bits of data) per gene.
 \item a {\bf data} pointer (GENEPTR) indicating the location of current
	   population of genes.  Each gene (GENE) has two components:
	   \begin{enumerate}
		\item a {\bf worth} variable (float) indicating the relative ``fitness''
			  or ``goodness'' of an individual gene. 
		\item a {\bf string} of data (GENE\_DATA, GENE\_DATAPTR).
       \end{enumerate}
\end{enumerate}

This functional module provides methods for manipulating and managing
the genetic pool.  


\begin{itemize}
\item Manage memory for genetic pool.

\begin{verbatim}
 GENEPTR
 get_gene(string_length)
 int      string_length;

 POOLPTR
 get_pool(string_length, pool_size)
 int      string_length, pool_size;

 void
 free_gene(gene)
 GENEPTR   gene;

 void
 free_pool(pool, start_pt)
 POOLPTR   pool;
 int       start_pt;
\end{verbatim}

Note that free\_pool() includes a start\_pt parameter which allows you to
specify the location in the genetic pool at which to begin freeing genes.
Specifying a 0 will free all genes and deallocate the pool structure 
itself.  Specifying a non-zero number N will free all genes from N to
the pool size and set the pool\_size variable of the pool structure
to N.  This feature is included to assist in dynamic population sizing;
it allows you to decrease the population size with little effort.  However,
if you wish to increase the population size, you should do the following:

\begin{enumerate}
\item allocate a new, larger pool of size M
\item copy the existing, smaller pool of size N into the first N slots
	  of the new pool
\item free the smaller pool
\item initialize slots N through M in the new pool (see below). 
\end{enumerate}


\item Initialize the genetic pool.  

The genes may be randomly generated or input from a disk file, or some 
combination of both.  The start\_pt and stop\_pt variables of each 
initialization routine allow you to directly control the contents of 
genetic pool data structure during initialization.  Regardless, after 
initialization, the pool data array must be continguously initialized
from 0 to the population size(-1).  

The init\_pool() function provides the simplest form of pool initialization.
If seed\_file is NULL, random\_init\_pool() is called to randomly initialize
the genetic pool from start\_pt to stop\_pt.  Thus, to randomly initialize
the entire genetic pool, set start\_pt to 0 and stop\_pt to the population size.
If seed\_file has a value, seed\_pool() is called to read initialization
data from the indicated disk file.  If the seed\_file does not contain 
enough data to fill the genetic pool from start\_pt to stop\_pt, 
init\_pool() calls random\_init\_pool() to complete the initialization.

All three initialization functions return the number of genes initialized.
random\_init\_pool() and seed\_pool() may also be called directly if desired.
\begin{verbatim}
 int
 init_pool (seed_file, pool, start_pt, stop_pt, eval_fun)
 char      *seed_file;
 POOLPTR    pool;
 int        start_pt, stop_pt;
 float     (*eval_fun)();

 int
 random_init_pool(pool, start_pt, stop_pt, eval_fun)
 POOLPTR          pool;
 int              start_pt, stop_pt;
 float           (*eval_fun)();

 int
 seed_pool(fp, pool, start_pt, stop_pt, eval_fun)
 FILE     *fp
 POOLPTR   pool;
 int       start_pt, stop_pt;
 float    (*eval_fun)();
\end{verbatim}

\item Sort genetic pool according to fitness.  The sort routine
assumes a minimization problem, thus the most fit individuals have
the lowest worth value.  If your application requires the opposite
sorting procedure (i.e. you want high values to be considered good
and low values bad), the suggested method is to have the evaluation
function negate its value before return.
\begin{verbatim}
 void
 sort_pool(pool)
 POOLPTR   pool;
\end{verbatim}

\item Insert individuals into sorted genetic pool.
\begin{verbatim}
 void
 insert_gene(gene, pool)
 GENE        gene;
 POOLPTR     pool;
\end{verbatim}
\end{itemize}

\subsection{Selection}

ga\_selection.c, ga\_selection.h
\\
This functional module provides the capability to choose individuals
from the genetic pool for reproduction.

\begin{itemize}
\item Selection functions.
\begin{verbatim}
 void 
 get_parents (mom, dad, pool, bias_fun, bias_num);
 GENEPTR      mom, dad;
 POOLPTR      pool;
 int         (*bias_fun)();
 float        bias;
\end{verbatim}
\item Bias functions.
\begin{verbatim}
 int
 linear (max, bias_num);
\end{verbatim}
\end{itemize}


\subsection{Search Status}

ga\_status.c, ga\_status.h
\\
The file `ga\_status.c' contains the functional module which manages
displaying genetic algorithm execution status.  There are subroutines
to perform the following services:

\begin{itemize}

\item Compute average fitness of genetic pool.
\begin{verbatim}
 float
 avg_pool(pool)
 POOLPTR  pool; 
\end{verbatim}

\item Print the best, worst, average, and mean individuals in the population.
\begin{verbatim}
 void
 show_progress (fileptr, pool, current_generation)
 FILE          *fileptr; /*direct output to this fileptr*/
 POOLPTR        pool;
 int            current_generation;
\end{verbatim}

\item Print the contents of genetic pool, beginning at start\_pt for count.
\begin{verbatim}
 void
 print_pool(fp, pool, start_pt, count)
 FILE      *fp;
 POOLPTR    pool;
 int        start_pt,
			count;
\end{verbatim}

\item Save the current state of an experiment for later restart.  Two
files will be created: a configuration file (.config) with the current
settings of Genitor's execution parameters and a pool file (.pool) with the 
population data.
\begin{verbatim}
 void
 dump_status (pool, basename)
 POOLPTR      pool;
 char         basename[]; /*basename of files in which to save state*/
\end{verbatim}

\end{itemize}

Additionally, the Search Status module contains the following convenience
routines:

\begin{itemize}

\item Pause execution (until input is received).
\begin{verbatim}
 void
 pause_it();
\end{verbatim}

\item Print a warning message to stderr. 
\begin{verbatim}
 void
 warning(message)
 char    message[];
\end{verbatim}

\item Print an (optional) message to stderr and exit the program with
a -1.  If the input message is non-NULL it is printed to stderr.  
The standard message ``FATAL ERROR: Exiting.'' is always printed
immediately before exit.
\begin{verbatim}
 void
 fatal_error(message)
 char       *message;
\end{verbatim}

\end{itemize}


ga\_xdr.c, ga\_xdr.h\\
Finally, Genitor provides a binary data format for population data
disk files using the ``External Data Representation'' (XDR) standard. 
The XDR standard is most frequently used in conjunction with the RPC
utility, but need not be and, indeed, is useful outside the domain of RPC
procedures.  XDR was designed to work across different languages, operating
systems, and machine architectures.  Thus, an XDR format binary file may be
written on one machine, say a VAX running VMS, and read on another machine,
say a SUN Sparc running Unix. Even when machine portability is not an issue,
XDR is still a good choice because its protocols are standardized and
reliable.

In practice, you will probably only want to use the XDR format when
your application is dealing with floating point data.  When you use
floating point data, XDR can provide significant disk space savings.
However, for other data types (character or integer) you are better
off remaining with the normal ascii file format, as the overhead of
XDR's machine independent format actually makes files of this type
larger than their ascii counterparts.
\begin{verbatim}
int
print_xdr (fp, pool)
FILE      *fp;
POOLPTR    pool;

int 
read_xdr (fp, pool)
FILE     *fp;
POOLPTR   pool;
\end{verbatim}


\subsection{Signal Handling}

ga\_signals.c, ga\_signals.h
\\
This functional module module provides signal handling capabilites
during algorithm execution.  The subroutine `setup\_signal' may be
called at the beginning of the main() driver routine to install
signal handlers as appropriate.  The file `ga\_signals.c' is
the suggested location for signal handling subroutines as well.
The following signal handling routine is already provided:

\begin{itemize}
\item ga\_dump\_interrupt()\\
This subroutine will print the current execution status to the
screen and dump the current state of the genetic algorithm
execution to file(s). \footnote {See {\it show\_progress} and
{\it dump\_status} subroutines in {\bf Search Status} section above.} 
\end{itemize}


\subsection{Utilities}

ga\_copy.c, ga\_copy.h
\\
This functional module provides a data translation and copying
utility independent of data type.
\begin{verbatim}
 void
 gene_copy (gene1, gene2, string_length)
 GENEPTR    gene1, gene2;
 int        string_length;
\end{verbatim}


\newpage

\section{Different Data Types}

gene.h
\\
The Genitor algorithm may be applied to a number of problems, which
inherently means that the algorithm must be able to manipulate a
number of different data types.  The Genitor code is designed to
manipulated genes of various data types:

\begin{itemize}
\item {\bf Character} This is generally used for the traditional
binary encodings.
\item {\bf Integer} This might be used for a problem requiring a
symbolic encoding, such as the Traveling Salesman Problem.
\item {\bf Floating point} This can be used for symbolic problems
as well. For example this data reprensentation is well-suited to
neural network weight optimization problems.
\end{itemize}

You may use essentially the same code and the same function calls 
regardless of the data type due to the centralization of type-specific
information found in the header file `gene.h'.  All references to
Genitor's population data type are made using the typedefs contained
in `gene.h'.  Thus, you should include `gene.h' in any file which 
will reference Genitor's population data and use the standard
typedefs to do so.  In other words, to refer to:
\begin{itemize}
 \item the data type to be manipulated: use GENE\_DATA and GENE\_DATAPTR.
 \item a gene structure containing a string of data and a worth:
	   use GENE and GENEPTR.
 \item a pool structure containing a population of genes, a population
	   size, and a data string length: use POOL and POOLPTR.
\end{itemize}

The file `gene.h' has three sections, one for each of the three data
types enumerated above.  Each section typedefs GENE\_DATA to the
desired data type.  Additionally, each section contains a few macros
to insert data type specific methods and syntaxes at compile time.
In particular, the INIT() macro is used to call the appropriate random
initialization routines for each data type.\footnote{ INIT() is used
by random\_init\_pool() in the file ga\_pool.c.}  Additionally, the
input and output of data must be data type specific, and so three macros
(GENE\_DATA\_IN\_FORMAT, GENE\_DATA\_OUT\_FORMAT(), and XDR\_DATA)
\footnote{GENE\_DATA\_IN\_FORMAT is used by seed\_pool() in the file
ga\_pool.c.  GENE\_DATA\_OUT\_FORMAT() is used by print\_pool() in the
file ga\_status.c. XDR\_DATA is used by xdr\_GENE() in the file ga\_xdr.c.}
are defined for this purpose.

Since only one data type is active at a given time, the sections of
the gene.h file are bracked by \#ifdefs.  The appropriate constant
(BIN\_DATA, CHAR\_DATA, or FLOAT\_DATA) should be defined on the 
command line at compilation time (i.e. cc file.c -DCHAR\_DATA)

\newpage

\section{Genetic Operators:}

\subsection{Reduce Surrogate Crossover}

ga\_red\_surrog.c, ga\_red\_surrog.h
\\
Reduced Surrogate Crossover is a refined form of simple crossover.  Simple
crossover begins with choosing two points at which to {\it break} each of
two genetic strings.  Next, fragments from each string are used to build
a third string, thus forming a new string (a child) which contains
information from each of the initial two (parent) strings.  Reduced
Surrogate crossover uses this same idea, but enhances the simple
algorithm by insuring that the exchanged information is unique to each
parent (exchanging identical information is a waste of time).  The
Reduced Surrogate operator implementation provided with the Genitor package
returns the number of (bit positions) differences between the two parent
strings as the function value.  This return value can be useful in triggering 
mutation based on population diversity.  

\begin{verbatim}
  int
  red_surrogate_cross (mom_string, dad_string, string_length);
  GENE_DATA            mom_string[], dad_string[];
  int                  string_length;

\end{verbatim}

\subsection{Adaptive Mutation}

ga\_mutate.c, ga\_mutate.h
\\
Adaptive mutation bases the amount of disruption to a given string
on two factors: the relative similarity of its two parent strings,
and a {\it mutation rate}.  The more similar the two parent strings,
the more likely mutation is to occur.  The actual mutation which
occurs is the product of this similarity and a fractional mutation
rate.  Thus, if the mutation rate is .20 and the parents of a particular
string where identical, approximately 20\% of the string will be mutated.
Or, if the parents contained 50\% unique information, only about 10\% of
the string will be mutated.
\begin{verbatim}
  void
  adaptive_mutate (gene_string, string_length, diffs, mutate_rate);
  GENE_DATA        gene_string[];
  int              string_length;
  int              diffs;
  float            mutate_rate;

\end{verbatim}

\subsection{Edge Recombination}

ga\_tsp.c, ga\_tsp.h
\\
The Traveling Salesman Problem (TSP) is defined as follows: given a finite
list of points and information about the distances between each possible
pair of points, find the shortest circuit which visits every point exactly once.
The TSP is a classic schedule optimization exercise, and methodologies which
can be successfully applied to it are often of use in many types of scheduling
problems.  

The {\bf Edge Recombination} operator invented at Colorado State University
has had significant success in solving TSPs.  The novelty and success of CSU's
operator is due to the fact that the operator focuses on manipulating the
{\it edges} between points rather than the points themselves.  Classically,
genetic algorithms have worked by recombining the explicit data string
representations of individual genes.  For example, in function optimization,
genetic recombination involves exchanging parameter settings which are
positionally, explicitly encoded in the gene data strings.  However, in a TSP,
every string encodes the same explicit data (the list of cities to be visited).
Further, the real information lies in the position of each data element in
the string {\it relative} to other data elements (as opposed to the absolute
positioning of the function optimization example, in which relative position
is of little or no importance.)

The foundation of the Edge Recombination operator's implementation is an 
{\it edge table}.  Every time recombination takes place, an edge table
representing the edges of two parent genes is created.  A new gene is 
generated by selecting edges from this edge table, such that the new gene
is constructed of interspersed edges from each parent. 

The following section lists the functions which implement the Edge
Recombination operator.  These functions assume that each point in the
circuit is represented symbolically by an integer.  Thus, a 30 ``city'' tour
will have city number 1, number 2, and so on.  This integer naming scheme
lends itself easily to an implementation based on array indexing.  Before
examining the Edge Recombination functions provided, a brief description
of the edge table data structure is in order:

The edge table is an array of EDGE\_NODE structures.  The array is
indexed by city; for example the EDGE\_NODE for city 3 is found at
edge\_table[3].  Each EDGE\_NODE enumerates the cities to which it
has edges.  Thus, if parent A is 52314 parent B is 41253, the EDGE\_NODE
at edge\_table[3] will have references \footnote{Actually, each EDGE\_NODE
has an array of 4 elements to store connecting cities.  Also, note that
all 4 elements may not be used: often parents {\it share}
edges and thus a particular EDGE\_NODE may only reference 2 or 3 other
cities.} to cities 2, 1, 5, and 4 since the edges containing 3 in the parents
are 23, 31, 53, and 34.  Likewise, the EDGE\_NODEs at edge\_table[2], [1],
[5], and [4] will include a reference to city 3.  A new gene is constructed
by moving through the connections represented in the edge\_table, removing
cities from the edge\_table as they are visited.

\begin{verbatim}
 EDGE_NODE *
 get_edge_table (num_points)
 int             num_points;

 free_edge_table (edge_table);
 EDGE_NODE        edge_table[];

 float
 build_edge_table (mom_string, dad_string, num_points, edge_table)
 GENE_DATA         mom_string[], dad_string[];
 int               num_points;
 EDGE_NODE         edge_table[];

 GENEPTR
 build_tour (edge_table, child_string, num_points)
 EDGE_NODE   edge_table[];
 GENE_DATA   child_string[];
 int         num_points;
\end{verbatim}

\newpage

\section{Examples}
\end{document}
\end{article}
