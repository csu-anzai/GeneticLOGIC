\documentstyle [12pt]{article}

\setlength{\textwidth}{7.25in}
\setlength{\textheight}{9.25in}
\setlength{\oddsidemargin}{-.2in}
\setlength{\evensidemargin}{-.2in}
\setlength{\topmargin}{-.5in}

\begin{document}

\begin{center}
{\huge Getting Started with Genitor}
\end{center}

\section{Introduction}

The purpose of this document is to get you started using the Genitor package
as quickly as possible.  The Genitor package can be useful in three ways:

\begin{enumerate}
 \item You can learn about genetic algorithm technology by running the
	   simple examples provided.  You can change the values of the 
	   basic parameters controlling the genetic algorithm execution
	   and note their effect.
 \item You can use the genetic algorithm to solve one of your own
	   optimization problems.  To do this, you will need to write 
	   write a few C functions specific to your optimization problem.
	   You will use them in conjunction with the libraries provided with
	   the Genitor package to create an optimization algorithm 
	   customized for your problem. 
 \item You can pursue genetic algorithm research.  The Genitor package
       is designed to allow you to concentrate efforts on research worthy
	   issues and leave the basic mechanics of a genetic algorithm up to
	   the Genitor library functions.  For example, you might write a 
	   new genetic reproduction operator; the integration of new operators
	   is relatively simple with the Genitor package.	
\end{enumerate}

\section{Examples}

\subsection{Creating an Executable Program}

When the Genitor package is properly installed you should have a Genitor
directory containing five subdirectories: Examples, doc, include, lib,
and src. In order to create and run the Examples, the lib directory must
contain the Genitor libraries. Make sure that lib/ga contains three .a 
files before trying to use the Examples.  \footnote{See the
{\bf Genitor Installation Instructions} document for details about creating
these libraries.}  If everything is installed correctly, you will only need
to concern yourself with the Examples directory in order to run the sample
problems.

The Examples directory contains three subdirectories: Traditional,
TSP, and NeuralNet. The Traditional subdirectory contains the code
required to perform a basic function minimization by manipulating the
traditional data type of genetic algorithms, binary strings.  The TSP
subdirectory contains the code required to solve Traveling Salesman
Problems. The NeuralNet subdirectory contains the genetic algorithm
code to solve a two-bit adder problem.

Everything should be essentially ready to go, with one exception:
each subdirectory contains a ``Makefile''.  Before you can create
the example programs, you must change 1 line in each makefile.

\begin{verbatim}
The line which reads

GaRoot = 

must be altered to read

GaRoot = /home/you/Genitor

where `/home/you/Genitor' is the location of the Genitor directory.
\end{verbatim}

All other pathnames in the Genitor package makefiles are relative to
GaRoot. 

Once you have updated each makefile appropriately, you are ready to
run the examples.  The following two steps are the same for each example:

\begin{enumerate}
 \item Change to the Example subdirectory desired (Traditional, TSP, or
	   Neural Net). 
 \item type `make' to create an executable Genitor program.
\end{enumerate}

\subsection{Running a Genitor Program}

Parameters for Genitor program execution may be specified via the
command line, a configuration file, or a combination of both.  For
complete details on the available command line options and using/writing a
configuration file, refer to the {\bf Genitor Command Line Options}
documentation.

Each example includes one (or more) configuration file(s) containing
a reasonable default set of parameter values for the particular
genetic experiment.  Configuration files are found in the SampleConfigs
subdirectory of each Example subdirectory.  To use these default values,
you will type

Genitor{\it example} -c {\it config.file}

The algorithm will print to the console (stdout) the parameter values
used for the experiment and periodically print a population summary.
At completion, the `best solution' string will be printed.

\subsection{Traditional Function Optimization}
\subsection{Traveling Salesman Problem}

The Traveling Salesman Problem (TSP) is a classic schedule optimization
problem defined as follows:  given a fixed set of cities, in what relative
sequence must the cities be visited such that the shortest route is taken
AND such that each city is visited once and only once? The
methodologies which can be successfully applied to the TSP are often of use in
many types of scheduling problems.  The {\bf Edge Recombination} genetic
operator invented at Colorado State University has had significant success in
solving TSPs.  This example allows you to run a genetic algorithm using the
{\bf Edge Recombination} operator to solve two common benchmark TSP
problems: a 15 city problem and a 30 city problem.

Each sample problem has an associated configuration file, (tsp15.config and
tsp30.config).  Additionally, the x,y coordinates of the cities for each
problem are included (node15.tsp and node30.tsp).  The coordinates in these
files will correspond to a unique integer representing each city.  The integer
to be associated with each set of coordinates is relative to the ordering of
the coordinate file.  Therefore, the first two numbers in the file node15.tsp
represent the x and y coordinates of city 1.  The second two coordinates in
the file represent city 2, and so on.  It is important that you understand
this scheme because when the TSP program is executed and the final result is
printed to the screen (stdout), it will appear as a string of integers
separated by spaces (i.e. 3 7 8 9 ... 1 15 ) and you may wish to map this
result back to the original coordinates.

Once you understand a bit about the problem and interpreting its data,
perform the following steps to execute the algorithm:
\begin{enumerate}
 \item Make sure that an executable program named GenitorINT exists in
	   the TSP directory.  Details on making this executable program
	   can be found in the {\bf Creating an Executable Program} section above.
 \item Change to the SampleConfig directory for the 30 city problem.
 \item Type {\bf ../../GenitorINT -c tsp30.config}.
 \item Watch the screen messages to see the algorithm progress and converge.
	   The best known solution for this 30 city problem is 420.0.  
\end{enumerate}

NOTE:\\
When the Genitor package tests were run prior to shipment, the tsp30.config
file was filled with the parameter settings that always suceeded in
finding the correct solution within 30,000 genetic recombinations on our
test machines.  However, machines vary and it is possible that your machine
will find the solution much sooner (actually, our average was about
25000 recombinations), take longer, or fail.  The genetic algorithm is
very dependent upon {\it good} random number generators and this 
varies from platform to platform.  Let us know if you find a significant
variation!

\subsection{Neural Net}

\section{Solving Your Problem}

Perhaps you have a particular problem, say a tricky function that needs
optimizing, that you would like to try solving with the Genitor algorithm.
Basically, you will need to do the following things:

\begin{itemize}
 \item Determine a way to encode the critical data of your problem into
	   a {\it string}.  
	   
	   For example, if your problem is function optimization,
	   perhaps you can represent each parameter setting using 8 bits.
	   If you interpret the first 8 bits of a {\it string}
	   as parameter setting one, the second 8 bits as parameter setting
	   two, etc., you have described the mapping of your problem data into
	   a string manipulable by the genetic algorithm.

 \item Determine the data type of the string.  

	   For example, if you chose the {\it binary} representation 
	   described above, your data type will be {\bf BIN}.  Alternatively,
	   if you decided that your string would be composed of floating
	   point numbers, each representing a function parameter, your
	   data type will be {\bf FLOAT}.  The final data type supported
	   by the Genitor code package is {\bf INT}, such as is used with
	   the Traveling Salesman Problem.

	   The choice of data type determines which Genitor library 
	   you will later link in with your application.  

	   The only other consideration with data type is that any code
	   you write should {\bf always} refer to the data type as
	   {\bf GENE\_DATA} (never char, int, or float).  A string should
	   {\bf always} be referred to as an array of GENE\_DATA.

 \item Write an evaluation function which takes as input a {\it string}
	   and a string length.  The evaluation function then returns a
	   floating point number representing the {\it goodness} or
	   {\it badness} of the input string.  
	   
	   The more accurate the evaluation function, the better the results
	   of the genetic algorithm usually are.  However, the evaluation 
	   function is often the most costly aspect of genetic optimization
	   methods, so take care to implement this function efficiently.

 \item Write a main() program which utilizes the basic Genitor library
	   routines to perform the mechanics of the genetic algorithm.  Your
	   main program will also rely on the genetic operator code modules
	   shipped with Genitor.  However, your main() program will 
	   call your evaluation function and uses status reporting mechanisms
	   to your taste.  {\bf You should read the Genitor Code Structure
	   document before attempting to code your own main function.}

	   Actually, the Example programs shipped with Genitor should provide
	   an excellent starting point.  You may wish to copy a main program
	   file from one of the Genitor Example subdirectories and use this
	   as a starting point which you can edit for your own purposes.

 \item Compile your evaluation function and main program and link it with
	   the Genitor libraries (appropriate for your data type) and the operator
	   module(s) you have chosen.

	   Again, the Example programs shipped with Genitor will be of considerable
	   help to you here.  You may wish to copy a Makefile from one of the
       Genitor Example subdirectories and use this as a starting point.

\end{itemize}

\section{Research}

\end{document}
\end{article}
