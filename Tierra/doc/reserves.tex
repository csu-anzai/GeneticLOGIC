\documentstyle[12pt]{article}

\flushbottom
\textheight 9in
\textwidth 6.5in
\textfloatsep 30pt plus 3pt minus 6pt
\parskip 7.5pt plus 1pt minus 1pt
\oddsidemargin 0in
\evensidemargin 0in
\topmargin 0in
\headheight 0in
\headsep 0in

% Hanging Paragraph
\def\XP{\par\begingroup\parindent 0in\everypar{\hangindent .3in}}
\def\eXP{\par\endgroup}

% Left Justified Paragraph
\def\LP{\par\begingroup\parindent 0in\everypar{\hangindent 0in}}
\def\eLP{\par\endgroup}

% Indented Paragraph
\def\IP{\par\begingroup\parindent 0.3in\everypar{\hangindent 0.3in}}
\def\eIP{\par\endgroup}

% Hanging Paragraph with no parskip
\def\XPNS{\vspace{7.5pt}\par\begingroup\parskip 0pt\parindent 0in\everypar{\hangindent .3in}}
\def\eXPNS{\par\endgroup}

\begin{document}
\thispagestyle{empty}

\begin{center}
{\Large {\bf A PROPOSAL TO CREATE TWO\\
BIODIVERSITY RESERVES:\\
ONE DIGITAL AND ONE ORGANIC\\ }}
\end{center}

\vspace{1cm}

{\large {\bf
The digital reserve will be distributed across the global net, and will
create a space for the evolution of new virtual life forms.  The organic
reserve will be located in the rain forests of northern Costa Rica, and
will secure the future of existing organic life forms.
}}

\begin{center}
{\bf The Digital Reserve:}
\end{center}

The proposed project will create a very large, complex and inter-connected
region of cyberspace that will be inoculated with digital organisms which
will be allowed to evolve freely through natural selection.  The objective
is to set off a digital analog to the Cambrian explosion of diversity, in
which multi-cellular digital organisms (parallel MIMD processes) will
spontaneously increase in diversity and complexity.  If successful, this
evolutionary process will allow us to find the natural form of parallel
processes, and will generate extremely complex digital information
processes that fully utilize the capacities inherent in our parallel and
networked hardware.  The project will be funded through the donation of
spare CPU cycles from thousands of machines connected to the net, by
running the reserve as a low priority background process on participating
nodes.

\begin{center}
{\bf The Organic Reserve:}
\end{center}

The proposed project aims to prevent the imminent destruction of
some of the last remaining large areas of rain forest in the
Sarapiqu\'{\i} region of Northern Costa Rica, and at the same time,
to use those forests to establish a conservation economy through a
community based nature tourism project.  The conservation economy is
needed to insure the long term stability of both the present and future
protected areas of Sarapiqu\'{\i}, by creating an economic interest group
within the human populations surrounding the protected forests.  The
overall project contains six stages, which will be executed in succession,
as funding reaches the corresponding target levels.  The Nature Conservancy
(TNC) is acting as fiscal agent, and the Costa Rican government has expressed
a willingness to expropriate the target properties if necessary.  This
project is being organized and supported by a coalition of conservation
organizations: Sarapiqu\'{\i} Association for Forests and Wildlife,
OTS (The Organization for Tropical Studies), TNC, FUNDECOR (The Foundation
for the Development of the Central Volcanic Mountain Range), COMBOS (The
Conservation and Management of Tropical Forests), and ABAS (The Association
for the Environmental Well being of Sarapiqu\'{\i}).

\newpage
\setcounter{page}{1}

\begin{center}
{\Large {\bf
A PROPOSAL TO CREATE A NETWORK-WIDE\\
BIODIVERSITY RESERVE FOR\\
DIGITAL ORGANISMS\\
}}
\end{center}

\LP
{\bf Thomas S. Ray} \\
ATR Human Information Processing Research Laboratories\\
2-2 Hikaridai, Seika-cho, Soraku-gun, Kyoto, 619-02, Japan\\
ray@hip.atr.co.jp, ray@santafe.edu, ray@udel.edu\\
March 18, 1994\\
\eLP

\begin{abstract}
The proposed project will create a very large, complex and inter-connected
region of cyberspace that will be inoculated with digital organisms which
will be allowed to evolve freely through natural selection.  The objective
is to set off a digital analog to the Cambrian explosion of diversity, in
which multi-cellular digital organisms (parallel MIMD processes) will
spontaneously increase in diversity and complexity.  If successful, this
evolutionary process will allow us to find the natural form of parallel
and distributed processes, and will generate extremely complex digital
information processes that fully utilize the capacities inherent in our
parallel and networked hardware.  The project will be funded through the
donation of spare CPU cycles from thousands of machines connected to the
net, by running the reserve as a low priority background process on
participating nodes.
\end{abstract}

\section{\bf WHY}

The process of evolution by natural selection is able to create
complex and beautiful information processing systems (such as primate
nervous systems) without the guidance of an intelligent supervisor.
Yet intelligent programmers have not been able to produce software
systems that match even the full capabilities of primitive organisms such
as insects.  Recent experiments demonstrate that evolution by natural
selection is able to operate effectively in genetic languages based on
the machine codes of digital computers (Ray 1991a, 1991b, In press).
This opens up the possibility of using evolution to generate complex
software.

Ideally we would like to generate software that utilizes the full
capability of our most advanced hardware, particularly massively
parallel and networked computational systems.  Yet it remains an open
question if evolution has the ability to achieve such complexity in
the computational medium, and if it does, how that goal can be achieved.
Successful efforts at the evolution of machine codes have generally
worked with programs of under a hundred bytes.  How can we provoke
evolution to transform such simple algorithms into software of vast
complexity?

Perhaps we can gain some clues to solving this problem by studying the
comparable evolutionary transformation in organic life forms.  Life
appeared on Earth roughly 3.5 thousand million years ago, but remained
in the form of single celled organisms until about 600 million years
ago in the Cambrian period.  At that point in time, life made an abrupt
transformation from simple microscopic single celled forms lacking
nervous systems, to large and complex multi-celled forms with nervous
systems capable of coordinating sophisticated behavior.  This
transformation occurred so abruptly, that evolutionary biologists refer
to it as the ``Cambrian explosion of diversity.''

It is heartening to observe that once conditions are right, evolution
can achieve extremely rapid increases in complexity and diversity, generating
sophisticated information processing systems where previously none existed.
However, our problem is to engineer the proper conditions for digital
organisms in order to place them on the threshold of a digital version of
the Cambrian explosion.  Otherwise we might have to wait millions of
years to achieve our goal.  Ray (1994a) has reviewed the biological
issues surrounding the evolution of diversity and complexity, and
they lead to the following conclusions:

Evolution of complexity occurs in the context of an ecological community
of interacting evolving species.  Such communities need large complex spaces
to exist.  A large and complex environment consisting of partially isolated
habitats differing and occasionally changing in environmental conditions
would be the most conducive to a rapid increase in diversity and complexity.

These are the considerations that lead to the suggestion of the creation
of a large and complex ecological reserve for digital organisms.  Due to
its size, topological complexity, and dynamically changing form and
conditions, the global network of computers is the ideal habitat for the
evolution of complex digital organisms.

\section{\bf ``MANAGING'' EVOLUTION}

Some questions frequently asked about software evolution are:
How can we guide evolution to produce useful application software?
How can we validate the code produced by evolution to be sure that
it performs the application correctly?  These questions reveal a limited
view of how software evolution can be used, and what it can be used for.
I will articulate a fairly radical view here.

Evolution would not be an appropriate technique for generating accounting
software, or any software where precise and accurate computations are
required.  Evolution would be more appropriate for more fuzzy problems
like pattern recognition.  For example, if you get a puppy that you want
to raise to be a guard dog, you can't verify the neural circuitry or the
genetic code, but you can tell if it learns to bark at strangers and
is friendly to your family and friends.  This is the type of application
that evolution can deliver.  We don't need to verify the code, but
verification of the performance should be straightforward.

Furthermore, attempts to guide early evolution towards a desired application
are likely to inhibit its creative potential.  Once evolution
by natural selection has already produced an incipient application, then
guidance through artificial selection (breeding) can enhance the quality
of the application.  However, we should not attempt to guide evolution to
generate the application in the first place.  Instead, we should wait to
see what evolution offers us.  After all, we don't necessarily know what
we want.

Computer magazines bemoan the search for the ``next killer application'',
some category of software that everybody will want, but which nobody has
thought of yet.  The markets for the existing major applications (word
processors, spread sheets, data bases, etc.) are already saturated.
Growth of the software industry depends on inventing completely new
applications.  This implies that there are categories of software that
everyone will want but which haven't been invented yet.
We need not attempt to use evolution to produce superior
versions of existing applications.  Rather we should allow evolution to
find the new applications for us.  To see this process more clearly,
consider how we manage applications through organic evolution.

Some of the applications provided by organic evolution are: rice, corn,
wheat, carrots, beef cattle, dairy cattle, pigs, chickens, dogs,
cats, guppies, cotton, mahogany, tobacco, mink, sheep, silk moths,
yeast, and penicillin mold.  If we had never encountered any one of
these organisms, we would never have thought of them either.  We have
made them into applications because we recognized the potential in some
organism that was spontaneously generated within an ecosystem of
organisms evolving freely by natural selection.

If the silk moth never existed, but we somehow came up with a complete
description of silk, it would be futile to attempt the guide the evolution
of any existing creature to produce silk.  It is much more productive to
survey the bounty of organisms already generated by evolution with an eye
to spotting new applications for existing organisms.  Some breeding may
be necessary to make the application practical.  For example, corn, dogs,
and cattle are all highly bred organisms, of much greater utility in their
present form than that of their wild ancestors.

Imagine for a moment that a team of earth biologists had arrived at a
planet at the moment of the initiation of its Cambrian explosion of
diversity.  Suppose that these biologists came with a list of the
application organisms listed above (rice, corn, etc.), and a complete
description of each.  Could those biologists intervene in the
evolutionary process to hasten the production of any of those
organisms?  Not only is that unlikely, but any attempts to intervene in
the process are like to inhibit the diversification itself.

It is preposterous to suppose that humans could guide the evolution
of useful complex organisms from their simple single celled ancestors.
In fact, we couldn't even imagine what the possibilities are, much less
know how to reach those possibilities if we could conceive of them.
Fortunately, our intervention is not necessary.  Evolution by natural
selection will produce a wealth of complex organisms, and we can survey
them and bring those with potential uses into breeding and domestication
programs.

\section{\bf A BETTER MEDIUM}

Natural evolution in the digital medium is a new technology, about which
we know very little.  The hope is to evolve software with sophisticated
functionality far beyond anything that has been designed by humans.  But
how long might this take?  Evolution in the organic medium is known to be
a slow process.  Certainly there remains the possibility that evolution in
the digital medium will be too slow to be a practical tool for software
generation, but several observations can be made that provide encouragement.

First, computational processes occur at electronic speeds, and are in fact
relatively fast.  Second, as was noted in the first section above, during the
Cambrian explosion, evolution produced such a rapid inflation of complexity
and diversity, that it has come to be known as an ``explosion''.  A third
point remains to be made and is the subject of this section.  Let us consider
a thought experiment.

Imagine that we are robots.  We are made out of metal, and our brains are
composed of large scale integrated circuits made of silicon or some other
semi-conductor.  Imagine further, that we have no experience of carbon
based life.  We have never seen it, never heard of it, nor ever contemplated
it.  Now suppose a robot enters the scene with a flask containing methane,
ammonia, hydrogen, water and a few dissolved minerals.  This robot asks our
academic gathering: ``Do you suppose we could build a computer out of this
material.''  The theoreticians in the group would surely say yes, and propose
some approaches to the problem.  But the engineers in the group would say:
``Why bother when silicon is so much better suited to information processing
than carbon.''

From our organo-centric perspective the robot engineers might seem naive,
but in fact I think they are correct.  Carbon chemistry is a lousy medium
for information processing.  Yet the evolutionary process embodies such a
powerful drive to generate information processing systems, that it was able
to rig up carbon based contraptions for processing information, capable of
generating the beauty and complexity of the human mind.  What might such
a powerful force for information processing do in a medium designed for
that purpose in the first place?  It is likely to arrive more quickly
at sophisticated information process than evolution in carbon chemistry,
and would likely achieve comparable functionality with a greater economy
of form and process.

\section{\bf HOW}

The Tierra system creates a virtual computer (a software emulation of
a computer that has not been built in hardware) whose architecture,
instruction set, and operating system have been designed to support
the evolution of the machine code programs that execute on that
virtual machine.  A network version of the Tierra system is under
development that will allow the passage of messages between Tierra
systems installed on different machines connected to the network,
via ``sockets''.

The instruction sets of the Tierran virtual computers will have some
new instructions added that allow the digital organisms to communicate
between themselves, both within a single installation of Tierra, and
over the net between two or more installations.  The digital organisms
will be able to pass messages consisting of bit strings, and will also
be able to send their genomes (their executable code) over the network
between installations of Tierra.

The network installation of Tierra will create a virtual sub-network
within which digital organisms will be able to move and communicate
freely.  This network will have a complex topology of interconnections,
reflecting the topology of the internet within which it is embedded.
In addition, there will be complex patterns of ``energy availability''
(availability of CPU cycles) due to the Tierra installations being run
as low priority background processes and the heterogeneous nature of the
real hardware connected to the net.  A miniature version of this
concept has already been implemented in the form of a CM5 version
of Tierra, which will be used to simulate the network version (Thearling
and Ray, submitted).

Consider that each node on the net tends to experience a daily cycle of
activity, reflecting the habits of the user who works at that node.  The
availability of CPU time to the Tierra process will mirror the activity
of the user, as Tierra will get only the cycles not required by the user
for other processes.  Statistically, there will tend to be more ``energy''
available for the digital organisms at night, when the users are sleeping.
However, this will depend a great deal on the habits of the individual
users and will vary from day to day.

There will be strong selective pressures for digital organisms to
maintain themselves on nodes with a high availability of energy.  To a
first approximation, this will involve daily migrations around the planet,
keeping on the dark side.  However, they need to evolve some direct
sensory capabilities in order to respond to local deviations from the
expected patterns.  When rich energy resources are detected on a local
sub-net, it may be advantageous to disperse locally within the sub-net,
rather than to disperse long distances.  Thus there is likely to be
selection to control the ``directionality'' and distances of movement
within the net.

All of these conditions should encourage the evolution of ``sensory''
capabilities to detect energy conditions and spatial structure on the
net, and also evolution of the ability to detect temporal patterns in
these same features.  In addition to the ability to detect these
patterns, the digital organisms need the ability to coordinate
their actions and movements in response to changing conditions.
In short, the digital organisms must be able to intelligently navigate
the net in response to the dynamically changing circumstances.

A primary obstacle to the evolution of complexity in the Tierra system
has been that in the relatively simple single node installation, a very
simple twenty to forty byte algorithm that quickly and efficiently copies
itself can not be beat by a much more complex algorithm, which due to
its greater size would take much longer to replicate.  There is just
no need to do anything more complicated than copy yourself quickly.
However, the heterogeneous and changing patterns of energy availability
and network topology of the network version will reward more complex
behavior.  It is hoped that this will launch evolution in the direction
of more complexity.  Once this trajectory has begun, the interactions
among the increasingly sophisticated organisms themselves should lead
to further complexity increases.

It is imagined that individual digital organisms will be multi-celled,
and that the cells that constitute an individual will be dispersed over
the net.  The remote cells might play a sensory function, relaying
information about energy levels around the net back to some ``central
nervous system'' where the incoming sensory information can be processed
and decisions made on appropriate actions.  If there are some massively
parallel machines participating in the virtual net, digital organisms
may choose to deploy their central nervous systems on these arrays of
tightly coupled processors.

\section{\bf HARVEST TIME}

The strategy being advocated in this proposal is to let natural selection
do most of the work of directing evolution and producing complex
software.  This software will be ``wild'', living free in the digital
biodiversity reserve.  In order to reap the rewards, and create
useful applications, we will need to domesticate some of the wild
digital organisms, much as our ancestors began domesticating the
ancestors of dogs and corn thousands of years ago.

The process must begin with observation.  Digital naturalists must
explore the digital jungle, observing and publishing on the natural
history, ecology, evolution, behavior, physiology, morphology, and
other aspects of the biology of the life forms of the digital ecosystem.
Much of this work will be academic, like the work of modern day tropical
biologists exploring our organic jungles (which I have been doing for
twenty years).

However, occasionally, these digital biologists will spot an interesting
information process for which they see an application.  At this point,
some individuals will be captured and brought into laboratories for closer
study, and farms for breeding.  Sometimes, breeding may be used in combination
with genetic engineering (insertion of hand written code, or code transferred
from other digital organisms).  The objective will be to enhance the
performance of the process for which there is an application, while
diminishing unruly wild behavior.  Some digital organisms will domesticate
better than others, as is true for organic organisms (alligators don't
domesticate, yet we can still ranch them for their hides).

Once a digital organism has been bred and/or genetically engineered to
the point that it is ready to function as an application for end users,
they will probably need to be neutered to prevent them from proliferating
inappropriately.  Also, they will be used in environments free from the
mutations that will be imposed on the code living in the reserve.
By controlling reproduction and preventing mutation, their evolution
will be prevented at the site of the end user.  Also the non-replicating
interpreted virtual code, might be translated into code that could
execute directly on host machines in order to speed their operation.

The organisms living in the biodiversity reserve will essentially be in
the public domain.  Anyone willing to make the effort can observe them
and attempt to domesticate them.  However the process of observation,
domestication and genetic engineering of digital organisms will require
the development of much new technology.  This is where private enterprise
can get involved.  The captured, domesticated, engineered and neutered
software that is delivered to the end user will be a salable product,
with the profits going to the enterprise that made the efforts to bring
the software from the digital reserve to the market.

It seems obvious that organisms evolving in the network-based biodiversity
reserve will develop adaptations for effective navigation of the net.
This suggests that the most obvious realm of application for these
organisms would be as autonomous network agents.  It would be much less
likely that this kind of evolution could generate software for control
of robots, or voice or image recognition, since network based organisms
would not normally be exposed to the relevant information flows.  Yet
at this point we surely can not conceive of where evolution in the digital
domain will lead, so we must remain observant, imaginative in our
interpretations of their capabilities, and open to new application
possibilities.

\section{\bf COMMITMENT}

Those who wish to support the digital biodiversity reserve by contributing
spare CPU cycles should be prepared to make a long-term commitment.
Nobody knows how long it will take for complex software to evolve in the
reserve.  However, a few years will likely be enough time to shake down
the system and get a sense of the possibilities.  If the desired complexity
does begin to evolve, then the reserve should become a permanent fixture
within the net.

The same problems are faced in the creation of reserves for organic
biodiversity.  Great effort and financial resources are required just
to establish the reserves.  However, that is only the first step.
The objective of the reserves is to limit the extent to which human
activity causes the extinction of other species.  The survival or
extinction of organic species is a process that is played out over vast
expanses of time: thousands or millions of years.  This means that if
our rain forest reserves should be converted into pastures or housing
developments a thousand years from now, they will have failed.

The organic companion proposal (Ray 1994b) is focused on the sustainability
issue.  The present strategy is to insure the long term survival of the
nature reserves by finding ways for the surrounding human populations
to derive an economic benefit from the presence of the reserves.  In
Costa Rica, at present, this can most easily be done through nature tourism.
In the future other economic activities may be more appropriate, or
perhaps some centuries or millennia in the future, humans will be willing
to protect other species without the motivation of self-interest.

Similar concerns apply to the sustainability of the digital reserve.
If the Tierra process provides no reward to those who run it on their
nodes, they are likely to terminate the process within a few days, weeks,
or months.  Such a short participation would be meaningless.  As an initial
hedge against this problem, a tool will be distributed to allow anyone to
observe activity at any participating node, from any node.  Yet even
this may not be enough, as such tools don't tell a lot about what is
going on.  To really know the interesting details requires greater effort
than most contributors of CPU cycles will have time for.

An even more serious problem is that experience with operation of the
system will certainly lead to redesign requiring reinstallation.  The
ideal situation would be to have the reinstallation done by the same
people who do the redesign.  However, this would be likely to require
that the designers of the reserve actually have accounts on the
participating nodes.  Where the designers don't have accounts, the
contributors would have to do the reinstallation themselves, and they
would likely tire of the chore.

The willingness of people to support the reserve for the long term
is likely to depend initially on the level of faith that people put
in the evolutionary process as a potential generator of rewarding
digital processes.  Eventually, if all goes well, the harvest of
some complex and beautiful digital organisms will provide rewards
beyond our imaginations, and should replace faith with solid proof
and practice.

\section{\bf CONTAINMENT}

The Tierra system is a containment facility for digital organisms.
Because Tierra implements a virtual computer, one that has never been
implemented in hardware, the digital organisms can only execute on the
virtual machine.  On any real machine, Tierran organisms are nothing but
data.  They are no more likely to be functional on a real computer than
a program that is executable on a Mac is likely to run on an IBM PC, or that
the data in a spread sheet is likely to replicate itself by executing on
a machine.

Similarly, the network version of Tierra will create a virtual sub-net,
within which the digital organisms will be able to move freely.  However,
the Tierran digital organisms will not access the real net directly.
All communication between nodes will be mediated by the simulation software
which does not evolve.  When Tierran organisms execute a virtual machine
instruction that results in communication across the net, that instruction
will be interpreted by the simulation software running on the real machine.
The simulation software will pass the appropriate information to a
Tierra installation on another machine, through established socket based
communication channels.  These socket communication channels will only
exist between Tierra installations at participating nodes.  The digital
organisms will not be able to sense the presence of real machines or
the real net, nor will they have any way of accessing them.

To further understand the nature of the system, consider a comparison
between the Tierra program and the mail program.  The mail program is
installed at every node on the net and can send data to any other node
on the net.  The data passing between mail programs is generated by
processes that are completely out of control: humans.  Humans are beyond
control, and sometimes actually malicious, yet the messages that they
send through the mail program do not cause problems on the net because
they are just data.  The same is true of the Tierra program.  While the
processes that generate the messages passing between Tierra installations
are wild digital organisms, the messages are harmless data as they pass
through the net.  The Tierra program that passes the messages
does not evolve, and is as well behaved as the mail program.

A related issue is network load.  We do not yet know the
level of traffic that would be generated by networked installations
of Tierra communicating in the manner described.  We will place hard
limits on the volume of communication allowed to individual digital
organisms in order to prevent mutants from spewing to the net.  As we
start experimenting with the system, we will monitor the traffic levels
to determine if it would have a significant impact on network loads.
If the loads are significant, additional measures will need to be taken
to limit them.  This can be done by charging the organisms for their
network access so that they will evolve to minimize their access.

\section{\bf REFERENCES}

\XP

Ray, T. S.  1991a.  An approach to the synthesis of life.
{\em In\/}: Langton, C., C. Taylor, J. D. Farmer, \& S. Rasmussen [eds],
Artificial Life II, Santa Fe Institute Studies in the Sciences of
Complexity, vol. X, 371--408.  Redwood City, CA: Addison-Wesley.

\rule[0pt]{3em}{.4pt}.  1991b.  Evolution and optimization of digital
organisms.  {\em In\/}: Billingsley K. R., E. Derohanes, H. Brown, III [eds.],
Scientific Excellence in Supercomputing: The IBM 1990 Contest Prize Papers,
Athens, GA, 30602: The Baldwin Press, The University of Georgia.

\rule[0pt]{3em}{.4pt}.  1994a.  An evolutionary approach to synthetic
biology: Zen and the art of creating life.  Artificial Life 1(1/2): 195--226

\rule[0pt]{3em}{.4pt}.  1994b.  A proposal to consolidate and stabilize
the rain forest reserves of the Sarapiqu\'{\i} region of Costa Rica.
Available by anonymous ftp: tierra.slhs.udel.edu [128.175.41.34] and
life.slhs.udel.edu [128.175.41.33] as tierra/doc/reserves.tex.

\rule[0pt]{3em}{.4pt}.  In press.  Evolution, complexity, entropy,
and artificial reality.  Physica D.

Thearling, Kurt, and T. S. Ray.  In press.  Evolving multi-cellular
artificial life.  Artificial Life IV conference proceedings.

\eXP

\newpage
\setcounter{page}{1}

\begin{center}
{\Large {\bf

A PROPOSAL TO CONSOLIDATE AND\\
STABILIZE THE RAIN FOREST\\
RESERVES OF THE SARAPIQUI\\
REGION OF COSTA RICA\\}}

\vspace{1cm}
{\large {\bf              TABLE OF CONTENTS\\}}
\vspace{.5cm}

\begin{tabular}{lll}
    & ABSTRACT                                      & 1\\

1.  & INTRODUCTION                                  & 2\\

2.  & COSTA RICA                                    & 2\\

3.  & NATURE TOURISM                                & 3\\

4.  & SARAPIQUI                                     & 3\\

5.  & CHILAMATE                                     & 4\\

6.  & WHAT IS NEEDED                                & 5\\

7.  & SARAPIQUI ASOCIACION FOR FORESTS AND WILDLIFE & 5\\

8.  & SARDINAL AND ARREPENTIDOS HILLS               & 6\\

9.  & FINCA LA MARTITA                              & 7\\

10. & THE PLAN                                      & 8\\

11. & DONOR'S PROGRAM                               & 9\\

12. & LETTER FROM THE NATURE CONSERVANCY            & \\
    & (HOW TO CONTRIBUTE)                           & 10\\
\end{tabular}

\end{center}

\begin{abstract}

The proposed project aims to prevent the imminent destruction of
some of the last remaining large areas of rain forest in the
Sarapiqu\'{\i} region of Northern Costa Rica, and at the same time,
to use those forests to establish a conservation economy through a
community based nature tourism project.  The conservation economy is
needed to insure the long term stability of both the present and future
protected areas of Sarapiqu\'{\i}, by creating an economic interest group
within the human populations surrounding the protected forests.  The
overall project contains six stages, which will be executed in succession,
as funding reaches the corresponding target levels.  The Nature Conservancy
(TNC) is acting as fiscal agent, and the Costa Rican government has expressed
a willingness to expropriate the target properties if necessary.  This
project is being organized and supported by a coalition of conservation
organizations: Sarapiqu\'{\i} Association for Forests and Wildlife,
OTS (The Organization for Tropical Studies), TNC, FUNDECOR (The Foundation
for the Development of the Central Volcanic Mountain Range), COMBOS (The
Conservation and Management of Tropical Forests), and ABAS (The Association
for the Environmental Well being of Sarapiqu\'{\i}).

\end{abstract}

\pagebreak

\LP
{\large {\bf 1. Introduction}}
\eLP

Conserving biodiversity is a long-term enterprise.  One means of
conserving biodiversity is through the creation of biological
reserves, whether they be national or private.  These reserves must
remain intact for thousands of years to be meaningful in protecting
species from extinction.  If possible, they must also include
surrounding buffer zones and be connected, through a network of
biological corridors, to other reserves and/or conservation areas
in the region.  In the past decades, many tropical rain forest
reserves have been created worldwide, through great effort and
involving many individuals and organizations.  However, the most
difficult part of the work lies ahead.

Creating biodiversity reserves is only the first step.  The necessary
conditions must be created that will ensure their long-term survival.
Although we cannot hope to manipulate events thousands of years in the
future, we must recognize the challenge now and move in what appear
to be the most promising directions.

It is critical that the surrounding human populations want the
reserves to exist or the land will eventually be converted to other
uses.  A powerful means of creating an interest group is through
economic incentives.  Ways must be found to create a ``conservation
economy'' through the use of intact rain forest.

\LP
{\large {\bf 2.  Costa Rica}}
\eLP

The diverse and once abundant natural resources of the Central
American region are being over-exploited at an alarming rate, and
if the situation is not remedied, continued deterioration of the
economic, environmental and social well-being of the area will
result.  The degradation of the natural environment is due to a
combination of factors including accelerated human population
growth, the exploitation of forests by lumber companies, and the
conversion of forest to other land uses such as pasture, coffee,
sugar cane and banana plantations.

In Costa Rica, the remnants of the original forested environment
exist in national parks and reserves, or as forest patches and
strips in already colonized areas.  Privately owned forests
represent the majority of the 300,000 hectares of unprotected
primary forest remaining in Costa Rica.  By viewing satellite and
aerial photos, it is evident that most of these forest remnants are
rapidly disappearing.  If nothing is done to conserve them, it is
most likely that they will disappear in the next few years.

The northeastern part of Costa Rica has been a center of
conservation activity for many years.  Within this region are
located various projects such as the Barro Colorado Wildlife
Refuge, Tortuguero National Park and the biological corridor being
formed that will connect the (above mentioned) park and wildlife refuge,
the SiAPaz National Park shared by Nicaragua and Costa Rica, the
Ca\~{n}o Tambor area where a new national park is being planned called
Maquenque National Park, Braulio Carrillo National Park with its new
park extension connecting it to the La Selva biological research
station, etc.  The Paseo Pantera Project, with its vision of
creating a Meso-American biological corridor from Mexico to
Colombia, plans to include all the above mentioned projects, with
help from local conservation groups, as links within the greater
Meso-American biological corridor, together with any other forested
lands still existing in the area, be they in the form of private
eco-tourism projects, reserves, refuges, parks, etc.

\LP
{\large {\bf 3. ``Nature Tourism''}}
\eLP

Costa Rica, because of its exemplary system of national parks and
reserves, and because it is a peaceful, democratic country with
educated, friendly people, has created a flourishing ``nature
tourism'' industry.  Large numbers of people from all levels of Costa
Rican society are employed in the ``eco-tourism'' business.  In the
areas immediately surrounding the parks and reserves, people from
the local communities are employed as naturalists, guides, park
guards, and hotel and restaurant employees.  The creation of this
``conservation economy'' has caused a change in attitudes among the
local populations.  Costa Rica now has approximately 140 grass-roots
environmental organizations, where 20 years ago, there were almost
none.

Thus far, nature tourism has worked beautifully in Costa Rica in
creating a ``conservation economy'' and an interest group supportive
of maintaining the system of nature reserves.  However, various
problems exist with the current state of affairs.  Most of the
nature tourism projects are foreign owned.  While these projects generate
a lot of employment, there is a certain resentment among the local
people with having to work for a salary for foreigners, who are
perceived as carrying the profits off to their home countries and
not investing much if any of it into the local communities.
Historically, the needs of the communities in rural areas have
often been ignored by conservation groups and tourism companies,
even though their cooperation is absolutely essential in the
running of the hotels, restaurants, shops, etc. and in preventing
the forested lands from being overrun by illegal loggers, squatters
and poachers.

The main obstacle which has prevented Costa Ricans from creating
their own nature tourism projects is simply lack of capital,
particularly among the rural populations actually living around the
parks and reserves.  The people in these communities have gained a
great deal of experience in the past decade in all aspects of
operating nature tourism projects, as well as a wealth of knowledge in
the importance of protecting their surrounding natural resources.
Recently, they have expressed a desire to mount their own projects,
but lack the funds to do so.

\LP
{\large {\bf 4.  Sarapiqu\'{\i}}}
\eLP

Sarapiqu\'{\i} is a ``canton'' (county) in the northeastern province of
Heredia, Costa Rica.  It occupies 85\% of the total land area of the
Heredia Province.  The town of Puerto Viejo is the population center
and political seat of the Sarapiqu\'{\i} county.  It is located at the
confluence of the Sarapiqu\'{\i} and Puerto Viejo rivers, 30 kilometers
south of the Nicaraguan border.  The town is in an area that was formerly
lowland rain forest.  The land to the south rises very gradually to
3,000 meters in elevation over a distance of 35 kilometers.

Sarapiqu\'{\i} has been a center of conservation activity in Costa Rica
for many years, originally due to the presence of La Selva just
south of Puerto Viejo.  La Selva is the international biological
research station owned by the Organization for Tropical Studies, a
consortium of approximately 50 universities in the United States
and Costa Rica.  La Selva works closely with the surrounding
community by involving them directly in the research being carried
out by using their farms as study sites while at the same time
educating them in new conservation techniques, new viable native
species for reforestation, etc.  La Selva also offers a course to
train locals as naturalist guides.

In the 1980s, the Braulio Carrillo National Park was created in the
mountains to the south of La Selva.  The Park was subsequently
extended 20 kilometers northward to connect with the research
station.  This extension was accomplished through an international
effort that raised two million dollars, which included a one
million dollar challenge grant from the John D. and Catherine T.
MacArthur Foundation.  The protected areas in the central region
of Sarapiqu\'{\i} are highlighted in green on map A.

The Braulio Carrillo National Park and its extension occupy the
heart of the Sarapiqu\'{\i} county and have formed the nucleus for a
variety of conservation efforts and nature tourism projects.  FUNDECOR
(the Foundation for the Development of the Central Volcanic
Mountain Range -- translated from its name in Spanish) works closely
with land owners in the area promoting sound farm management
techniques and conservation practices.  COMBOS (The Conservation and
Management of Tropical Forests) is also helping community groups
and private individuals in the area with various projects promoting
the conservation and management of private forests.  ABAS (The
Association for the Environmental Well being of Sarapiqu\'{\i}) is a
strong environmental group working in the Sarapiqu\'{\i} region to halt
the poaching of endangered animals, hazardous waste dumping by
local banana plantations, and in general, watching for any other
environmental offenses which exist in the region while at the same
time promoting conservation of the rich natural resources of the
area and consolidation of the Maquenque National Park at Ca\~{n}o
Tambor.

Various nature tourism projects include Selva Tica, Islas del R\'{\i}o,
Rancho Leona which provides kayak trips on the Sarapiqu\'{\i} River,
Rara Avis, Selva Verde and their community learning center, various
individuals with small private reserves, and entrepreneurs selling
``eco-art'', secondary products from the forest, etc.  The most well
known of these projects are ``Rara Avis of Costa Rica'', owned by Amos
Bien (a North American) and ``Selva Verde Lodge'', owned by Giovanna
Holbrook and Juan Holbrook (North Americans), Gainesville, Florida-based
tour operators.

A problem which exists for the nature tourism businesses in the area
is that the lower elevations of the Braulio
Carrillo National Park are difficult to access, making it
inconvenient for direct use by the nature tourism projects.  For
example, the Rara Avis property borders directly on the Park,
however, access to the hotel involves a ride in a tractor-pulled
cart that takes four hours, more if it is raining.  There are also
a number of small nature tourism projects in the region which include
no forest of their own and, therefore, cannot offer forest access
to their clients.  La Selva is for scientific use and Selva Verde
Lodge is only available to residents and guests of the Lodge.  The
Park itself is too remote for tourists to access easily.

\LP
{\large {\bf 5. Chilamate}}
\eLP

The community of Chilamate exhibits a rare coincidence of forest
and road.  Chilamate is located where the Sarapiqu\'{\i} River passes
through the southern extension of the Sardinal Hills.  Probably
because of the ruggedness of the terrain, the land has not yet been
deforested.  The main road from Puerto Viejo to the capital city of
San Jos\'{e}, parallels the river, thus in Chilamate, there is easy
access to the forests of the Sardinal Hills.  In addition, electric
and telephone lines run along the road through Chilamate.  This
coincidence of forest with road access and basic utilities makes
Chilamate an ideal location for the development of nature tourism
projects, which is why Selva Verde Lodge and Islas del R\'{\i}o are
already located there.

The Selva Verde property includes 180 hectares of forest to the
south of the road, extending to within 1.7 kilometers of the
Braulio Carrillo National Park extension.  In addition, there are
three small private reserves to the north of the road, each
approximately 800 meters long and 200-300 meters wide.  These
reserves are currently separated by two strips of cow pasture, each
100 meters wide.  It is expected that in the coming years, these
pastures will be purchased by the owners of the small reserves and
the forests allowed to regenerate.  This will result in the creation
of one single reserve of approximately one square kilometer in
size.  Of the three private reserves, Giovanna Holbrook and Juan
Holbrook own the western one, Tom Ray (a North American) owns the central
reserve, and Isaias Alvarado (a native of Sarapiqu\'{\i} and resident of
Chilamate) owns the eastern one.

The areas already under protection in the Chilamate area are
highlighted in green on map B.  The areas proposed for protection
are highlighted in blue and pink.

\LP
{\large {\bf 6.  What is needed}}
\eLP

The region of Sarapiqu\'{\i} needs an area of forest easily accessible
to tourists, which is owned and operated by Costa Rican nationals
who are members of the local community.  This property should be
large enough to be ecologically viable, in the sense that it can
preserve populations of flora and fauna in the area important to
tourism (e.g., birds, monkeys, etc.).  Access to this forest should
be made open to any of the nature tourism projects in the area,
including the small ones which do not currently have forest access.
Ideally, a tourist lodge would be constructed in conjunction with
this property, which would also be owned and operated by members of
the community.

In addition, the narrow corridor connecting the La Selva Biological
Station to the Braulio Carrilo National Park needs to be widened to
better protect the fauna and flora of the middle elevations, and to
facilitate the movements of the altitudinally migrating species
(particularly birds).

\LP
{\large {\bf 7.  Sarapiqu\'{\i} Association for Forests and Wildlife}}
\eLP

The non-profit Asociaci\'{o}n Bosque y Vida para Sarapiqu\'{\i}
(Sarapiqu\'{\i} Association for Forests and Wildlife) was formed,
with the help of COMBOS, in September, 1993 in order to address the
needs described above.  The charter includes the following goal and
objectives:

Goal: Promote the conservation and sustainable development of the
natural and human resources in the county of Sarapiqu\'{\i}.

\LP
Objectives:
\eLP

\XPNS
1)   The protection of the hills, wetlands and buffer zones in the area.

2)   Natural forest regeneration.

3)   To promote the creation of biological corridors.

4)   Development of educational projects.

5)   Protection of the rivers.

6)   To stimulate the creation of private reserves or refuges.

7)   Promotion of eco-tourism.

8)   Promotion of participative planning and land ordinance in the county.

9)   To promote reforestation using species native to the area.
\eXPNS

Membership in the Association is available to persons 16 years of
age or older for a fee of 1,000 colones (approximately \$7) and an
annual membership fee, also 1,000 colones.  The Association is
managed by the Executive Committee which is elected every two
years.  Checks can only be signed by the President and Treasurer.
The first members elected to the Executive Committee were:

\vspace{.5cm}

\begin{tabular}{llcll}

Name               & Office         & Age & Profession       & Residence\\
\\
Joel Alvarado      & President      & 30  & Naturalist       & Chilamate\\
Alexander Martinez & Vice-President & 44  & Environmentalist & Puerto Viejo\\ 
Gonzalo Ramirez    & Treasurer      & 41  & Economist        & San Jose\\
Nelci Oconitrillo  & Secretary      & 20  & Student          & Cristo Rey, PV\\
Leona Barrantes    & Officer        & 44  & Business Woman   & La Virgen\\
Jaime Alvarado     & Fiscal         & 33  & Naturalist       & Chilamate\\

\end{tabular}

\LP
{\large {\bf 8.  Sardinal and Arrepentidos Hills}}
\eLP

The Sarapiqu\'{\i} Association for Forests and Wildlife has identified
the Sardinal Hills, The Arrepentidos Hills and the more than two square
kilometers area of wetlands connecting the two as an ideal site for
preservation because of its high ecological value and unsuitability for
conventional development.  The area also
has the natural infrastructure necessary for a regional nature tourism
project.  The hills-wetlands complex is a unique ecosystem in Costa
Rica and covers multiple life zones extremely rich in biodiversity.
At their southern end, the Sardinal Hills join the three private
reserves in Chilamate and are thus easily accessed through
Chilamate.  The Arrepentidos Hills come down to within a few meters
of the road to the north of Puerto Viejo, and on their southern
side, come within approximately 200 meters of the northernmost part
of La Selva.  The Arrepentidos Hills are easily accessible from the
main street in Puerto Viejo, and thus would allow the hotels in
Puerto Viejo to provide forest access to their guests.

The three private reserves in Chilamate are separated from the
Holbrook reserve only by the road and a river.  Ideally, the
Holbrook reserve would be extended 1.7 kilometers to the southeast
to connect it to the Braulio Carrillo National Park extension,
creating a biological corridor between the Sardinal-Arrepentidos
Hills-wetlands complex and the national park.  This connection would
also conceivably be made between the La Selva property and the
Arrepentidos Hills.  This area could then be included as another
vital link in the greater Meso-American Biological Corridor
mentioned earlier.  The Sardinal-Arrepentidos Hills-wetlands complex
is one of the very few intact, ``healthy'' areas left north of the
road that could be linked to the Meso-American Biological Corridor.

The areas proposed for inclusion in this phase of the project are all
either rugged hills or wetlands, not suitable for agricultural, industrial
or residential developments.  These areas are highlighted in pink
or blue on the enclosed map B.

\LP
{\large {\bf 9. Finca La Martita}}
\eLP

It is an unfortunate reality that the nature tourism industry which
has developed under the ideal conditions found in Chilamate, is
based on the presence of large areas of unprotected forested land,
which are, in fact, in immediate danger of destruction.  The bulk of
the area of the Sardinal Hills is contained within one single 670
hectare property, Finca La Martita (highlighted in pink on the
enclosed maps).  The current owner has applied for a logging permit
to cut and sell the lumber in the Sardinal Hills area of the
property in order to pay part of the purchase price still owed to
the sellers.

This development has greatly alarmed the community of Chilamate and
led to the formation of the Sarapiqu\'{\i} Association for Forests
and Wildlife.  On July 21, 1993 the members of the community sent a
petition to Dr.\ Orlando Morales, the Minister of Natural Resources
in Costa Rica.  The petition reads as follows (translated from
Spanish):

\LP
Very Esteemed Minister:
\eLP

\begin{quote}

     Let this letter serve in the first place to greet you and
congratulate you for your dedicated efforts in the defense of our
natural resources.  We, the neighbors of Sarapiqu\'{\i}, wish to request
your help in preventing the destruction of the forests of Finca
La Martita.  We understand that the owner of the property has
requested, and may have already obtained, a logging permit to cut
and sell the lumber.  The property is located in Chilamate and
includes the greater part of the Sardinal Hills.  It includes 670
hectares, half of which is primary forest, a fourth of which is
secondary forest, and the rest pasture.  Its complete deforestation
would not only contribute to aggravating the problems of erosion
within the zone, but gravely enhance its already precarious
situation, a consequence of the uncontrolled exploitation by cattle
ranchers and banana plantations.

     The region of Sarapiqu\'{\i} has developed a strong eco-tourism
industry, by which many of us live.  We would like to see this
industry grow, rather than destroying the nature which forms its
base.  There already exist in Chilamate some successful eco-tourism
projects which offer the visitor access to the forests contiguous
to those of La Martita.  In reality, the forests of La Martita
represent the largest part of the forested area in Chilamate.  Their
destruction would do great harm to our eco-tourism industry.  It
would also signify the elimination of the monkeys, wild cats, and
many other mammals and birds from the forests of Chilamate.  We
would like to see the Finca La Martita working together with us in
eco-tourism.  This would mean the expansion of eco-tourism in
Chilamate, rather than its disappearance.

     We, the neighbors of Sarapiqu\'{\i}, are willing to make every
effort necessary to develop eco-tourism in La Martita, if you give
us the opportunity.  It worries us greatly to know that there exists
an immediate danger of the destruction of its forests, if already
granted (we do not know if the permission that authorizes this
ecological disaster was properly acquired).  We wish to ask you,
with all respect, to review the logging permit that may have been
given to the owner of La Martita, and consider the possibility of
revoking it.  Give us the opportunity to put La Martita to work in
eco-tourism, and in this way protect the industry of eco-tourism
that already exists in the region, which would be gravely damaged
if the animal species that presently circulate freely in the area
are reduced or all together pushed out of the area.

     We, and the future generations of our country, thank you for
all your efforts in solving this serious problem.

     We request that you notify us of your decision and all
information relative to this matter, at the office of Agustin
Atmetlla Cruz, lawyer and notary, located in San Pedro, 100 meters
east and 100 meters south of the Banco Popular, Apartado 7723-1000
San Jos\'{e}, telephone 53-0889.

\end{quote}

\begin{center}
                          Respectfully,\\
\vspace{.5cm}
\begin{tabular}{ll}
     Name                          &        Profession\\
\\
     Thomas Ray                    &        University Professor\\
     Leona Wellington Barrantes    &        Business Woman\\
     Gonzalo Ramirez Guier         &        Economist\\
     Francisco Madrigal            &        Farmer\\
     Orlando Vargas R.             &        Naturalist\\
     Alexander Martinez            &        Environmentalist\\
     Isaias Alvarado               &        Environmentalist\\
     Ileana Molina B.              &        Naturalist\\
     Joel Alvarado Chavarria       &        Naturalist\\
     Ken Upcraft                   &        Business Man\\
     Robert Wells                  &        Business Consultant\\
\end{tabular}
\end{center}

Minister Morales responded that if the request for a logging permit
is properly prepared (which it appears to be), he cannot legally
deny it.  However, he has said that he would be willing to delay the
granting of the permit in order to allow the Association time to
raise funds to purchase the property.  He has also said that if the
owner is not willing to sell the land to the Sarapiqu\'{\i} Association
for Forests and Wildlife, the government would be willing to expropriate
the property, if the money to pay for the property can be raised. 

\LP
{\large {\bf 10.  The Plan}}
\eLP

The Association has outlined an ambitious multi-stage, long-term
plan which aims to solve the immediate problem of La Martita, and
help with the development of a locally owned and operated
nature tourism industry while at the same time consolidating existing
reserves and promoting the linkage of local areas to the Meso-American
Biological Corridor.  Most of the stages of the plan will require
considerable financial resources.  The objective is to obtain these
through international fund raising efforts from philanthropic sources.
However, once the nature tourism project becomes well developed, some
of the profits will be re-invested into future stages of the plan.
In brief, the plan includes the following stages:

\XPNS
1)   Purchase of Finca La Martita by the Sarapiqu\'{\i} Association
     for Forests and Wildlife.  670 hectares.  Estimated cost: \$440,000

2)   Establish a community-based nature tourism project at La Martita
     which will function as a cooperative.
     Cost: \$136,000 for a one year startup.

3)   Join the Holbrook reserve at Selva Verde Lodge to the Braulio
     Carrillo National Park.  345 hectares.  Estimated cost: \$249,953

4)   Widen the corridor connecting the La Selva Biological Station
     to the bulk of the Braulio Carrillo National Park (based on the
     October 1993 proposal from the Fundacion de Parques Nacionales).
     596 hectares.  Cost: \$431,398

5)   Purchase the remainder of the Sardinal Hills, the wetlands, and
     the Arrepentidos Hills areas and join them to the La Martita
     nature tourism project.  1,330 hectares.  Estimated cost: \$818,685

6)   Widen the upper elevation portions of the corridor connecting the
     La Selva Biological Station to the Braulio Carrillo National Park.
     640 hectares.  Estimated cost: \$436,800
\eXPNS

Most of the costs listed above are estimated values.  The Ministry of
Finance of Costa Rica is currently in the process of appraising each of
the properties targeted for acquisition.  The budget can not be finalized
until the appraisal is completed.  However, it is felt that the figures
listed above are realistic.

The purchase of Finca La Martita is an essential step in promoting
community-based nature tourism projects and conservation efforts in
Costa Rica.  The creation of the Sarapiqu\'{\i} Association for Forests
and Wildlife nature tourism project will be an example to other
communities, demonstrating that it is possible for rural communities to
directly manage and benefit from nature tourism projects.  It also
provides a type of empowerment to local communities.  With actual
ownership of a property, community members have more incentive to
work together and manage the local natural, as well as human, resources
as efficiently and profitably as possible for their own benefit and
well being.

\LP
{\large {\bf 11.  Donor's Program}}
\eLP

Donors and allies (allies are persons who recruit quailfying donors)
will be offered a token gift of a free night in the Martita hotel, for
each ten thousand dollars contributed.  More importantly, an effort
will be made to coordinate the visits of these persons to the project,
so that the major donors and allies (of all nationalities) will have
an opportunity to meet one another, and to receive a guided tour of
the rainforests and conservation projects of the Sarapiqu\'{\i} region.
Tom Ray, one of the organizers of this project, will guide the tour
along with the very talented naturalists who will be managing the
project in Costa Rica.

\newpage

The maps referred to in the rainforest proposal are not included in
this text file, but are available on request from Tom Ray, and will
soon be placed in the ftp site in electronic form

\newpage

\begin{center}
{\large {\bf The Nature Conservancy}}\\
Latin America / Caribbean Region\\
International Headquarters\\
1815 North Lynn Street\\
Arlington, Virginia 22209\\
Tel: 703-841-4860\\
Fax: 703-841-4880\\
\end{center}

\LP
January 24, 1994

Dr. Thomas Ray\\
ATR Human Information Processing Research Laboratories\\
2-2 Hikaridai, Seika-cho Soraku-gun\\
Kyoto 619-02 Japan\\
Fax: (81)-7749-5-1008

Dear Dr. Ray,

In response to our several conversations and exchange of documents, I
would like to confirm, in writing, The Nature Conservancy's interest
in assisting your Sarapiqui project.

The Conservancy has been working in this part of Costa Rica for over
10 years and it is becoming increasingly clear that the private sector
will have to assume much of the burden of protecting forested lands
outside the Braulio Carillo National Park complex.  Your project,
carried out through a local community-based association, seems to
provide a model for how this can be done in a systematic and phased
approach.

I have just returned from Costa Rica where I was able to visit this
site with our local attorney, Robert Wells, and meet with various
leaders of the local community.  The project does indeed offer an
unusual opportunity to involve community residents in helping to plan
for the optimal use of the acquired parcels in ways which maximize the
conservation and campatible development values of the land.

The Conservancy would be willing to act as a fiscal agent for contributions
from donors interested in contributing through a U.S. c(3) tax exempt
organization and we would be glad to oversee the purchase of the land
targeted for protection.

Such land, once acquired, would be transferred to the non-profit
Sarapiqui Association for Forests and Wildlife with appropriate
conservation easements or deed restrictions to ensure its protection
in perpetuity.

I agree that this project should be executed in stages as sufficient
funds are raised to ensure the successful completion of each stage.
Donors can select which phase they are interested in supporting and
pledge accordingly.  They would not have to make the actual contribution
until all the funds required for that phase are pledged.

Contributions can be in the form of cash, wire transfers, appreciated
securities or even gifts or real estate and life income estates.  Each
method has its own tax advantages.  For example, a donation of stock
that has been held for over one year and that has increased in value
can be donated without having to pay capital gains taxes.  In addition,
the charitable contribution deduction is equal to the full market value
of the stock and may be taken over a six year period if a donor's
annual maximum limit has been attained.  Please let me know if you
require detailed information on any of these approaches.

Once a donor decides to contribute, they should send me a letter stating
the amount of their pledge, and the method of payment they prefer (cash,
wire, appreciated securities etc.).  The letter should state that the
donation is for the ``Sarapiqu\'{\i}'' project, and if they desire, should
specify earmarking (either positive or negative) for specific stages
of the project.  When the pledges reach the required target level, I will
then contact the donor with a letter calling in the pledge, and giving
specific instructions for their chosen method of payment.

I hope this covers the basic points.  I look forward to working with you
and our Costa Rican attorney in helping to put this important project
together.

Sincerely,

Randall K. Curtis\\
Director\\
Conservation Finance and\\
Costa Rica Country Programs
\eLP

\end{document}
