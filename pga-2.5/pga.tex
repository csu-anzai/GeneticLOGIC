\documentstyle[a4,12pt]{article}

\setlength{\oddsidemargin}{-0.5in}
\setlength{\evensidemargin}{-0.5in}
\setlength{\textwidth}{7in}

\title{About PGA v2.5}
\author{Peter Ross}
\date{26 Sep 1993}

\begin{document}
\maketitle

PGA was originally written by Geoffrey H. Ballinger and much modified
and extended by Peter Ross, both at the University of Edinburgh. It is a
simple testbed for multi-population genetic algorithms, with a choice of
six well-known problems and a range of options. It is useful as a first
introduction to genetic algorithms, but a package such as John
Grefenstette's Genesis provides much more flexibility.

\section*{The problems}

In each of the problems below a chromosome (potential solution of the
problem) is a string of 0s and 1s, which is decoded in a problem-specific way.
The function value itself is taken as the `fitness function' of a
chromosome, so the maximum of the function is also the maximum fitness.

\begin{description}
\item[max] The problem is to maximise the number of 1-bits in the
  chromosome. The fitness of a chromosome is just the number of
  1-bits.

\item[dj1] De Jong's first function: the problem is to maximise the
  function
\[
f(x_{1},x_{2},x_{3}) = 100 - x_{1}^{2} - x_{2}^{2} - x_{3}^{2}
\] 
  An $N$-bit chromosome is interpreted as three $N/3$-bit integers (last
  two bits are ignored) and each integer is then rescaled to be a real
  number in the range $-5.12$ to $+5.12$. The theoretical maximum
  fitness is of course 100.

\item[dj2] De Jong's second function: the problem is to maximise the
  function
\[
f(x_{1},x_{2}) = 1000 - 100(x_{0}^{2}-x_{1})^{2} - (1-x_{0})^{2}
\]
  An $N$-bit chromosome is interpreted as two $N/2$-bit numbers, which
  are then rescaled to be reals lying in the range $-2.048$ to $+2.048$.
  The theoretical maximum fitness is 1000.

\item[dj3] De Jong's third function: the problem is to maximise the
  function
\[
f(x_{1},x_{2},x_{3},x_{4},x_{5}) = 25 -  \sum_{i=1}^{5}  \mbox{floor}(x_{i})
\]
  where the \verb|floor| function returns the largest integer not larger
  than its argument. An $N$-bit chromosome is interpreted as five $N/5$-bit
  numbers, which are then rescaled to lie in the range $-5.12$ to $5.12$.
  This function has many flat plateaux; it would cause a hill-climbing
  search algorithm severe problems.
  The theoretical maximum fitness is 55.

\item[dj5] De Jong's fifth function: the problem is to maximise the
  function
\[
f(x_{1},x_{2}) = 500 - \frac{1}{0.002 + \sum_{j=1}^{25} \frac{1}{j + 
   (x_{1} - a_{1j})^{6} + (x_{2} - a_{2j})^{6}}}
\]
  where an $N$-bit chromosome is interpreted as two $N/2$-bit
  numbers, which are then rescaled to lie in the range $-65.536$ to
  $65.536$. The numbers $a_{ij}$ are constants, typically taken from the
  set $(0, {\pm}16, {\pm}32)$, so that the function has 25
  equally-spaced and very pronounced maxima, only one of which is the
  true global maximum. A hill-climbing search algorithm would be very
  unlikely to find the global maximum. The theoretical maximum fitness
  is about $499.001997$.

\item[bf6] A modified form of the function `binary F6' used by Schaffer
  et al (see {\em Proceedings of the Third International Conference on
  Genetic Algorithms, 1989}) as a test for comparing efficiency of
  different forms of genetic algorithm. The problem is to maximise the
  function
\[
f(x_{1},x_{2}) = \frac{(\sin(\sqrt{x_{1}^{2} + x_{2}^{2}}))^{2}}{
   1 + 0.001(x_{1}^{2} + x_{2}^{2})^{2}}
\]
  where an $N$-bit chromosome is interpreted as two $N/2$-bit numbers,
  which are then rescaled to lie in the range $-100$ to $100$.
  This function has many maxima, arranged in concentric rings
  about the origin; all points on the ring closest to the origin 
  (but not the origin itself) are the true global maxima. Each ring of
  local maxima is separated from the next by a ring of local minima.
  The theoretical maximum fitness of this function is about $0.994007$.

\item[knap] A one-dimentsional knapsack problem. A target integer and
  a further set of integers are read in from a file called `weights';
  the size of the set can be as large as the size of a chromosome. The
  input routine skips non-integers, so you can include textual
  comments if you like; however, the very first item in the file must
  be the target value. The task is to find a subset whose sum is as
  close to the given target as possible, preferably exactly equal to
  it. The fitness function is
\[
f(x_{1},\cdots,x_{n}) = \frac{1}{1 + 
     \left| \mbox{target} - \sum_{i=1}^n x_i \right|}
\] 
  so that it has maximum value $1.0$, and is $0.5$ or less unless
  the target is exactly obtained.

\item[rr] The `Royal Road' function(s). In this problem each
  chromosome is regarded as being composed of regularly-spaced
  non-overlapping blocks of $K$ bits, separated by a number of
  irrelevant bits. Various parameters control the scoring: $b$, $g$,
  and $m^*$ are integers and $u^*$, $u$ and $v$ are real numbers. The
  low-level blocks are of size $b$ bits, with $g$ irrelevant bits
  making up a gap between each. Each low-level block scores 0 if
  completely filled, or $mv$ if it contains $m$ bits and $m \leq m^*$,
  or $-mv$ if there are more than $m^*$ bits set but the whole block
  is not filled. Thus the low-level blocks are mildly deceptive. In
  addition there is a hierarchy of completed blocks which earn bonus
  points. The hierarchy has a number of levels; level 1 is the lowest,
  and level $(j+1)$ has half the number of blocks that level $j$ has -
  its blocks are adjacent pairs from the level below. If level $j$ has
  $n_j > 0$ `filled' blocks then it earns a bonus score of $u^* +
  (n_j-1)u$; thus $u^*$ is a special bonus for getting at least one
  filled at that level. The topmost layer of the hierachy has one
  block, which is filled if and only if every block in the lowest
  level is filled.

  See the paper by M.Mitchell, S.Forrest and J. Holland, ``The royal
  road for genetic algorithms'' in (eds) F.J.Varela and P.Bourgine,
  {\em Towards a Practice of Autonomous Systems: Proceedings of the
  First European Conference on Artificial Life}, MIT Press, 1992
  and a later paper by Mitchell and Holland available by FTP from
  santafe.edu in /pub/mm and in {\em Foundations of Genetic Algorithms
  2}, ed.\ L.D.Whitley, Morgan Kaufmann, 1993.  These papers report
  that such `royal road' functions are very hard for some GAs and
  analyse why. See also the challenge issued by John Holland in the GA
  list, vol 7 no 22.

  The various parameters can be specified in a file named
  \verb|rrdata|; the program ignores non-numeric text and expects
  the parameters to occur in a specific order. For example, the file
  might contain this:
\begin{center}
\begin{verbatim}
These are the defaults specified in Holland's challenge:

Blocksize (int) 8    Bits in a block
Gapsize (int)   7    Irrelevant bits between blocks
m* (int)        4    Threshold for reward in low-level blocks
u* (double)     1.0  First block in a higher level earns this
u (double)      0.3  Later blocks in a higher level earn this
v (double)      0.02 Reward-or-penalty/bit in low-level block

Note that Holland specifies number of levels (k) rather than
a gapsize so that there are two-to-the-k blocks at the lowest
level.

This file is read by pga when you specify -err for Royal Road
functions.  It looks for numbers in this file, in the order 
and of the types shown above. Non-numeric text is  ignored.
\end{verbatim}
\end{center}
  If the file is absent, or there are fewer than six numbers,
  the missing numbers default to the values shown here.
\end{description}

\section*{Running the program}

If you run the program with the \verb|-h| option you get brief help
as shown in figure \ref{fig:help}.
\begin{figure}[p]
\begin{center}
\begin{verbatim}
% pga -h
PGA: parallel genetic algorithm testbed, version 2.5
   -P<n>    Set number of populations. (5)
   -p<n>    Set number of chromosomes per population. (50)
   -n<n>    Set chromosome length. (32)
   -l<n>    Set # of generations per stage. (100)
   -i<n>    Set reporting interval in generations. (10)
   -M<n>    Interval between migrations. (10)
   -m<n>    Set bit mutation rate. (0.02)
   -c<n>    Set crossover rate (only for `gen'). (0.6)
   -b<n>    Set selection bias. (1.5)
   -a       Adaptive mutation flag (FALSE)
   -t       Twins: crossover produces pairs (FALSE)
   -C<op>   Set crossover operator. (two)
   -s<op>   Set selection operator. (rank)
   -r<op>   Set reproduction operator. (one)
   -e<fn>   Set evaluation function. (max)
   -S<n>    Seed the random number generator. (from clock)
   -h       Display this information.
   <file>   Also log output in <file>. (none)

   Crossover operators ... one, two, uniform.
   Selection operators ... rank, fitprop,
                           tnK (K=integer > 1),
                           tmK (K=integer > 1).
   Reproduction operators ... one, gen,
                              ssoneN (N=integer > 0),
                              ssgenN (N=integer > 0).
   Evaluation functions ... max, dj1, dj2, dj3,
                            dj5, bf6, knap,
                            rr.
\end{verbatim}
\caption{PGA help summary}
\label{fig:help}
\end{center}
\end{figure}
The default values are shown in brackets. The program displays the
chosen values of the various parameters, and runs the genetic algorithm
using the chosen problem and parameters for the appropriate number of
generations per population as set by the \verb|-l| (for `limit') option.
At each reporting interval, as set by the \verb|-i| option, the display
of the average and best fitness is updated. If you name a file on the
command line and it can be opened, the same information at each
reporting interval is appended to the file, for your later analysis.
Note that it is possible to generate {\em large} files quickly by this
means, so please tidy up after your experiments by deleting or
compressing such files. Simple AWK scripts are provided which
show how one might use such a file to plot a graph of average or
best fitness, in one or all populations, using xgraph.

PGA allows you to have multiple populations of chromosomes which
`evolve' separately, with occasional migration of a chromosome
from one population to all the others. Migration is controlled
by the \verb|-M| option, which selects the interval (in generations)
between migrations. At each migration, a chromosome is chosen
from one population (the first population is chosen for the first
migration, the second for the second and so on) according to the
current selection procedure and copied into all the other populations.
The populations remain fixed in size, so this insertion causes the
least fit in a population to be destroyed. A migration interval of
zero means that no migrations ever happen.

The selection procedure is determined by the \verb|-s| option. There
are four choices. The \verb|fitprop| option causes chromosomes in a
population to be selected for reproduction or migration in proportion
to their fitness. Obviously, the effect of this will depend on whether
fitness values lie in the range 0 to 100 (say) or 1000 to 1100 (say).
To counter this occasionally undesirable dependence on the range you
can select the \verb|rank| procedure instead. This puts the
chromosomes in rank ordering of fitness and then selects from that
ordered set with a probability proportional to ranking. If the bias,
as set by \verb|-b|, is 1.0 then the probability is a linear function
of the ranking (first-ranked is most-favoured, and it behaves as
\verb|fitprop| would if the rankings were the actual fitnesses); a
bias of 2.0 favours the first-ranked markedly more strongly, and the
probability distribution is essentially parabolic. Bias values between
1.0 and 2.0 vary the distribution smoothly between these extremes.
Tournament selection can be chosen using the \verb|tnN| option, where
\verb|N| is some positive integer. In tournament selection, \verb|N|
chromosomes are chosen uniformly randomly (that is, irrespective of
fitness; and a chromosome could be chosen more than once), and the
fittest of these is selected. For large \verb|N| this process can
be expensive, so a modified form of tournament selection is also
available using the \verb|tmN| option. In this, one chromosome is
chosen at random, and then up to \verb|N| tries are made at random
to pick another which is fitter than the first. The first such that is
found is the selected chromosome; if none is found, the first-chosen
is selected. This procedure is modelled on the classic `marriage
problem' in dynamic programming textbooks in mathematics.

At each generation, reproduction occurs in each population. There are
four methods of reproduction, set by the \verb|-r| option.  The
\verb|one| method selects two chromosomes according to the current
selection procedure, performs two-point crossover on them to obtain a
child or two, perhaps mutate it/them as well, and installs the result back
in that population; the least fit of the population is destroyed.
Mutation consist of randomly altering each bit with a probability
determined by the \verb|-m| option, or adaptively if the \verb|-a|
option is used. Adaptive mutation happens with a probability governed
by the degree of similarity of the parents -- the probability would be
0.0 for two totally dissimilar parents, running up to the value
specified by \verb|-m| if the parents are identical.

In \verb|gen| reproduction, the whole of a population is potentially
replaced at each generation. The procedure is to loop N times where N is
the population size, selecting two chromosomes each time according to
the current selection procedure. These are crossed with a probability
governed by the \verb|-c| option. If crossover didn't happen the parent
is just retained. If crossover happened the child or children are also
mutated with probability determined by the \verb|-m| option and the
adaptive mutation flag. The resulting chromosomes are inserted back into
the population, the least fit being destroyed as usual.

There are also two spatially-oriented reproduction operators,
\verb|ssoneN| and \verb|ssgenN| when $N$ is a positive integer. In these
two, the populations are not sorted; instead they are regarded as a
two-dimensional grid, roughly square (the population size is adjusted if
necessary to ensure a rectangle, and the dimensions of the rectangle are
shown alongside the population size in the output). This grid is also
toroidal: leave one edge and you reappear at the opposite edge. In
either, the chromosome at a given location is replaced by the child of
two parents, each found by seeking the fittest on a random walk of
length $N$ starting at that location. Replacement only happens if that
child is fitter than the chromosome already there. In \verb|ssoneN| a
single random location is chosen at each generation. In \verb|ssgenN| a
complete replacement population is constructed by doing this for every
location; the newly-constructed population then replaces the old one. If
\verb|-t| (see below) is also specified then the parents produce two children
and the second gets put in the location immediately `below' the first in
the grid. If either of these operators are used, the selection procedure
is forced to be this special spatial selection process, shown as
\verb|*spatial*| in the output; it is an error to specify something
different. These spatially-oriented reproduction operators are loosely
modelled on Wright's shifting-balance model of evolution, whereas all
the others are based loosely on Fisher's panmictic model.

The crossover procedure is selected by the \verb|-C| option. The
\verb|one| option selects one-point crossover: a random point along
the chromosome is selected, and the child contains a copy
of one parent up to that point and the other parent thereafter.
The \verb|two| option selects two-point crossover: two random
points are selected, and the child contains a copy of one parent
between these two points and of the other parent before and after
these points. The \verb|uniform| option selects uniform crossover:
at each point of the chromosome, a random choice is made between
the two parents as to which to copy at that point. There are
important differences bewteen these three kinds of crossover; to
see them, consider the unlikely event that two maximally dissimilar
parents are chosen (such as \verb|000..0| and \verb|111..1|), and 
that the chromosome length is $L$. With one-point crossover there
are only $2(L+1)$ possible children. With two-point crossover there
are $L(L+1)$ possible children. With uniform crossover any child
is possible, there are $2^L$ possibilities.

If the \verb|-t| flag is used, crossover produces two complementary
children rather than one; this is often claimed to be useful to retain
genetic diversity in the population.

It is possible to change the length of a chromosome by using the
\verb|-n| option. The default length is 32. The system will run slower
if you increase the length much; each chromosome is held as a
character array and long chromosomes therefore take longer to
manipulate and decode. The upper limit on chromosome length is
dictated by the amount of memory available and the `resolution' of the
chromosome decoding algorithm if the problem chosen is such that the
chromosome represents one or more numbers. For shortish chromosomes,
the decoding to two or more numbers is done using integer arithmetic
for speed; if you ask for long chromosomes floating point arithmetic
may be necessary, which further reduces the speed. The decoding
algorithm should cope with quite long chromosomes: each number could
occupy up to 1022 chromosome positions (bits).

An example of the system's display is shown in figure \ref{fig:output}.
\begin{figure}[bthp]
\begin{center}
\begin{verbatim}
(A)gain, (C)ontinue, (Q)uit: 
          Populations: 5            Chromosomes per pop: 50
                                      Chromosome length: 32
Generations per stage: 100                 Reproduction: one
   Reporting interval: 10                Crossover type: uniform, twins
   Migration interval: 10                Crossover rate: n/a
        Eval function: max                Mutation rate: 0.020000
            Selection: rank
       Selection bias: 1.50                  Generation: 100
                                     Evaluations so far: 500
             Pop.......Average..........Best.(max = 32)
             0          21.5200000       27.0000000
             1          20.7800000       24.0000000
             2          21.7800000       24.0000000
             3          22.1800000       25.0000000
             4          20.4400000       25.0000000
\end{verbatim}
\caption{Example of PGA screen display}
\label{fig:output}
\end{center}
\end{figure}
An `=' will appear to the right of the population number (`Pop') whenever
the average and best fitness values become equal.
The `Evaluations so far' shows the number of times that the problem
function has been called (after initialisation of the population(s))
to find the fitness of some chromosome; it gives some feel for the
work done by the genetic algorithm in finding the maximum fitness.
Note that there are very many more evaluations of the problem function
per generation in \verb|gen| reproduction than in \verb|one|
reproduction, so the display will be updated much more slowly. You may
wish to reduce the `generations per stage' (\verb|-l|) and `reporting
interval' (\verb|-i|) to compensate.

The program cycles for the number of generations shown against
`Generations per stage' (the \verb|-l| option), updating the
average/best information every `reporting interval' generations (the
\verb|-i| option). It then pauses, and asks you
\begin{verbatim}
 (A)gain, (C)ontinue, (Q)uit:
\end{verbatim}
Type `a' to start again with the same parameters, basically to see
if a random re-initialisation of the populations produces a
significantly different result this time. Type `c' to continue evolving
for further generations. Type `q' to quit. If you have specified an
output file too, you will be asked
\begin{verbatim}
 (A)gain, (C)ontinue, (S)ave+continue, (Q)uit:
\end{verbatim}
instead. Type `s' to append information about the chromosomes in a
file with extension `.chr' and then continue. The chromosomes are
printed out in population order. A typical line looks like:
\begin{verbatim}
3   17  100     25.0000000  1001110...1    179     0     0
\end{verbatim}
The first number is the population. The second is the number within
this population (ordered; fittest is first). The third is the generation
number when this line was written out. The fourth is the fitness, this
is followed by the chromosome itself. The last three numbers are used
to track a chromosome's parentage: each chromosome has a unique ID
number, starting from 1, and the first of these three numbers is this
chromosome's ID. The next two give the IDs of its parents. If both
are zero it is an aboriginal chromosome created at initialisation
time.

Note that the chromosome file, like the log file, can grow very fast.
With default settings, the file will increase in size by about 19k bytes
every time you press `s'! A separate document contains some basic ideas
for analysing populations using standard Unix tools.

PGA uses a random number generator that is seeded from the system
clock. However, the \verb|-S| option allows you to seed it with a
number of your choice so that you can repeat runs exactly. The initial
set of chromosomes will depend {\em only} on the seed value and the
\verb|-n|, \verb|-P| and \verb|-p| parameters. This allows you to run
precise experiments on the effects of changing other parameters.

\section*{Technical points}

As mentioned above, each chromosome is held as a character array so
it should easy to adapt the source to a good range of other problems.
Function pointers are used for such operations as selection,
reproduction and decoding in order to avoid the overhead of 
conditional switching at each cycle. A simple reference-counting scheme is
used to decide which chromosomes are still in use and which can be
freed, so memory usage is independent of the number of cycles and/or
number of restarts. This would be trivial were it not for the periodic
migration of chromosomes between populations; such migrating chromosomes
are not copied, for the sake of speed and memory turnover.

\end{document}
