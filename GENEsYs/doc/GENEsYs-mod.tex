%
%	GENEsYs-mod.tex		Ba, 22 feb 91		LaTeX 2.09
%				Ba, 15 jun 92
%
%	How to make modifications to GENESIS.
%{

\section{Inserting New Test Functions}\label{genesis-mod}
 
The current set of objective functions can easily be extended by the user.
The parameter list of objective functions is standardized according to 
the following two principal possibilities:
%
\begin{Enumerate}
%
\item	A pseudoboolean objective function is implemented.
	In this case, the arguments of the objective function are an 
	array of characters ('0' and '1', forming the bitstring) and
	an integer value giving the complete length of the bitstring.
%
\item	An objective function mapping continuous, real-valued object 
	variables to fitness values is implemented.
	In this case, the arguments of the objective function are
	an array of doubles and an integer value giving the dimension
	of the object variable vector.
%
\end{Enumerate}
%
Examples for both parameter lists and formulations of the objective 
function are given in the following source code.

\vskip1.0\baselineskip
\noindent\rule{\textwidth}{0.1pt}

\begin{scriptsize}
%
% This file was automatically produced at Mon Jun 15 17:22:25 1992 by
% c2latex f_01.c
%
\expandafter\ifx\csname indentation\endcsname\relax
\newlength{\indentation}\fi
\setlength{\indentation}{0.5em}
\begin{flushleft}
{\it /$\ast$$\ast$$\ast$$\ast$$\ast$$\ast$$\ast$$\ast$$\ast$$\ast$$\ast$$\ast$$\ast$$\ast$$\ast$$\ast$$\ast$$\ast$$\ast$$\ast$$\ast$$\ast$$\ast$$\ast$$\ast$$\ast$$\ast$$\ast$$\ast$$\ast$$\ast$$\ast$$\ast$$\ast$$\ast$$\ast$$\ast$$\ast$$\ast$$\ast$$\ast$$\ast$$\ast$$\ast$$\ast$$\ast$$\ast$$\ast$$\ast$$\ast$$\ast$$\ast$$\ast$$\ast$$\ast$$\ast$$\ast$$\ast$$\ast$$\ast$$\ast$$\ast$$\ast$$\ast$\mbox{}\\
\hspace*{1\indentation}$\ast$\mbox{}\\
\hspace*{1\indentation}$\ast$\hspace*{2\indentation}Copyright\hspace*{1\indentation}(c)\hspace*{1\indentation}1990\mbox{}\\
\hspace*{1\indentation}$\ast$\hspace*{2\indentation}Thomas\hspace*{1\indentation}Baeck\mbox{}\\
\hspace*{1\indentation}$\ast$\hspace*{2\indentation}Computer\hspace*{1\indentation}Science\hspace*{1\indentation}Department,\hspace*{1\indentation}LSXI\mbox{}\\
\hspace*{1\indentation}$\ast$\hspace*{2\indentation}University\hspace*{1\indentation}of\hspace*{1\indentation}Dortmund\mbox{}\\
\hspace*{1\indentation}$\ast$\hspace*{2\indentation}Baroper\hspace*{1\indentation}Str.\hspace*{1\indentation}301\mbox{}\\
\hspace*{1\indentation}$\ast$\hspace*{2\indentation}D-4600\hspace*{1\indentation}Dortmund\hspace*{1\indentation}50\mbox{}\\
\hspace*{1\indentation}$\ast$\mbox{}\\
\hspace*{1\indentation}$\ast$\hspace*{2\indentation}e-mail:\hspace*{1\indentation}baeck@ls11.informatik.uni-dortmund.de\mbox{}\\
\hspace*{1\indentation}$\ast$\mbox{}\\
\hspace*{1\indentation}$\ast$\hspace*{2\indentation}Permission\hspace*{1\indentation}is\hspace*{1\indentation}hereby\hspace*{1\indentation}granted\hspace*{1\indentation}to\hspace*{1\indentation}copy\hspace*{1\indentation}all\hspace*{1\indentation}or\hspace*{1\indentation}any\hspace*{1\indentation}part\hspace*{1\indentation}of\mbox{}\\
\hspace*{1\indentation}$\ast$\hspace*{2\indentation}this\hspace*{1\indentation}program\hspace*{1\indentation}for\hspace*{1\indentation}free\hspace*{1\indentation}distribution.\hspace*{3\indentation}The\hspace*{1\indentation}author's\hspace*{1\indentation}name\mbox{}\\
\hspace*{1\indentation}$\ast$\hspace*{2\indentation}and\hspace*{1\indentation}this\hspace*{1\indentation}copyright\hspace*{1\indentation}notice\hspace*{1\indentation}must\hspace*{1\indentation}be\hspace*{1\indentation}included\hspace*{1\indentation}in\hspace*{1\indentation}any\hspace*{1\indentation}copy.\mbox{}\\
\hspace*{1\indentation}$\ast$\mbox{}\\
\hspace*{1\indentation}$\ast$$\ast$$\ast$$\ast$$\ast$$\ast$$\ast$$\ast$$\ast$$\ast$$\ast$$\ast$$\ast$$\ast$$\ast$$\ast$$\ast$$\ast$$\ast$$\ast$$\ast$$\ast$$\ast$$\ast$$\ast$$\ast$$\ast$$\ast$$\ast$$\ast$$\ast$$\ast$$\ast$$\ast$$\ast$$\ast$$\ast$$\ast$$\ast$$\ast$$\ast$$\ast$$\ast$$\ast$$\ast$$\ast$$\ast$$\ast$$\ast$$\ast$$\ast$$\ast$$\ast$$\ast$$\ast$$\ast$$\ast$$\ast$$\ast$$\ast$$\ast$$\ast$$\ast$$\ast$/}\mbox{}\\
\mbox{}\\
{\it /$\ast$\mbox{}\\
\hspace*{1\indentation}$\ast$\hspace*{8\indentation}file:\hspace*{8\indentation}f\_1.c\mbox{}\\
\hspace*{1\indentation}$\ast$\mbox{}\\
\hspace*{1\indentation}$\ast$\hspace*{4\indentation}author:\hspace*{8\indentation}Thomas\hspace*{1\indentation}Baeck\mbox{}\\
\hspace*{1\indentation}$\ast$\mbox{}\\
\hspace*{1\indentation}$\ast$\hspace*{3\indentation}purpose:\hspace*{3\indentation}objective\hspace*{1\indentation}function\hspace*{1\indentation}f\_1,\hspace*{1\indentation}the\hspace*{1\indentation}sphere\hspace*{1\indentation}model.\mbox{}\\
\hspace*{1\indentation}$\ast$\mbox{}\\
\hspace*{1\indentation}$\ast$$\ast$TeX\mbox{}\\
\hspace*{1\indentation}$\ast$\hspace*{16\indentation}$\backslash$Fct\{Sphere\hspace*{1\indentation}model\}\{Jon75\}\mbox{}\\
\hspace*{1\indentation}$\ast$\hspace*{16\indentation}$\backslash$Expr\{f\_1($\backslash$vec\{x\})\hspace*{1\indentation}$=$\hspace*{1\indentation}$\backslash$sum\_\{i$=$1\}$\wedge$n\hspace*{1\indentation}x\_i$\wedge$2\}\mbox{}\\
\hspace*{1\indentation}$\ast$\hspace*{16\indentation}$\backslash$begin\{Cst\}\mbox{}\\
\hspace*{1\indentation}$\ast$\hspace*{24\indentation}-5.12\hspace*{1\indentation}$\backslash$leq\hspace*{1\indentation}x\_i\hspace*{1\indentation}$\backslash$leq\hspace*{1\indentation}5.12\mbox{}\\
\hspace*{1\indentation}$\ast$\hspace*{16\indentation}$\backslash$end\{Cst\}\mbox{}\\
\hspace*{1\indentation}$\ast$\hspace*{16\indentation}$\backslash$Min\{$\backslash$min(f\_1)\hspace*{1\indentation}$=$\hspace*{1\indentation}f\_1(0,$\backslash$ldots,0)\hspace*{1\indentation}$=$\hspace*{1\indentation}0\}\mbox{}\\
\hspace*{1\indentation}$\ast$$\ast$TeX\mbox{}\\
\hspace*{1\indentation}$\ast$\mbox{}\\
\hspace*{1\indentation}$\ast$\hspace*{8\indentation}\$Id\$\mbox{}\\
\hspace*{1\indentation}$\ast$\mbox{}\\
\hspace*{1\indentation}$\ast$/}\mbox{}\\
\mbox{}\\
\mbox{}\\
{\tt \#include} "{\tt ../extern.h}"\mbox{}\\
\mbox{}\\
{\bf double}\mbox{}\\
f\_01(x, n)\mbox{}\\
{\bf register} {\bf double}		x[];\mbox{}\\
{\bf register} {\bf int}		n;\mbox{}\\
\mbox{}\\
\mbox{}\\
\{\mbox{}\\
\hspace*{1\indentation}\mbox{}\\
\hspace*{8\indentation}{\bf register} {\bf int} 	i;\mbox{}\\
\hspace*{8\indentation}{\bf double} 		Sum;\mbox{}\\
\mbox{}\\
\hspace*{8\indentation}{\bf for} (Sum = 0.0, i = 0; i $<$ n; i++) \{\mbox{}\\
\hspace*{16\indentation}Sum += x[i]$\ast$x[i];	\mbox{}\\
\hspace*{8\indentation}\}\mbox{}\\
\mbox{}\\
\hspace*{8\indentation}{\bf return} (Sum);\mbox{}\\
\}\mbox{}\\
\mbox{}\\
{\it /$\ast$$\ast$\hspace*{1\indentation}end\hspace*{1\indentation}of\hspace*{1\indentation}file\hspace*{1\indentation}$\ast$$\ast$/}\mbox{}\\
\end{flushleft}

\end{scriptsize}

\noindent\rule{\textwidth}{0.1pt}

\begin{scriptsize}
%
% This file was automatically produced at Mon Jun 15 17:22:31 1992 by
% c2latex f_12.c
%
\expandafter\ifx\csname indentation\endcsname\relax
\newlength{\indentation}\fi
\setlength{\indentation}{0.5em}
\begin{flushleft}
{\it /$\ast$$\ast$$\ast$$\ast$$\ast$$\ast$$\ast$$\ast$$\ast$$\ast$$\ast$$\ast$$\ast$$\ast$$\ast$$\ast$$\ast$$\ast$$\ast$$\ast$$\ast$$\ast$$\ast$$\ast$$\ast$$\ast$$\ast$$\ast$$\ast$$\ast$$\ast$$\ast$$\ast$$\ast$$\ast$$\ast$$\ast$$\ast$$\ast$$\ast$$\ast$$\ast$$\ast$$\ast$$\ast$$\ast$$\ast$$\ast$$\ast$$\ast$$\ast$$\ast$$\ast$$\ast$$\ast$$\ast$$\ast$$\ast$$\ast$$\ast$$\ast$$\ast$$\ast$$\ast$\mbox{}\\
\hspace*{1\indentation}$\ast$\mbox{}\\
\hspace*{1\indentation}$\ast$\hspace*{2\indentation}Copyright\hspace*{1\indentation}(c)\hspace*{1\indentation}1991\mbox{}\\
\hspace*{1\indentation}$\ast$\hspace*{2\indentation}Thomas\hspace*{1\indentation}Baeck\mbox{}\\
\hspace*{1\indentation}$\ast$\hspace*{2\indentation}Computer\hspace*{1\indentation}Science\hspace*{1\indentation}Department,\hspace*{1\indentation}LSXI\mbox{}\\
\hspace*{1\indentation}$\ast$\hspace*{2\indentation}University\hspace*{1\indentation}of\hspace*{1\indentation}Dortmund\mbox{}\\
\hspace*{1\indentation}$\ast$\hspace*{2\indentation}Baroper\hspace*{1\indentation}Str.\hspace*{1\indentation}301\mbox{}\\
\hspace*{1\indentation}$\ast$\hspace*{2\indentation}D-4600\hspace*{1\indentation}Dortmund\hspace*{1\indentation}50\mbox{}\\
\hspace*{1\indentation}$\ast$\mbox{}\\
\hspace*{1\indentation}$\ast$\hspace*{2\indentation}e-mail:\hspace*{1\indentation}baeck@ls11.informatik.uni-dortmund.de\mbox{}\\
\hspace*{1\indentation}$\ast$\mbox{}\\
\hspace*{1\indentation}$\ast$\hspace*{2\indentation}Permission\hspace*{1\indentation}is\hspace*{1\indentation}hereby\hspace*{1\indentation}granted\hspace*{1\indentation}to\hspace*{1\indentation}copy\hspace*{1\indentation}all\hspace*{1\indentation}or\hspace*{1\indentation}any\hspace*{1\indentation}part\hspace*{1\indentation}of\mbox{}\\
\hspace*{1\indentation}$\ast$\hspace*{2\indentation}this\hspace*{1\indentation}program\hspace*{1\indentation}for\hspace*{1\indentation}free\hspace*{1\indentation}distribution.\hspace*{3\indentation}The\hspace*{1\indentation}author's\hspace*{1\indentation}name\mbox{}\\
\hspace*{1\indentation}$\ast$\hspace*{2\indentation}and\hspace*{1\indentation}this\hspace*{1\indentation}copyright\hspace*{1\indentation}notice\hspace*{1\indentation}must\hspace*{1\indentation}be\hspace*{1\indentation}included\hspace*{1\indentation}in\hspace*{1\indentation}any\hspace*{1\indentation}copy.\mbox{}\\
\hspace*{1\indentation}$\ast$\mbox{}\\
\hspace*{1\indentation}$\ast$$\ast$$\ast$$\ast$$\ast$$\ast$$\ast$$\ast$$\ast$$\ast$$\ast$$\ast$$\ast$$\ast$$\ast$$\ast$$\ast$$\ast$$\ast$$\ast$$\ast$$\ast$$\ast$$\ast$$\ast$$\ast$$\ast$$\ast$$\ast$$\ast$$\ast$$\ast$$\ast$$\ast$$\ast$$\ast$$\ast$$\ast$$\ast$$\ast$$\ast$$\ast$$\ast$$\ast$$\ast$$\ast$$\ast$$\ast$$\ast$$\ast$$\ast$$\ast$$\ast$$\ast$$\ast$$\ast$$\ast$$\ast$$\ast$$\ast$$\ast$$\ast$$\ast$$\ast$/}\mbox{}\\
\mbox{}\\
{\it /$\ast$\mbox{}\\
\hspace*{1\indentation}$\ast$\hspace*{8\indentation}file:\hspace*{8\indentation}f\_12.c\mbox{}\\
\hspace*{1\indentation}$\ast$\mbox{}\\
\hspace*{1\indentation}$\ast$\hspace*{4\indentation}author:\hspace*{8\indentation}Thomas\hspace*{1\indentation}Baeck,\hspace*{1\indentation}27.\hspace*{1\indentation}Feb.\hspace*{1\indentation}'91\mbox{}\\
\hspace*{1\indentation}$\ast$\mbox{}\\
\hspace*{1\indentation}$\ast$\hspace*{16\indentation}A\hspace*{1\indentation}simply\hspace*{1\indentation}binary\hspace*{1\indentation}testfunction\hspace*{1\indentation}(`mimikry\hspace*{1\indentation}problem').\mbox{}\\
\hspace*{1\indentation}$\ast$\hspace*{16\indentation}The\hspace*{1\indentation}global\hspace*{1\indentation}optimum\hspace*{1\indentation}is\hspace*{1\indentation}given\hspace*{1\indentation}by\hspace*{1\indentation}the\hspace*{1\indentation}bitstring\hspace*{1\indentation}(000000...000),\mbox{}\\
\hspace*{1\indentation}$\ast$\hspace*{16\indentation}the\hspace*{1\indentation}fitness\hspace*{1\indentation}function\hspace*{1\indentation}is\hspace*{1\indentation}simple\hspace*{1\indentation}defined\hspace*{1\indentation}as\hspace*{1\indentation}the\hspace*{1\indentation}Hamming\hspace*{1\indentation}distance\mbox{}\\
\hspace*{1\indentation}$\ast$\hspace*{16\indentation}to\hspace*{1\indentation}this\hspace*{1\indentation}optimum.\mbox{}\\
\hspace*{1\indentation}$\ast$\mbox{}\\
\hspace*{1\indentation}$\ast$$\ast$TeX\mbox{}\\
\hspace*{1\indentation}$\ast$\hspace*{8\indentation}$\backslash$Fct\{Hamming\hspace*{1\indentation}distance\hspace*{1\indentation}to$\sim$\$0$\wedge$l\$\}\{\}\mbox{}\\
\hspace*{1\indentation}$\ast$\hspace*{8\indentation}$\backslash$Expr\{f\_\{12\}($\backslash$vec\{a\})\hspace*{1\indentation}$=$\hspace*{1\indentation}$\backslash$sum\_\{i$=$1\}$\wedge$l\hspace*{1\indentation}$\backslash$alpha\_i\}\mbox{}\\
\hspace*{1\indentation}$\ast$\hspace*{8\indentation}$\backslash$Min\{$\backslash$min(f\_\{12\})\hspace*{1\indentation}$=$\hspace*{1\indentation}f\_\{12\}(0,$\backslash$ldots,0)\hspace*{1\indentation}$=$\hspace*{1\indentation}0\}\mbox{}\\
\hspace*{1\indentation}$\ast$$\ast$TeX\mbox{}\\
\hspace*{1\indentation}$\ast$\mbox{}\\
\hspace*{1\indentation}$\ast$\hspace*{8\indentation}\$Id\$\mbox{}\\
\hspace*{1\indentation}$\ast$\mbox{}\\
\hspace*{1\indentation}$\ast$\mbox{}\\
\hspace*{1\indentation}$\ast$/}\mbox{}\\
\mbox{}\\
{\tt \#include} "{\tt ../extern.h}"\mbox{}\\
\mbox{}\\
\mbox{}\\
{\bf double}\mbox{}\\
f\_12(x, Length)\mbox{}\\
{\bf register} {\bf char}		x[];\mbox{}\\
{\bf register} {\bf int}		Length;\mbox{}\\
\mbox{}\\
\{\mbox{}\\
\hspace*{1\indentation}\mbox{}\\
\hspace*{8\indentation}{\bf register} {\bf int} 	i, Dst;\mbox{}\\
\mbox{}\\
\hspace*{8\indentation}{\bf for} (Dst = 0, i = 0; i $<$ Length; i++) \{\mbox{}\\
\hspace*{16\indentation}Dst += (x[i]) ? 1 : 0;\mbox{}\\
\hspace*{8\indentation}\}\mbox{}\\
\mbox{}\\
\hspace*{8\indentation}{\bf return} (({\bf double}) Dst);\mbox{}\\
\}	 \mbox{}\\
\mbox{}\\
{\it /$\ast$$\ast$$\ast$\hspace*{1\indentation}end\hspace*{1\indentation}of\hspace*{1\indentation}file\hspace*{1\indentation}$\ast$$\ast$$\ast$/}\mbox{}\\
\end{flushleft}

\end{scriptsize}
\noindent\rule{\textwidth}{0.1pt}
\vskip1.0\baselineskip

\subsection{Automatic \LaTeX-Documentation}

In order to have the documentation up-to-date, when inserting a new 
objective function the user should also add a \LaTeX\ description of
the objective function to the source code, included in the comment 
header and bracketed by \verb/**TeX/ keywords.
The meaning of the macros provided to do so is as follows:
%
\begin{Itemize}
%
\item	\verb/\Fct{}{}/							\\
	The description of the objective function.
	First parameter: A short textual description of the objective 
	function, e.g.~the name of the function (if known).		\\
	Second parameter: A bibliographic key to an entry of the
	bibliography database file.
	This parameter may be left empty.
%
\item	\verb/\Expr{}/							\\
	The arithmetic expression which describes the objective function.
	Several repetitions of this macro may occur in the description.
%
\item	\verb/\begin{Cst}...\end{Cst}/					\\
	All constant definitions for the objective function are included
	in this environment.				
%
\item	\verb/\Min{}/							\\
	A specification of the minimum point and objective function
	value at the minimum point (if known).
%
\end{Itemize}

To have your documentation up-to-date, simply process
%
\begin{center}
	\verb/make dvi/
\end{center}
%
two times in the directory GENEsYs/doc.
This will automatically extract the \LaTeX-description from the 
function sources and create the updated documentation by running
\LaTeX\ on the \verb/.tex/-files.

\subsection{Function Table Entries}

In order to add a new objective function, the user should proceed
according to the following stepwise description:
%
\begin{Enumerate}
%
\item	Write an objective function source file according to the examples
	given above.
	The file containing the objective function should be named according to
	the pattern \File{f\_xx.c} where \File{xx} is the next free number for
	objective functions.
\ifUS
	The file must be placed in the directory GENEsYs/ src/fct.
\else
	The file must be placed in the directory GENEsYs/src/fct.
\fi
%
\item	In the file GENEsYs/src/fct/ftab.c, provide the following entries
	(here demonstrated by referring to an already existing example):
	%
	\begin{footnotesize}
	%
% This file was automatically produced at Tue Jun 16 08:48:47 1992 by
% c++2latex ft.c
%
\expandafter\ifx\csname indentation\endcsname\relax
\newlength{\indentation}\fi
\setlength{\indentation}{0.5em}
\begin{flushleft}
\mbox{}\\
{\bf extern} {\bf double}	f\_09();		\hfill{\it /$\ast$\hspace*{1\indentation}Ackley's\hspace*{1\indentation}multimodal\hspace*{1\indentation}ascending\hspace*{1\indentation}path\hspace*{8\indentation}$\ast$/}\mbox{}\\
\mbox{}\\
\mbox{}\\
\{20, VRBL, REAL, -32.768, 32.768, f\_09, {\tt "f\_09"},\mbox{}\\
{\tt "Ackley's multimodal descending path"},\mbox{}\\
\{20.0, 0.2, 2.0 $\ast$ M\_PI\},\mbox{}\\
\{{\tt "Factor A"}, {\tt "Factor B"}, {\tt "Factor C"}\},\mbox{}\\
\{\{0, \{({\bf char} $\ast$) NULL, ({\bf char} $\ast$) NULL, ({\bf char} $\ast$) NULL\}, ({\bf char} $\ast$) NULL\},\mbox{}\\
\hspace*{1\indentation}\{0, \{({\bf char} $\ast$) NULL, ({\bf char} $\ast$) NULL, ({\bf char} $\ast$) NULL\}, ({\bf char} $\ast$) NULL\},\mbox{}\\
\hspace*{1\indentation}\{0, \{({\bf char} $\ast$) NULL, ({\bf char} $\ast$) NULL, ({\bf char} $\ast$) NULL\}, ({\bf char} $\ast$) NULL\}\}\},\mbox{}\\
\end{flushleft}

	\end{footnotesize}
	%
	The first line simply declares your new function to be extern.
	More interesting is the entry you have to provide in the data
	structure \Vrb{f\_tab[]}, where the entries in the order of their
	appearance in the example have the following meaning:
	%
	\begin{Itemize}
	%
	\item	The default dimensionality of the objective function.
	%
	\item	A keyword which describes, whether the default dimensionality
		may be changed (\Vrb{VRBL}) or not (\Vrb{STRC}).
	%
	\item	A keyword which classifies the objective function as a
		pseudoboolean function (\Vrb{BINY}) or a continuous 
		function mapping real vectors to fitness values (\Vrb{REAL}).
	%
	\item	Default values for lower and upper bound of the object
		variables (for decoding purposes).
	%
	\item	The name of the objective function (i.e.~a pointer to the			function).
	%
	\item	The name of the file containing the objective function code,
		without the extension \File{.c}.
	%
	\item	A textual description of the objective function (for the
		online help).
	%
	\item	Up to three real-valued additional parameters, which can 
		during a call to the \GA\ directly be passed to the 
		objective function.
		The options for these values are defined to be 
		\O{-p}, \O{-q}, and \O{-r}, respectively, in
		the order of definition of the options.
		Here, these are the parameters~$a$, $b$, and $c$ of the 
		objective function by Ackley.
		Referring to section~\ref{genesis-fct}, the reader may easily
		identify these parameters in function number nine.
	%
	\item	For each of the three real-valued parameters, a short textual
		description for the online help may be given.
		Here, this reduces to ``Factor A'', ``Factor B'', and
		``Factor C''.
	%
	\item	Up to three discrete options, each having up to three
		different values, are given next.
		This can be used to pass certain flags to objective functions,
		if necessary.
		The options for these discrete values are defined to be
		\O{-s}, \O{-t}, and \O{-u}, respectively,
		in the order of definition of the options.
		For each of the three possible values, short textual 
		descriptions are allowed to be entered in the fields
		actually occupied by \Cmd{NULL} pointers.
		The first number in each of these three rows gives the 
		default value of the option.
		For an example on this kind of option handling, the user
		may have a look a \Vrb{f\_13()}.
	%
	\end{Itemize}
%
\item	In the makefile GENEsYs/src/Makefile, add the new object file name
	\File{f\_xx.o} to the list \Vrb{FUNS} of objective functions.
%
\item	In the directory GENEsYs/src, type \Cmd{make}.
%
\end{Enumerate}

%}
