%
%	GENEsYs-opt.tex		Ba, 22 feb 91		LaTeX 2.09
%				Ba, 15 jun 92
%
%	Informations how to run the GENESIS program
%{

\section{Options}\label{genesis-opt}

\GEN\ may be run from any directory by using one of two different 
possible methods.
The first one is based on a program called ``setup'', which explicitly
asks for each parameter of the \GA\ to be typed in by the user.
Default values are announced explicitly during this procedure by
printing them in square brackets.
This procedure is suggested for use by the unexperienced user of \GEN;
it is explained here first.

\subsection{The ``setup''-program}

Run ``setup'' which prompts for the following  parameters
(a  \Cmd{<cr>} response to any prompt gets the default value shown
in brackets):
%
\begin{description}
%
\item[	Objective Function Number {[1]}:]			
	This defines the objective function is to be used during the
	optimization.
	For a list of objective functions, see section~\ref{genesis-fct}.      	
%
\item[	Objective Function Dimension {[30]}:]
	The number of real-valued objective va\-ri\-a\-bles which have to be
	encoded in one individual.
	If the dimension specified here is~$0$, a pseudoboolean function is
	assumed, i.e.~a direct mapping of bitstrings to real values.
	Then, the meaning of the next parameter becomes different (see there).
%
\item[	Length per Object Variable {[32 or -1]}:]
	\ifUS
	\mbox{}\\
	\fi
	For real-valued object variables, the number of bits per object
	variable is specified here.
	For a pseudoboolean function, the total length of the bitstring has
	to be given.
%
\item[	Population Size {[50]}]{}
%
\item[ 	Reproduction Mechanism {[P]}:]
	The selection mechanism (default: proportional selection) is
	specified by this parameter.
	For an explanation of the different options, see 
	subsection~\ref{genesis-subopt}.
	Depending on different specifications given here by the user,
	some more information might be necessary. 
	The different options causing more information requests are explained
	here briefly:
	%
	\begin{Itemize}
	%
	\item[R:] {\bf Maximum Expected Value for Ranking {[1.1]}:}
		This value determines the maximum expected value. 
		It's default value is set to $\eta_{\max} = 1.1$ 
		according to Baker's suggestions~\cite{Bak85}. 
	%
	\item[B:] {\bf Temperature Control Parameter {[0.99]}:}
		The temperature control parameter for the cooling schedule
		in case of Boltzmann selection 
		(see section~\ref{genesis-pro}).			\\
		{\bf Constant Temp.~Cooling Interval (Gens) {[5]}:}
		The number of generations to remain at a constant temparature
		in case of Boltzmann selection (see section~\ref{genesis-pro}).
	%
	\item[W:] {\bf Value of Whitley's Constant 'a' {[1.100]}:}
		For Whitley's selection scheme (see section~\ref{genesis-pro}),
		the constant~$a$ is specified here.
	%
	\end{Itemize}
	%
%
\item[	Lower Selection Index {[1]}:]
	This index specifies the index of the first individual which 
	has to be taken into account by selection, when individuals are sorted
	by fitness.
	A default value of~$1$ determines to start at the best individual,
	$2$ at the second best, and so on.
	Only individuals having a rank equal to or larger than this parameter
	are considered during selection.
	Normally, you should not decide to use another than the default value.
%
\item[	Upper Selection Index {[50]}:]
	This is essentially the value of~$\mu$ as described in 
	section~\ref{genesis-pro}.
%
\item[	Mutation Mechanism {[S]}:]
	\ifUS
	\mbox{}\\
	\fi
	The mutation mechanism (default: simple mutation).
	For an explanation of the different options, see 
	subsection~\ref{genesis-subopt}.
	Again, different options can cause more information requests as
	follows:
	%
	\begin{Itemize}
	%
	\item[A:] {\bf Number of Adaptive Mutation Rates {[30]}:}
		For mutation according to the {\sc Amim} mechanism, the
		number of adaptive mutation rates must be given here.
		This is by default suggested to be identical to the number of
		object variables.
		Alternatively, it may take the value~$1$, which is a must
		in case of a pseudoboolean function to be optimized.	\\
		{\bf Number of Bits for each Mutation Rate {[20]}:}
		The number of bits used for encoding each of the adaptive
		mutation rates has to be given here.
		In experiments, the default value of $20$ turned out to be
		a reasonable choice.
	%
	\item[X:] {\bf Number of Mutation Rates {[30]}:}
		The same as for option {\bf A} is true for this option, except
		that the mutation mechanism is different ({\sc Amem}).
	%
	\end{Itemize}
%
\item[	Mutation Probability {[0.001]}]{}
%
\item[	Recombination Mechanism {[S\_]}:]
	Both for the object variables and for the adaptive mutation rates,
	a recombination mechanism must be specified here, resulting in a
	pair consisting of two letters.
	The default is standard recombination (two-point crossover) for object
	variables and no recombination for mutation rates (since by default
	no adaptive mutation rates are enabled).
	For a description of available options, see 
	subsection~\ref{genesis-subopt},
%
\item[	Number of Crossover Points {[2]}:]
	The user may arbitrarily choose the number of cross\-over--points 
	between the boundaries~$0$ and the length of the individuals.
	However, typically the number chosen will not exceed~$12$.
	The default value is~$2$, according to the definition in the 
	original \Ogen.
%
\item[	Crossover Application Rate {[0.60]}]{}
%
\item[	Generation Gap {[1.0]}:]
       	The generation gap indicates the percentage of  the  population 
	which is replaced in each generation.
%
\item[	Scaling Window Size {[5]}:]
       	When minimizing a numerical function with a GA, it is common
       	to  define  the  performance  value $u(x)$ of a structure $x$ as
       	$u(x) = f_{max} - f(x)$, where $f_{max}$ is the maximum  value  that
       	$f(x)$ can assume in the given search space.  This transformation 
	guarantees  that  the  performance  $u(x)$  is  positive,
       	regardless  of the characteristics of $f(x)$.  Often, $f_{max}$ is
       	not available a priori, in which case we may define  $u(x)  =
       	f(x_{max})  - f(x)$, where $f(x_{max})$ is the maximum value of any
       	structure evaluated so far.  Either definition of  $u(x)$  has
       	the  unfortunate  effect  of making good values of $x$ hard to
       	distinguish.  For  example,  suppose  $f_{max}  =  100$.   After
       	several  generations,  the  current population might contain
       	only structures $x$ for which $5 < f(x) < 10$.  At  this  point,
      	no structure in the population has a performance which deviates 
	much from the  average.   This  reduces  the  selection
       	pressure  toward the better structures, and the search stagnates.  
	One solution is to define a new parameter $F_{max}$ with
       	a  value  of,  say, $15$, and rate each structure against this
       	standard. For example, if $f(x_i) = 5$ and  $f(x_j)  =  10$,  then
       	$u(x_i)  =  F_{max} - f(x_i) = 10$, and $u(x_j) = F_{max} - f(x_j) = 5$;
       	the performance of $x_i$ now appears to be twice as good as the
       	performance  of $x_j$.  The scaling window $W$ allows the user to
       	control how often the baseline performance is updated.  If $W > 0$
	then the system sets $F_{max}$ to the greatest value of $f(x)$
       	which has occurred in the last $W$ generations.  A value of  $W =  0$
  	indicates  an  infinite  window  (i.e. $u(x) = f(x_{max}) -f(x)$).
	
	The scaling mechanism indiciated here is necessary only in case of
	proportional selection.
%
\item[	Number of Experiments to Perform {[1]}:]
	This parameter defines the number of independent experiments 
	to perform for the same function.
%
\item[	Number of Trials per Experiment {[1000]}]
%
\item[	Report Interval, Evaluations {[100]}:]
	This parameter specifies the number of function evaluations
	between data collections.
%
\item[	Number of Structures to Save {[1]}:]
	\ifUS
	\mbox{}\\
	\fi
	The number of best structures which should be saved for each 
	experiment.
%
\item[	Maximum No.~of Gens.~w/o Evaluation {[2]}:] 	
	\ifUS
	\mbox{}\\
	\fi
	The maximum number of generations with\-out any evaluations occurring.
%
\item[ 	Interval for Bitmap Dumps (Gens) {[0]}:]
	When option \verb/b/ is passed to the algorithm, a bitmap dump
	containing the best structure of the populations is created.
	This parameter specifies the interval (in generations) for adding
	a new structure to the dump-file.
%
\item[	Interval for Population Dumps (Gens) {[0]}:]
	Similar to bitmap dumps, when option \verb/d/ is passed to the
	algorithm, a phenotype (object variable) dump of the whole populations
	is created.
	This parameter specifies the interval (in generations) for 
	creating a new dump of the population.
	Each population dump is written to a separate file, while all 
	bitmap dumps (since they create only one new structure each time)
	are appended to one file.
%
\item[	Internal \GA-Options {[an]}:]
	The special options which are passed to the \GA.
	These options are described in the section ``Further Options'' 
	(see \ref{genesis-subsubfurther}).
%
\item[	Seed for Random Number Generator {[123456789]}]
%
\item[	Suffix for \GA-Data Infile {[1.30.32.P50S0.001S\_2.0.6]}:]
	The suffix for the directory and the data-file which will be 
	created by \GEN. 
	The default looks cryptic, because it includes all necessary
	information to identify the parameters of the \GA-run; in this case
	the number of the objective function ($1$), the dimension of the
	objective function ($30$), the number of bits per object 
	variable ($32$), the selection scheme (P), the population
	size ($50$), the mutation mechanism (S), the mutation rate ($0.001$),
	the recombination mechanism (S\_), the number of crossover 
	points ($2$), and the crossover rate ($0.6$).
% 
\item[	The number of bits which encode the mutation rate {[0]}:]
	\ifUS
	\mbox{}\\
	\fi
	By this value it is defined, how many bits of any individual specify
	the mutation rate for that individual.
	This takes any effect only if the option \O{A} or \O{X} is set.
%
\end{description}
%
Setup then echoes the input file, and prompts with ``Execute \GA\ ?''.  
A \Cmd{<cr>}  in  response to this question starts the programs running.  
Any other response queues the execution  command  in  file  ``\GA-queue''.
This  is useful if you want to edit the input file.  
The  programs run in background mode.

\ifUS
	\newpage
\else
\fi 
\subsection{Command Line Options}\label{genesis-subopt}

All options as described in the previous subsection can be given to the
algorithm also by specifying them on the command line when calling the 
program \verb/ga/.
Furthermore, there are some additional options which can be specified this way.
Generally, this way of passing options to the \GA\ will be more convenient
to the user, since the complete setup-procedure during which all possible
parameter and configuration options have to be passed is a longish way of
creating a run configuration.
Using command line options, defaults remain unchanged, and the user must only
specify whichever option he wants to change.
The following option description divides into three parts:
%
\begin{Itemize}
%
\item	\GA-specific options (\ref{genesis-subsubspecific}).
	This part describes all options necessary for specifying an
	algorithmic parameterization and configuration setting.
%
\item	General options (\ref{genesis-subsubgeneral}).
	A part which describes all options related to data collection,
	I/O-facilities, and random number generation, as far as these
	options require parameters.
%
\item	Further options (\ref{genesis-subsubfurther}).
	A part containing options related to the algorithm and data collection
	as far as they require no further arguments.
%
\item	Function options (\ref{genesis-subsubfunction}).
	In \GEN, it is possible to pass options directly to an objective
	function, when the program is called from the UNIX command line level.
	The general format of such a call is as follows:
	%
\ifUS
	\begin{flushleft}
     \verb/ga <some_options> -f <No.> { <options_to_function> }/	\\
     \verb/   <more_options>/
	\end{flushleft}
\else
	\begin{center}
     \verb/ga <some_options> -f <No.> { <options_to_function> } <more_options>/
	\end{center}
\fi
	%
	Here, the options in curly brackets are function-specific options.
	It is important to note that the curly-bracketed function-specific
	options must directly follow the function number specification,
	as shown in the example.
	Blanks are mandatory between the curly brackets and 
	\verb/<options_to_function>/.
	In section~\ref{genesis-subsubfunction} the possible options 
	which may be passed to a function are explained.
%
\end{Itemize}

The general outline of the option description is a follows:
For each option, the default value as well as the range of values allowed
for the parameter setting are specified in the first line of the description.	
Then, the option is described informally, as also done in the preceding
description of the setup-procedure.

\subsubsection{\GA-Specific Options}\label{genesis-subsubspecific}

\begin{Enumerate}
%
\item	\Opt{-A}{$n_p$}{$0$}{$n_p \in \{0,1,n\}$}
	The number of adaptive mutation rates to be added to the genotype.
	For a pseudoboolean function, this can only be~$0$ or~$1$.
	Otherwise, it can also attain the value~$n$ (dimension of the objective
	function).
%
\item	\Opt{-B}{$l_m$}{$0$}{$0 \leq l_m \leq I_s$}			
	The number of bits which are used to encode one adaptive mutation
	rate. (Note that this default value is different to the default value
	used in the setup program. 
	This is due to the fact that in the setup-program this value is
	only being asked for if adaptive mutations are selected by the user.)
%
\item	\Opt{-C}{$p_c$}{$0.6$}{$0 \leq p_c \leq 1.0$}		
	The {\em crossover rate\/}~$p_c$.
%
\item	\Opt{-E}{}{}{}
	The {\em elitist\/} selection strategy is used in case of 
	option~\O{E}.
	The elitist selection strategy stipulates that the best performing
	structure always survives from one generation to the next.
	In the absence of this strategy, it is possible that the best
	structure disappears thanks to crossover or mutation.
%
\item	\Opt{-G}{$G$}{$1.0$}{$0 \leq G \leq 1.0$}
	The {\em generation gap\/}~$G$ indicates the percentage of the
	population which is replaced in each generation.
%
\item	\Opt{-L}{$l$}{$32$}{$0 < l$}
	The number of bits used for encoding one object variable~$x_i$
	in case of a continuous optimization problem (total length of the
	genotype is then~$l \cdot n$).
	In case of a pseudoboolean function, $l$ denotes the length of the
	genotype.
%
\item	\Opt{-M}{\verb/<MttScm>/}{\verb/S/}{\verb/S,A,X/}
	This option determines the mutation scheme which is used within
	the \GA. 
	The choices implemented in the current version of \GEN\ are
	(see section~\ref{genesis-pro} for details):			\\
	%
	\begin{tabular}{ll}
		\verb/S/ & Standard mutation mechanism.			\\
		\verb/A/ & {\sc Amim} adaptive mutation mechanism.	\\
		\verb/X/ & {\sc Amem} adaptive mutation mechanism.	\\
	\end{tabular}
%
\item	\Opt{-N}{$\eta_{max}$}{$1.1$}{$1.0 \leq \eta_{max} \leq 2.0$}
	\vspace*{-1.0\baselineskip}
	\Opt{-N}{$a$}{$1.1$}{$a \neq 1$}
	\vspace*{-1.0\baselineskip}
	For the {\em ranking\/} selection scheme according to 
	Baker~\cite{Bak85} the value of~$\eta_{max}$ is given by this option.
	This is equivalent to the value of~$a$ in Whitley's selection
	mechanism~\cite{Whi89}, but the latter provides a greater flexibility
	concerning the variability of the parameter.
%
\item	\Opt{-P}{$\lambda$}{$50$}{$0 < \lambda$}
	Population size.
%
\item	\Opt{-R}{$p_m$}{$0.001$}{$0 \leq p_m \leq 1.0$}
	The mutation probability is given by~$p_m$.
%
\item	\Opt{-S}{\verb/<SltScm>/}{\verb/P/}{\verb/P,B,R,I,W,M,C/}
	The selection scheme used within the \GA. 
	Possible choices are (see section~\ref{genesis-pro} for details): \\
	%
	\begin{tabular}{ll}
\ifUS
		\verb/P/ & 	Proportional selection, i.e.~the standard 
				selection 				\\
			 &	scheme~\cite{Hol75}.		\\
\else
		\verb/P/ & 	Proportional selection, i.e.~the standard 
				selection scheme~\cite{Hol75}.		\\
\fi
		\verb/B/ & 	Boltzmann selection~\cite{Gol90c}.	\\
		\verb/R/ & 	Ranking according to Baker~\cite{Bak85}.\\
		\verb/I/ & 	Inverse ranking.			\\
		\verb/W/ & 	Ranking according to 
				Whitley~\cite{Whi89}.			\\
		\verb/M/ & 	\mlSel. 				\\
\ifUS
		\verb/C/ & 	\mlSel, but with copying the 
				best~$\mu$ individuals			\\
			 &	completely, i.e.~modeling the		
			  	deterministic mechanism as used 	\\
			 &	in \ESs~\cite{Schw81}.			\\
\else
		\verb/C/ & 	\mlSel, but with copying the 
				best~$\mu$ individuals completely,	\\
			 &	i.e.~modeling the deterministic 
				mechanism as used in \ESs~\cite{Schw81}.\\
\fi
	\end{tabular}
%
\item	\Opt{-U}{$\mu$}{$50$}{$0 < \mu < \lambda$}
	Value of~$\mu$.
%
\item	\Opt{-V}{$\rho$}{$1$}{$0 < \rho < \mu$}
	The lower selection index of the first individual which is taken into
	account by selection.
%
\item	\Opt{-W}{$W$}{$5$}{$0 \leq W$}
	Windowsize for the special scaling mechanism used by Grefenstette
	in case of proportional selection.
%
\item	\Opt{-X}{$n_c$}{$2$}{$0 \leq n_c < l$}
	The number of crossover points for the crossover operator.
%
\item	\Opt{-Y}{\verb/<RecScm>/}{\verb/S\_/}{\verb/S,A,U,D,I,R,\_/}
	The recombination operator mechanism actually used. 
	Possible choices are (see section~\ref{genesis-pro} for details): \\
	%
	\begin{tabular}{ll}
\ifUS
		\verb/S/ & 	Standard recombination, i.e.~$m$--point 
				crossover.  				\\
			 &	Default: $m = 2$.		\\
\else
		\verb/S/ & 	Standard recombination, i.e.~$m$--point 
				crossover.  Default: $m = 2$.		\\
\fi
		\verb/U/ &	Uniform Crossover.			\\
		\verb/D/ & 	Discrete recombination.			\\
		\verb/I/ & 	Intermediate recombination.		\\
		\verb/R/ & 	Random intermediate recombination.	\\
		\verb/_/ & 	No recombination.			\\
	\end{tabular}

	To indicate the recombination mechanism, two letters {\em must\/} be
	specified in the command line, i.e.~one recombination scheme for
	the object variables and one for the mutation rates (which is
	\verb/_/ if no adaptive mutation rates are used).
%
\end{Enumerate}

\subsubsection{General Options}\label{genesis-subsubgeneral}
		
\begin{Enumerate}
%
\item	\Opt{-b}{\verb/<Bitmap-freq>/}{$0$}{$0 \leq \mbox{\tt<Bitmap-freq>}$}
	The bitmap dump interval, i.e.~the number of generations between
	adding the best individual bit pattern to the bitmap dump file.
%
\item	\Opt{-d}{\verb/<Dump-freq>/}{$0$}{$0 \leq \mbox{\tt<Dump-freq>}$}
	The dump interval, i.e.~the number of generations between dumps.
%
\item	\Opt{-e}{\verb/<Totalexp>/}{$1$}{$1 \leq \mbox{\tt<Totalexp>}$}
	The number of independent optimizations of the same function.
%
\item	\Opt{-f}{\verb/<F-nbr>/}{$1$}{$1 \leq \mbox{\tt<F-nbr>} \leq \NbrFct$}
	The index of the actual objective function to optimize.
	See section~\ref{genesis-fct} for an description of the 
	functons actually provided, and section~\ref{genesis-mod} 
	for an explanation of how to implement your own objective functions.
	Note that the upper bound on the number of objective functions does
	of course depend on the number of objective functions enclosed
	in your modified implementation.
%
\item	\Opt{-g}{\verb/<Maxspin>/}{$2$}{$0 \leq \mbox{\tt<Maxspin>}$}
	The number of generations without any evaluations occurring.
%
\item	\Opt{-h}{}{}{}
	The online-help information of the \GEN\ software.
	Besides a short explanation of all available options, this will
	also give you an overview of objective functions actually installed
	in the system as well as the additional options you can submit
	to the objective functions (see section~\ref{genesis-subsubfunction}).
%
\item	\Opt{-i}{\verb/<Infile>/}{\verb/in.<suf>/}{}
	A filename of the \File{in}-file to be used for this run of the \GA.
	If the option \O{-i} is set, no other option is allowed due
	to the fact, that all data is provided by the \File{in}-file.
	Hence, by a call \Cmd{ga -i in.<suf>} a setup-data file can
	be used later to start a run.
%
\item	\Opt{-n}{$n$}{$30$}{$1 \leq n$}
	The dimension of the objective function.
	If the function chosen from the function set is a pseudoboolean one,
	you may not use this option, because in this case the length of
	the individuals is determined by the \O{-L} option solely.
%
\ifUS
\item	\Opt{-o}{\verb/<Options>/}{\verb/anc/}{\verb/a,c,e,i,l,m,n,p/}
	\vspace*{-1.0\baselineskip}
	\Opt{}{}{}{\verb/s,t,v,z/}
	\vskip-1.0\baselineskip
\else
\item	\Opt{-o}{\verb/<Options>/}{\verb/anc/}{\verb/a,c,e,i,l,m,n,p,s,t,v,z/}
\fi
	Additional options for the \GA.
	These are exactly identical to the options described in 
	section~\ref{genesis-subsubfurther}, which can be set during 
	the ``setup'' procedure.
%
\item	\Opt{-r}{\verb/<Seed>/}{$123456789$}{$0 < \mbox{\tt<Seed>}$}
	Seed for the random number generator.
%
\item	\Opt{-s}{\verb/<Savesize>/}{$10$}{$0 \leq \mbox{\tt<Savesize>}$}
	How many of the best structures should be saved.
%
\item	\Opt{-t}{\verb/<Totaltrials>/}{$1000$}{$0 < \mbox{\tt<Totaltrials>}$}
	The total number of function evaluations per experiment.
%
\item	\Opt{-v}{\verb/<Interval>/}{$100$}{$1 < \mbox{\tt<Interval>}$}
	The number of trials between data collections.
%
\item	\Opt{-x}{\verb/<Suffix>/}{}{}
	The suffix for files created by \GEN.	
%
\end{Enumerate}

\subsubsection{Further Options}\label{genesis-subsubfurther}

Options described in this section are combined to form a string which is
passed to the \GA\ by using option \O{-o <options>}, or given to the 
setup-procedure answering the question for ``Internal \GA-Options''.

\begin{Enumerate}
%
\item	\Opt{a}{}{}{} 
	Evaluate all structures in each generation.   
	This may  be  useful  when  evaluating a noisy function, since it
       	allows the GA to sample a given structure several times.  
	Furthermore, it is sueful for comparison of different runs under
	consideration of the number of function evaluations.
	If this option is not selected then structures which are identical 
	to parents are not evaluated.
%
\item	\Opt{c}{}{}{}
	Collect statistics concerning the  convergence  of
       	the  algorithm.   
	These  statistics are written to the \File{out} file, 
	after every \Vrb{Interval} trials (option \O{-v}).   
	The  intervals are  approximate, since statistics are 
	collected only at the end of a generation.  
%
\item	\Opt{e}{}{}{}
	Use the elitist selection strategy.  
	The elitist strategy  stipulates  that  the  best  
	performing  structure always survives intact from one 
	generation to the next.   
	In the  absence  of this strategy, it is possible that 
	the best structure disappears, thanks to crossover or mutation.
%
\item	\Opt{i}{}{}{}
	Read structures into the initial population.   
	The initial  population  will  be read from the \File{init} file.  
	If the file  contains  fewer  structures  than  the  population
       	needs,  the  remaining  structures  will be initialized randomly.  
	Note: it is good practice to  allow  at  least  some
       	randomness in the initial population.
%
\item	\Opt{l}{}{}{}
	Dump the last generation to the \File{ckpt} file.  
%
\item	\Opt{m}{}{}{}
	Collect data about the mutation rate behaviour in case of adaptive
	mutations.
	For each generation and each objective variable, the corresponding
	average mutation rate, variance, and skewness of the mutation rate
	are collected in files called \File{mavg}, \File{mvar}, and 
	\File{mskw}, respectively.
	Hence, for a long run, a lot of data is collected by using
	this option.
%
\item	\Opt{n}{}{}{}
	The standard \GEN\ algorithm terminates, if either the
	given number of trials is reached, the number of converged alleles 
	reaches the length of the individuals, or the number of 
	generations since a individual had to be evaluated exceeds a 
	given maximum. 
	A futher criterion is a Bias greater than $0.99$. 
	This latter point may raise some problems, if the option \O{a} is 
	set, and may cause termination with useless data. 
	If option \O{N} is set, the algorithm only terminates, 
	if the given number of trials (i.e.~function evaluations) is reached. 
	The Bias condition is not respected then.
%
\item	\Opt{p}{}{}{}
	Dump the decoded generation into the file \File{pop}.
	This is done every \Vrb{Dump-freq} generations.
%
\item	\Opt{s}{}{}{}
	Trace the history of  one  schema.   
	This  option requires that a file named \File{schema} exist in 
	which the first line contains a string which has the length of 
	one structure and  which  contains  only  the characters 
	$0$, $1$, and $\#$ (and no blanks).  
	The system will append  one  line  to  the
       	schema file after each generation describing the performance
       	characteristics  of  the   indicated   schema   (number   of
       	representatives, relative fitness, etc.).
%
\item	\Opt{t}{}{}{}
	Trace each major function call  ({\bf for debugging}).
       	Tracing statements are written to the standard output.
%
\item	\Opt{v}{}{}{}
	Collect data about the object variable behaviour.
	For each generation and each objective variable, 
	average value of the object variable, its variance, and its skewness
	are collected in files called \File{vavg}, \File{vvar}, 
	and \File{vskw}, respectively.
	Hence, for a long run, a lot of data is collected by using
	this option.
%
\item	\Opt{z}{}{}{}
	Using this option, each \Vrb{Interval} (set by option \O{-v}
	from section~\ref{genesis-subsubgeneral}) function evaluations the
	complete set of selection probabilities in the population is
	dumped to a file \File{prb}.
%
\end{Enumerate} 

\subsubsection{Function Options}\label{genesis-subsubfunction}

In general, for each objective function up to eight additional option
settings may be specified.
For most objective functions, only two options make any sense, and for
pseudoboolean functions no options are available.
However, for some more complex user-defined functions continuous as well
as discrete option settings may be useful in addition.
In the most complex case, the following options may be used:
%
\begin{Itemize}
%
\item	\Opt{-L}{$x_{\min}$}{}{}
	The lower bound of the range of possible values each object variable
	may attain.
	Together with the option \O{-H}, these settings may be used to
	change the default range (see section~\ref{genesis-fct}) to which
	bitstring segments are mapped by the decoding function.
%
\item	\Opt{-H}{$x_{\max}$}{}{}
	The upper bound of the range of possible values each object 
	variable may attain.
%
\item	\Opt{-p,-q,-r}{}{}{}
	Three general options expecting real-valued arguments, each.
	The use of these options is completely due to the implementor of an
	objective function, who also has to specify a description of 
	the online-help.
	See section~\ref{genesis-mod} for further details.
%
\item	\Opt{-s,-t,-u}{}{}{}
	Three general options expecting an integer argument of 0,1 or 2, each.
	Again, use and documentation of these options is due to the implementor
	(see section~\ref{genesis-mod}).
%
\end{Itemize}

\subsection{Files}\label{genesis-subfiles}

According to the options set by the user, a smaller or larger number of
files is created in the directory which contains the data collected during
a run of the \GA.
The meaning of these files is described in this section.

\begin{enumerate}
%
\item	\File{ckpt}: 						\\
	A checkpoint file containing the current population.  
	This file is produced if the \O{d} option is set and given a positive
	argument and additionally option \O{p} is passed to the \GA.
%
\item	\File{in.<suffix>}: 					\\
	Contains input parameters and initial  random  number seeds.  
	The file suffix is either automatically created or specified by
	the user.
	The file is used for remembering all necessary parameterization
	information for a special run of the \GA.
%
\item	\File{init}: 						\\
	Contains structures which will be  included  in  the
       	initial  population.   
	This is useful if you have heuristics for selecting plausible 
	starting structures.  
	This  file  is read iff the option \O{i} is set
	(see also section~\ref{genesis-subsubfurther}).
%
\item	\File{log}: 						\\
	Logs start and finishing  time of a run as well as informations
	about the best individual contained in the last generation, its
	function value and the number of experiments performed.
%
\item	\File{pgm}:						\\
	Contains a bitmap dump of the best structure of generations,
	if option \O{b} is set and given a nonzero argument.
%
\item	\File{mavg}:						\\
	This file contains the associated average mutation rates 
	of each generation and object variable.
%
\item	\File{mskw}:						\\
	This file contains the skewness of the mutation rate distribution
	of each generation and object variable.
%
\item	\File{mvar}:						\\
	The variances of the mutation rate distribution of each generation
	and object variable.
%
\item	\File{min}: 						\\
	Contains the best structures found by the  GA.   
	The number  of elements in \File{min} is indicated by the response to
       	the ``save how many'' prompt during setup (option \O{s}).  
	If the  number  of experiments  is  greater  than  one, 
	the best structures are stored in \File{min.n} during experiment 
	number $n$.  
	This file  is produced if the number of saved structures is positive.
%
\item	\File{out}: 						\\
	Contains data describing the performance of  the  GA over all
	experiments indicated in the call.
%
\item	\File{pop}: 						\\
	If the option \O{p} is set, this file is created. 
	Every \Vrb{dump interval} generations the function values of 
	all members of the population are dumped to this file in a decoded way. 
	Lines are of the format
	%
	\begin{center}
	\begin{tabular}{rrrcrr}
		Generation	& $x_1$ & $x_2$ & $\ldots$ 
				& $x_N$ & $f(x_1,x_2,\ldots,x_N)$ \\
	\end{tabular}
	\end{center}
	%
	and include the generation number, the decoded object variables, 
	and the value of the objective function.
%
\item	\File{rep}: 						\\
	Summarizes  the  performance  of  the  algorithm.
       	This  file  is produced by the report program from the 
	\File{out} file. 
%
\item	\File{schema}: 						\\
	Logs a history of a single schema.  
	This file  is required for the \O{s} option.
%
\item	\File{vavg}:						\\
	The average object variable values of each generation and 
	each object variable.
%
\item	\File{vskw}:						\\
	The skewness of the distribution of object variables at each
	generation.
%
\item	\File{vvar}:						\\
	The variance of the object variable values at each generation.
%

\end{enumerate}

%}
