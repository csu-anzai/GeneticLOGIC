%
%	GENEsYs-int.tex		Ba, 22 feb 91		LaTeX 2.09
%
%	The general introduction to GENESIS
%{

\section{Introduction}\label{genesis-int}

This document describes the Genetic Algorithm test software \GEN, 
which is an extension of the original \Ogen\ program 
by John J.~Grefenstette~\cite{Gre87c}.
The extensions were developed in order to perform some experiments
with Genetic Algorithms~(GAs), and for testing some features which originally
stem from Evolution Strategies~\cite{Rec73,Schw81} (ESs) in the framework of
Genetic Algorithms.

The software described here has evolved over more than two years,
and the author disclaims any warranties of usefulness and fitness for
a particular problem. 
The code is a mixture of the original lines written by John and many
changes done by me.
Users who are experienced in working with the original \Ogen\
will notice similarities and differences quickly.
The software is written in C and runs under the 
UNIX\footnote{UNIX is a trademark of Bell Laboratories.} 
operating system.

In the following it is assumed that the reader is familiar with the
concept of a Genetic Algorithm or --- more generally --- an Evolutionary
Algorithm.
Should this not be the case, you might find it useful to have a look
at the book of David E.~Goldberg~\cite{Gol89b} or the original documentation
of \Ogen~\cite{Gre87c}.

Comparing \Ogen\ and \GEN, the following extensions implemented
in \GEN\ are most remarkably:
%
\begin{Itemize}
%
\item	Either command line options or the setup program can be used
	for invoking the GA.
%
\item	Enhanced data collection features are provided.
%
\item	A function table is used, from which the user choses an
	objective function when invoking the GA.
%
\item	Several extension of the basic GA are implemented, e.g.~$m$-point
	cross\-over~\cite{Jon75}, uniform crossover~\cite{Sys89},
	discrete and intermediate recombination \cite{BHS91}, adaptive
	mutation rates~\cite{Bae91b}, \cSelml-selection (proportional, ranking,
	linear)~\cite{BH91c}, Boltzmann selection~\cite{Gol90c}.
%
\end{Itemize}

The outline of this documentation is as follows:
In section~\ref{genesis-pro}, the major algorithmic extensions to a standard
\GA\ are briefly explained, mostly referring to the original 
publication sources.
Section~\ref{genesis-ist} describes, how the \GEN\ package is installed 
on your machine.
In section~\ref{genesis-opt}, the options of the program are presented.
Parameterization can either be performed by using the {\tt setup}-program,
or by specifying command-line options directly.
Furthermore, the most important information on the data-files created by
a run of \GEN\ is presented in that section.
Section~\ref{genesis-rep} describes the structure of the most important
file containing the extracted data from several runs under identical
parameter settings,  and in section~\ref{genesis-mod}
a hint is given how to extend the software by your own objective functions.
Finally, a list of objective functions actually included in the software
is presented in section~\ref{genesis-fct}.

\subsection{Warning}

\begin{sc}
The \GEN\ software was produced during a process of steadily changing and
extending its predecessor.
As a result, a reimplementation would have been necessary to concentrate
on essential parts, to refine implementation, and to add some bells and
whistles.
However, there was never time for me to do so, such that this plan 
was given up.
Please, be aware of some strangenesses when you are playing with the
software.
There is no warranty for this software\,!

Furthermore, the documentation is surely incomplete to cover all features
of the software sufficiently.
If something is unclear to you, try the software and don't trust the
documentation.
May be, the best way is to play with the software a little bit and look
what happens, which files are created, and so on.
\end{sc}

%}
